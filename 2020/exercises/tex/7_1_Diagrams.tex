\documentclass{article}
\usepackage{amsmath,mathtools,amssymb}
\usepackage{graphicx}
\usepackage{gensymb}
\usepackage{verbatim}
\usepackage{cancel}
\usepackage{mathrsfs}
\usepackage{bbm}
\usepackage{braket}
\usepackage{simplewick}
\usepackage{xcolor}
\definecolor{orange}{RGB}{255,127,0}
\usepackage{verbatim}
\usepackage[margin=1.0in]{geometry}
\newcommand{\ol}{\overline}
\newcommand{\ve}{\varepsilon}
\newcommand{\ap}{\ensuremath{a_p} }
\newcommand{\dg}{\ensuremath{^\dagger} }
\newcommand{\cd}{\ensuremath{\cdots} }
\newcommand{\apd}{\ensuremath{a_p^\dagger} }
\newcommand{\fctr}{\contraction}
\def\*#1{\mathbf{#1}}
\DeclarePairedDelimiter\floor{\lfloor}{\rfloor}

\title{Homework for Lecture 7.1 Wick's Theorem}
\date{}
\begin{document}
\maketitle
\noindent
\begin{enumerate}
\item Practice translating the following algebraic expressions into a diagrammatic form: 
\[ a_p^q\]
\[\tilde{a}_p^q \]
\[a_{pq}^{rs} \]
\[\tilde{a}_{pqr}^{stu} \]
\[\tilde{a}_i^a\]
\[\tilde{a}_a^i\]
\[\tilde{a}_{ab}^{ij}\]
\[\tilde{a}_{ijk}^{abc} \]
\[ \fctr{}{a}{{}_p }{a} a_p a_q\dg \]
\[ \fctr{}{a}{{}_p\dg }{a} a_p\dg a_q \]
\[ \fctr{}{a}{{}_p\dg }{a} a_p\dg a_q\dg \]
\[ \fctr{}{a}{{}_p }{a} a_p a_q \]
\item Practice translating the following matrix elements into diagrams and writing down the resulting fully contracted diagrams: 
\[ \braket{\Phi | h_p^q \tilde{a}_q^p|\Phi} \]
\[  \braket{\Phi | h_p^q \tilde{a}_q^p|\Phi_i^a} \]
\[  \braket{\Phi_j^b | h_p^q \tilde{a}_q^p|\Phi_i^a} \]
\[  \braket{\Phi | h_p^q \tilde{a}_q^p|\Phi_{ij}^{ab}} \]
\[  \braket{\Phi_{j}^{b}  |\Phi_{i}^{a}} \]
\end{enumerate}
\end{document}


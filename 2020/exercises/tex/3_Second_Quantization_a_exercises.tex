\documentclass{article}
\usepackage{amsmath,mathtools,amssymb}
\usepackage{graphicx}
\usepackage{gensymb}
\usepackage{verbatim}
\usepackage{mathrsfs}
\usepackage{bbm}
\usepackage{braket}
\usepackage[shortlabels]{enumitem}
\usepackage{verbatim}
\usepackage{cancel}
\usepackage[margin=1.0in]{geometry}
\newcommand{\ol}{\overline}
\newcommand{\ve}{\varepsilon}
\def\*#1{\mathbf{#1}}
\DeclarePairedDelimiter\floor{\lfloor}{\rfloor}

\title{Homework for Lecture 3.1 }
\date{}
\begin{document}
\maketitle
\noindent
\begin{enumerate}
\item Show that the number operator $N_p$ is hermitian ($N_p = N_p^\dagger$)
\item Prove the following anticommuntation relations:
\begin{enumerate}[(a)]
\item $\{a_p^\dagger, a_q\} = \delta_{pq}$ (Show this for three cases: $p = q$, $p < q$, $p > q$)
\item $\{a_p, a_q\} = 0$ (Show this for three cases: $p = q$, $p < q$, $p > q$)
\item $\{a_p^\dagger, a_q^\dagger\} = 0$ (Hint: show by forming Hermitian adjoint of 2b)
\end{enumerate}
\item Prove the six properties that are listed in Section 4.4 of the lecture notes using anticommuntation relations
\begin{itemize} 
\item \textit{Hint:} If you get stuck, write everything explicitly in terms of creation and annhilation operators and try applying anticommuntation relations to the middle of the expression.
\end{itemize}
\item Show that  $\braket{\Phi_{pq} | \Phi_{rs}} = \delta_{pr} \delta_{qs} - \delta_{ps} \delta_{qr}$
\begin{itemize} 
\item \textit{Hint 1:} Represent the expectation value in terms of strings of creation and annihilation operators (take care of the order of indices!)
\item \textit{Hint 2:} Apply anticommutation relations while moving annihilation operators to the right
\item Note: Another way to write $\delta_{pr} \delta_{qs} - \delta_{ps} \delta_{qr}$ is $\braket{p|r}\braket{q|s} - \braket{p|s}\braket{q|r}$. 
Note that this is just a Slater determinant of overlaps: 
\[
   \braket{\Phi_{pq} | \Phi_{rs}} =
  \left| {\begin{array}{cc}
   \braket{p|r} & \braket{p|s} \\
   \braket{q|r} & \braket{q|r} \\
  \end{array} } \right|
\]
In other words, an overlap of Slater determinants can be written as a Slater determinant of overlaps. 
\end{itemize}
\end{enumerate}
\end{document}


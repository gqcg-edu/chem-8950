\documentclass{article}
\usepackage{amsmath,mathtools,amssymb}
\usepackage{graphicx}
\usepackage{gensymb}
\usepackage{verbatim}
\usepackage{cancel}
\usepackage{mathrsfs}
\usepackage{bbm}
\usepackage{braket}
\usepackage{simplewick}
\usepackage{xcolor}
\definecolor{orange}{RGB}{255,127,0}
\usepackage{verbatim}
\usepackage[margin=1.0in]{geometry}
\newcommand{\ol}{\overline}
\newcommand{\ve}{\varepsilon}
\newcommand{\ap}{\ensuremath{a_p} }
\newcommand{\dg}{\ensuremath{^\dagger} }
\newcommand{\cd}{\ensuremath{\cdots} }
\newcommand{\Hc}{H_{\mathrm{c}}}
\newcommand{\apd}{\ensuremath{a_p^\dagger} }
\def\*#1{\mathbf{#1}}
\DeclarePairedDelimiter\floor{\lfloor}{\rfloor}

\title{Homework for Lecture 4, Configuration Interaction}
\date{}
\begin{document}
\maketitle
\noindent
\begin{enumerate}
\item Multiply both sides of the CI Schr{\"o}dinger equation on the left by
$\bra{\Phi_i^a}$ and simplify. What does the result tell you
about the CI coefficients of different excitations? What can
you infer about all higher order excitation coefficients?  

\item Suppose you have a system with 6 electrons, and 20 basis functions. How
many $\braket{\Phi_{m}^{e}|\Hc|\Phi_{i}^{a}}$ matrix elements are there
in the upper triangle of the full-CI matrix?
What about the $\braket{\Phi_{mn}^{ef}|\Hc|\Phi_{ijkl}^{abcd}}$ block?

\item Derive the CID coefficient equation from the notes
using KM notation (time consuming)

\item Explain why computing a potential energy surface with CISD
is a stupid idea. (hint: size consistency)
 
\end{enumerate}
\end{document}


\documentclass{article}
\usepackage{amsmath,mathtools,amssymb}
\usepackage{graphicx}
\usepackage{gensymb}
\usepackage{verbatim}
\usepackage{cancel}
\usepackage{mathrsfs}
\usepackage{bbm}
\usepackage{braket}
\usepackage{simplewick}
\usepackage{xcolor}
\definecolor{orange}{RGB}{255,127,0}
\usepackage{verbatim}
\usepackage[margin=1.0in]{geometry}
\newcommand{\ol}{\overline}
\newcommand{\fctr}{\contraction}
\newcommand{\ve}{\varepsilon}
\newcommand{\ap}{\ensuremath{a_p} }
\newcommand{\dg}{\ensuremath{^\dagger} }
\newcommand{\cd}{\ensuremath{\cdots} }
\newcommand{\apd}{\ensuremath{a_p^\dagger} }
\def\*#1{\mathbf{#1}}
\DeclarePairedDelimiter\floor{\lfloor}{\rfloor}

\title{Homework for Lecture 3.6 $a_p$ and $a_p\dg$ in the Particle-Hole Formalism}
\date{}
\begin{document}
\maketitle
\noindent
\begin{enumerate}
\item Prove:
 $$
 \fctr{}{a}{{}_p}{a}
 a_p a_q\dg = \pi(p)\delta_{pq} = \eta_{pq}
 $$
 $$
 \fctr{}{a}{{}_p}{a}
 a_p\dg a_q\dg = 0
 $$
\item Put the Hamiltonian in $\Phi$-Normal ordering, define the Fock operator
\item Prove Slater's first rule using the Particle-Hole formalism with a $\Phi$-Normal ordered Hamiltonian 
\[\braket{\Phi|H|\Phi} = \sum_i h_{ii} + \frac{1}{2} \sum_{ij} \braket{ij || ij} \]
\item Prove Slater's second rule using the Particle-Hole formalism with a $\Phi$-Normal ordered Hamiltonian 
\[\braket{\Phi|H|\Phi_i^a} = h_{ia} + \sum_{j} \braket{ij || aj} \]
\item Prove Slater's third rule using the Particle-Hole formalism with a $\Phi$-Normal ordered Hamiltonian 
\[\braket{\Phi|H|\Phi_{ij}^{ab}} = \braket{ij || ab} \]
\item Practice evaluating the following matrix elements: 
\[ \braket{\Phi_j^b |\Phi_{i}^{a}} \]
\[ \braket{\Phi_j^b |H| \Phi_{i}^{a}} \]
\[ \braket{\Phi_{kl}^{cd}|\Phi_{ij}^{ab}} \]
\[ \braket{\Phi_{kl}^{cd}|H|\Phi_{ij}^{ab}} \]
\end{enumerate}
\end{document}


\documentclass{article}
\usepackage{amsmath,mathtools,amssymb}
\usepackage{graphicx}
\usepackage{booktabs}
\usepackage{blkarray}
\usepackage{gensymb}
\usepackage{verbatim}
\usepackage{mathrsfs}
\usepackage{bbm}
\usepackage{braket}
\usepackage{hyperref}
\usepackage{verbatim}
\usepackage{cancel}
\usepackage[margin=1.0in]{geometry}
\newcommand{\ol}{\overline}
\newcommand{\lp}{\left(}
\newcommand{\rp}{\right)}
\newcommand{\eps}{\varepsilon}
\newcommand{\lam}{\lambda}
\newcommand{\h}{\circ}
\newcommand{\p}{\bullet}

\newcommand{\Ezero}{E^{(0)}}
\newcommand{\Phizero}{\Phi^{(0)}}
\newcommand{\Eone}{E^{(1)}}
\newcommand{\En}{E^{(n)}}
\newcommand{\Phione}{\Phi^{(1)}}
\newcommand{\Phin}{\Phi^{(n)}}

\newcommand{\Ecorr}{E_{\mathrm{corr}}}
\newcommand{\Hc}{H_{\mathrm{c}}}
\newcommand{\dg}{\ensuremath{^\dagger} }
\def\*#1{\mathbf{#1}}
\DeclarePairedDelimiter\floor{\lfloor}{\rfloor}

\title{Lecture 5: Perturbation Theory II}
\date{April 6, 2020}
\begin{document}
\maketitle
\noindent

In perturbation theory, we split our Hamiltonian into parts $H = H_0 + V$, and our ground 
state wavefunction and energy are expressed as an infinite series, where the first term is the unperturbed
quantity of some simpler model system described by $H_0$, and the rest of the terms
are corrections at various ``orders'':
\[\ket{\Psi_0} = \ket{\Phi_0} + \sum_{n=1}^{\infty} \ket{\Phin_0} \]
\[E_0 = \Ezero_0 + \sum_{n=1}^{\infty} \En_0\]

As we saw in the previous set of notes, the expression for each wavefunction and energy correction 
    needs to be derived from a system of equations.
This way of doing things is very inefficient at high orders, so in this set of notes 
    we will use a new scheme, which was primarily popularized and championed by Per-Olav L{\"o}wdin
    in his highly influential series of papers \textit{Studies in Perturbation Theory} in the 1960s.
The majority of the content in these notes can be found in the 9th (IX) paper in the series.
This new methodology will allow you to derive each order of PT using the same exact procedure. 
Though the higher order wavefunctions and energy expressions get more and more complex,
the procedure for deriving them does not change at higher orders.
This methodology also beautifully translates into clean diagrams, which we will see later on.

%\section{L{\"o}wdin and his Resolvents}
\section{The Wave Operator}
Suppose there was some magical operator $\Omega$ which
acts on the unperturbed state $\ket{\Phi_0}$ corresponding to $H_0$ 
and transforms it into the exact state $\ket{\Psi_0}$ described by $H_0 + V$
\[\ket{\Psi_0} = \Omega \ket{\Phi_0} \]
We will call $\Omega$ the \textit{wave operator}.
For now, we have no idea what it is, but we do know what it does.
Suppose further that we are using an intermediate normalized $\Psi_0$ and normalized $\Phi_0$ such that
\[ \braket{\Phi_0|\Phi_0} = 1 \]
\[ \braket{\Phi_0 | \Psi_0} = \braket{\Phi_0 | \Omega | \Phi_0} = 1 \]
Under the above conditions, we will now see what happens to the Schr{\"o}dinger equation,
\[(H_0 + V) \ket{\Psi_0} = E_0 \ket{\Psi_0} \]
Projecting both sides of this equation by $\bra{\Phi_0}$ gives
\[\braket{\Phi_0 | H_0 | \Psi_0} + \braket{\Phi_0 | V | \Psi_0} = E_0 \braket{\Phi_0|\Psi_0} \]
The first term reduces to $\Ezero_0$ by acting $H_0$ to the left. The second term 
can be re-expressed in terms of $\Omega$, and we obtain an expression for the 
perturbed eigenvalue $E_0$ in terms of a \textit{shift} from the unperturbed eigenvalue $\Ezero_0$.  
\[ E_0 = \Ezero_0 + \braket{\Phi_0 | V \Omega | \Phi_0}  \]
If we were to know what $\Omega$ is, at this point the eigenstates and eigenvalues would be completely
determined.
L{\"o}wdin derived that $\Omega$ is composed of generally determinable operator called a \textit{resolvent},
and under this definition, $\Omega$ yields the same perturbation expansion for $E_0$ and $\ket{\Psi_0}$.

\section{Resolvents}
A resolvent is a mathematical structure which is used to study the eigenspectrum
of operators.
The concept of a \textit{reduced resolvent} arises when one considers an eigenvalue problem \[A \phi = a \phi \]
and divides the basis of the eigenstates into two partitions $p$ and $q$. 
If the $p$ partition is just a single basis function and the $q$ partition 
    is all other basis functions,
    one finds with some manipulation that the eigenvalues are defined by the relation 
\[ a =  A_{11} + \mathbf{A}_{1q}( a \cdot \mathbf{1}_{qq} - \mathbf{A}_{qq})^{-1} \mathbf{A}_{q1} \]
The above is the ``reduced'' characteristic equation used for finding the eigenvalues.
It can be thought of as an alternative to the standard characteristic equation $\mathrm{det}(A - aI) = 0$.

If we replace the eigenvalues $a$ in the expression above with some continuous, variable parameter $\kappa$, we obtain the following: 
\[f(\kappa) =  A_{11} + \mathbf{A}_{1q}( \kappa \cdot \mathbf{1}_{qq} - \mathbf{A}_{qq})^{-1} \mathbf{A}_{q1} \]

Now, whenever this function is such that $f(\kappa) = \kappa $, we have found an eigenvalue $a = \kappa$.
If we vary this $\kappa$ until $f(\kappa) - \kappa = 0$, we pull out a discrete eigenvalue which is a solution to the eigenvalue problem. 

The key quantity above 
\[\mathcal{R}_{\kappa}(A) = ( \kappa \cdot \mathbf{1}_{qq} - \mathbf{A}_{qq})^{-1}\]
is the \textbf{reduced resolvent} operator of $\mathbf{A}$ for a particular $\kappa$.
\footnote{Here, $\mathbf{1}_{qq}$ is the identity matrix of size $q \times q$ and $\mathbf{A}_{qq}$ is 
 the matrix representation of the operator in just the $q$-space, (over just $q$ basis functions)}.
In words, the reduced resolvent is an operator that is assigned to another operator $\mathbf{A}$.
It is only defined in $q$-space part of the operator, and different values of $\kappa$ give different reduced resolvents.
As abstract as that is, it is tremendously useful, as we will see later.

Notice how the resolvent is dependent on just the matrix elements of the operator $\mathbf{A}$
which correspond to the $q$ partition; it lives in the $q$-space partition of our basis.
Expressed pictorially as a matrix representation divided into $pp$, $pq$, $qp$, $qq$ blocks,
\[\mathcal{R}_{\kappa}(A) =
\left( \begin{array}{c|c}
   \mathbf{0} & \mathbf{0} \\
   \midrule
   \mathbf{0} & (\kappa \mathbf{1}_{qq} - A_{qq})^{-1} \\
\end{array}\right)
  \]

\section{L{\"o}wdin's big idea: using reduced resolvents to define perturbation theory and solve Schr{\"o}dinger equation}
Back to the world of quantum chemistry and the Schrodinger equation, 
we can define two projection operators $P$ and $Q$ which partition our $n$-electron Hilbert space into two parts(note: if that sounds like gobbledygook to you, see the Appendix for details).
It is this partitioning which birthed the idea of a reduced resolvent above.

$P$ is the ``model space projection operator'', which can act on a state 
    and project it onto some reference determinant $\Phi_0$.
\[P = \ket{\Phi_0}\bra{\Phi_0}  \]
$Q$ is the \textit{orthogonal complement} of $P$, which projects out every determinant other than the reference $\Phi_0$
\[Q = 1 - P = \sum_{n \neq 0}  \ket{\Phi_n}\bra{\Phi_n}\]
$P$ and $Q$ basically just split our basis into two parts, and the sum of $P$ and $Q$ yield the resolution of the identity, $P + Q = 1$.

In terms of the $Q$ operator, the definition of the reduced resolvent for some operator $H$ is
\[\mathcal{R}_{\kappa} = (\kappa - H)^{-1} Q  \]

L{\"o}wdin's idea, in short, was to generalize the Schr{\"o}dinger equation
to a continous spectrum in $\kappa$, and use the $\kappa$-dependent reduced resolvent in the definition of 
a trial wavefunction to solve for the eigenvalues. Read that last sentence a few more times. 
Hopefully, it will soon become clear what I mean by this.

Starting from the re-arranged Schr{\"o}dinger equation,
\[(E_0 - H) \ket{\Psi_0} = 0 \]
we generalize it to a continuous spectrum in $\kappa$,
\[(\kappa - H) \ket{\Psi_{\kappa}} = ... \]
The above equation \textit{becomes} the Schrodinger equation whenever
$\kappa$ happens to be equal to a true eigenvalue $E_0$.

L{\"o}wdin derived that 
a \textbf{trial wavefunction} which is of the form $\ket{\Psi_0} = \Omega \ket{\Phi_0}$
can be expressed in terms of the reduced resolvent. 
\[\ket{\Psi_{\kappa}} = \ket{\Phi_0} + \mathcal{R}_{\kappa} H \ket{\Phi_0} \]
What happens when we apply $(\kappa - H)$ to our trial wavefunction $\ket{\Psi_{\kappa}}$? 
as in $(\kappa - H) \ket{\Psi_{\kappa}} = ... $?
Applying the resolution of the identity $P + Q = 1$, we find
\[ (\kappa - H) \ket{\Psi_{\kappa}} =  P (\kappa - H) \ket{\Psi_{\kappa}} + Q (\kappa - H) \ket{\Psi_{\kappa}}  \]
Looking at just the $Q$ term, we find that

\begin{align*}
Q (\kappa - H) \ket{\Psi_{\kappa}} &= Q (\kappa - H) \ket{\Phi_0} + Q (\kappa - H) \mathcal{R}_{\kappa} H \ket{\Phi_0}  \\
&= \kappa Q \ket{\Phi_0} - Q H \ket{\Phi_0} + Q H \ket{\Phi_0}
&= \kappa Q \ket{\Phi_0} 
&= 0
\end{align*}
Above we have used the fact that $ Q (\kappa - H) \mathcal{R}_{\kappa}  = Q$ , which follows
clearly from the definition of the reduced resolvent, and that $Q$ action on $\ket{\Phi_0}$ is 0, since 
$\ket{\Phi_0}$ lives in the $P$ space. 
We conclude the $Q$ term is 0, so we are left with:
\[ (\kappa - H) \ket{\Psi_{\kappa}} =  P (\kappa - H) \ket{\Psi_{\kappa}} \]
Using the definition of $P = \ket{\Phi_0}\bra{\Phi_0}$,
\begin{align*}
 (\kappa - H) \ket{\Psi_{\kappa}} &=  \ket{\Phi_0}\bra{\Phi_0} (\kappa - H) \ket{\Psi_{\kappa}} \\
 &=  \kappa \ket{\Phi_0} \braket{\Phi_0|\Psi_\kappa} - \ket{\Phi_0} \braket{\Phi_0| H | \Psi_\kappa} \\
 &= (\kappa - \braket{\Phi_0 | H | \Psi_\kappa}) \ket{\Phi_0} \\
 &= (\kappa - \braket{\Phi_0 | H + H \mathcal{R}_\kappa H  | \Phi_0} ) \ket{\Phi_0} \\
 &= (\kappa - f(\kappa)) \ket{\Phi_0}
\end{align*}

where we have defined the function of $\kappa$, denoted  $f(\kappa)$ as
\[ f(\kappa) =  \braket{\Phi_0 | H | \Psi_\kappa} = \braket{\Phi_0 | H + H \mathcal{R}_\kappa H  | \Phi_0}\]
L{\"o}wdin calls this the \textit{bracketing function}, since it enables one to obtain both lower and upper bounds to true eigenvalues, it ``brackets'' them.

Restating the above in one compact expression, we have 
\[ (\kappa - H) \ket{\Psi_{\kappa}} = (\kappa - f(\kappa)) \ket{\Phi_0} \]
This is a monumental result.
\textbf{When $\kappa$ is equal to $f(\kappa)$, the right side above goes to zero. If this occurs,
we conclude that our trial function $\ket{\Psi_{\kappa}}$
is a solution to the Schr{\"o}dinger equation, and the eigenvalue $\kappa$ is our 
energy eigenvalue $E_0$.}  

Thus, our chosen trial function (derived by L{\"o}wdin) gives us access to the solutions of 
the Schrodinger equation, by looking for zeros of $F(\kappa) = f(\kappa) - \kappa$.

When we find a zero, we have the following well-defined, obtainable quantities
\[ \kappa = E_0 \quad \quad (E_0 - H) \ket{\Psi_{\kappa=E_0}} = 0 \quad \quad \ket{\Psi_{\kappa=E_0}} = \ket{\Phi_0} + \mathcal{R}_{\kappa = E_0} H \ket{\Phi_0} \]

The above equation is always applicable so long as the wavefunction is not orthogonal to $\ket{\Phi_0}$
and the resolvent exists.

We conclude from the above that our wave operator $\Omega$ such that
\[\ket{\Psi_\kappa} = \Omega_\kappa \ket{\Phi_0} \]
\[\kappa = \braket{\Phi_0|H \Omega_{\kappa} | \Phi_0} \]
is equal to
\[\Omega_{\kappa} = 1 + \mathcal{R}_{\kappa} H \]
For brevity, it is more proper to express $\Omega$ in terms of $P$ to ensure it behaves properly when acted
on any state
\[\Omega_{\kappa} = P + \mathcal{R}_{E_0} H P \]
As defined in the case of $\ket{\Psi_0} = \Omega_{\kappa} \ket{\Phi_0} $, replacing $P$ with unity does not have an effect.
$\Omega$ has the following properties, which we will not prove:
\[ P \Omega = P \quad \quad \Omega P = \Omega \quad \quad \Omega^2 = \Omega \]
These conditions classify $\Omega$ as a Bloch wave operator.

Having defined $\Omega$, we can now re-express our main result above in terms of just our reference
state $\ket{\Phi_0}$,
\[ (\kappa - H) \Omega_\kappa \ket{\Phi_0} = (\kappa - f(\kappa)) \ket{\Phi_0} \]
\[ f(\kappa) = \braket{\Phi_0 | H \Omega_\kappa | \Phi_0}\]

\subsection{L{\"o}wdin's Rayleigh-Schr{\"o}dinger Perturbation Theory}
We can now take our generalized Schr{\"o}dinger equation from above and consider 
how it behaves when our Hamiltonian is partitioned into two parts $H = H_0 + V$. 
Our wave operator becomes
\[\Omega = P + \mathcal{R}_{\kappa} H_0 P + \mathcal{R}_{\kappa} V P \]
\[\Omega = P + \mathcal{R}_{\kappa} H_0 \ket{\Phi_0}\bra{\Phi_0} + \mathcal{R}_{\kappa} V P \]
\[\Omega = P + E_0 \mathcal{R}_{\kappa} P + \mathcal{R}_{\kappa} V P \]
\[\Omega = P + \mathcal{R}_{\kappa} V P \]
where the reduced resolvent acting on $P$ is zero, since the resolvent entirely resides in the $Q$ space, which is orthognoal to $P$.

After a few manipulations which are left as an exercise, 
the bracketing function $f(\kappa) = \braket{\Phi_0 | H \Omega_\kappa | \Phi_0}$ becomes 



\section{Appendix}
L{\"o}wdin was a smart dude. In order to understand what the heck he's talking about,
we need to put on our math hats and review a few things.

\subsection{Hilbert Space}
Recall that many of the objects we have been talking about (spin orbitals, Slater determinants, wavefunctions)
``live'' in some Hilbert space. This means many things, but for our purposes, we just need to know
there is some ``space'' of \textit{abstract vectors} which have a defined inner product with a certain set of properties.
Every abstract vector in the space can be described in terms of the orthonormal basis which makes up the space. 

\begin{itemize}
\item The set of all spin orbitals form a basis for what you might call the one-electron Hilbert space $\mathcal{H}$
\item $n$ tensor products of $\mathcal{H}$ with itself forms an $n$-electron Hilbert space $\mathcal{H}^n$. 
        A member of the basis of such a space is a product of spin orbitals (Hartree products)
\item Denote the \textit{antisymmetric subspace} of $\mathcal{H}^n$ as $\mathscr{H}^n$. The basis of this space
    is the set of $n$-electron Slater determinants. All possible Slater determinants for an $n$-electron system live in $\mathscr{H}^n$ 
    The full CI wavefunction for an $n$-electron system can be constructed from a linear combination of basis vectors in $\mathscr{H}^n$
\end{itemize}

\subsection{Partitioning}
Consider the following \textit{partition} of this $n$-electron antisymmetric Hilbert space $\mathscr{H}^n$ into
a ``$P$'' space and a ``$Q$'' space.
\[ \mathscr{H}^n = \mathscr{H}^n_P \oplus  \mathscr{H}^n_Q \]
where $P$ is the \textit{projection operator} onto some reference determinant 
\[P = \ket{\Phi_0}\bra{\Phi_0}  \]
and $Q$ is the \textit{orthogonal complement} of $P$, which projects onto every determinant other than the reference $\Phi_0$
\[Q = 1 - P = \sum_{n \neq 0}  \ket{\Phi_n}\bra{\Phi_n} \]
You can think of this as just dividing your Slater determinant basis into two parts, $P$ and $Q$.
The sum $P + Q$ is equal to unity. This can be most easily seen by thinking of $P$ and $Q$ as two distinct chunks of the resolution of the identity. 





\end{document}

\documentclass{article}
\usepackage{amsmath,mathtools,amssymb}
\usepackage{graphicx}
\usepackage{gensymb}
\usepackage{verbatim}
\usepackage{mathrsfs}
\usepackage{bbm}
\usepackage{braket}
\usepackage{verbatim}
\usepackage[margin=1.0in]{geometry}
\newcommand{\ol}{\overline}
\newcommand{\ve}{\varepsilon}
\def\*#1{\mathbf{#1}}
\DeclarePairedDelimiter\floor{\lfloor}{\rfloor}

\title{Lecture 3.2:  Second Quantization: Hamiltonian}
\date{February , 2020}
\begin{document}
\maketitle
\noindent
\section{Proving the form of the many-body operator in second-quantization}
\subsection{How do creation and annihilation operators act on a determinant }
%\section{advantages of Second Quantization}
%As we will later see, this choice will lead to easier mathematical manipulations. 
\section{Wick's Theorem}
%maybe outline of how it's gonna go - the things in wick's thrm treat one by one, etc.
\subsection{Normal Products}
\subsection{Contractions of Operators}
\subsection{Normal Products with Contractions}
\subsection{Proof of 2 Lemmas}
\subsection{Proof of Wick's Theorem}
%after this - maybe a map of why wick's theorem is useful: have normal products and normal products with contractions, have 
%rules for determining these in the vacuum, then wick's theorem provides a way to put any product of operators in these forms
\section{Derivation of Slater's rules using Wick's Theorem}

\section{Particle-Hole Formalism}
\subsection{Motivating the particle-hole formalism}
\subsection{Normal Products}
\subsection{Contractions of Operators}
\subsection{Normal Products with Contractions}
\subsection{Wick's Theorem}
\section{$\Phi$--Normal ordered Hamiltonian}

\begin{comment}
Types of problems:
1. Slater's rules
2. 


Types of ways to solve:
1. First quantization
2. 2nd quantization in the true vacuum without Wick's theorem
3. 2nd quantization in the true vacuum with Wick's theorem
4. PH formalism with Wick's theorem 
5. PH formalism with Wick's theorem and $\Phi$-normal ordered Hamiltonian

\end{comment}

\section{Exercises}

\section{Recommended Resources}
\end{document}

\documentclass{article}
\usepackage{amsmath,mathtools,amssymb}
\usepackage{simplewick}
\usepackage{graphicx}
\usepackage{booktabs}
\usepackage{blkarray}
\usepackage{gensymb}
\usepackage{verbatim}
\usepackage{mathrsfs}
\usepackage{bbm}
\usepackage{braket}
\usepackage{hyperref}
\usepackage{verbatim}
\usepackage{cancel}
\usepackage{xcolor}
\newcommand{\fctr}{\contraction}
\usepackage[margin=1.0in]{geometry}
\newcommand{\ol}{\overline}
\newcommand{\lp}{\left(}
\newcommand{\rp}{\right)}
\newcommand{\eps}{\varepsilon}
\newcommand{\lam}{\lambda}
\newcommand{\h}{\circ}
\newcommand{\p}{\bullet}
\newcommand{\mc}{\mathcal}
\newcommand{\mr}{\mathrm}
\newcommand{\cd}{\cdots}
\newcommand{\Ezero}{E^{(0)}}
\newcommand{\Rz}{\mathcal{R}_{0}}

\newcommand{\Phizero}{\Phi^{(0)}}
\newcommand{\Eone}{E^{(1)}}
\newcommand{\En}{E^{(n)}}
\newcommand{\Phione}{\Phi^{(1)}}
\newcommand{\Phin}{\Phi^{(n)}}

\newcommand{\Ecorr}{E_{\mathrm{corr}}}
\newcommand{\Ec}{E_{\mathrm{c}}}
\newcommand{\Vc}{V_{\mathrm{c}}}
\newcommand{\Hc}{H_{\mathrm{c}}}
\newcommand{\dg}{\ensuremath{^\dagger} }
\def\*#1{\mathbf{#1}}
\DeclarePairedDelimiter\floor{\lfloor}{\rfloor}

\title{Lecture 6: Coupled Cluster Theory}
\date{April 13, 2020}
\begin{document}
\maketitle
\noindent

\section{Quantum Chemistry Wavefunctions}
In the last set of notes, we introduced the concept of a wave operator $\Omega$, which transforms 
    our reference determinant $\Phi$ into the exact wavefunction $\Psi$
\[\ket{\Psi} = \Omega \ket{\Phi} \]
Different valid choices of the wave operator  $\Omega$ constitute different \textit{ansatz},  
    which each arrive at valid form for the same exact wavefunction $\ket{\Psi}$.
In perturbation theory (PT), our wave operator is of the form,
$\Omega = P + \sum_{n=0}^{\infty} (\Rz V')^n \Rz V P$, so that our wavefunction is
\[\ket{\Psi_{\mathrm{PT}}} =  \ket{\Phi_0} + \sum_{n=0}^{\infty} (\Rz V')^n \Rz V \ket{\Phi} \]
In configuration interaction (CI), our wave operator (though we did not call it that at the time)
takes the form $\Omega = (1 + C_1 + C_2 + \cdots + C_n)$ so that 
\[\ket{\Psi_{\mathrm{CI}}} = (1 + C_1 + C_2 + \cdots + C_n) \ket{\Phi} \]
In coupled cluster theory (CC), our wave operator, and therefore our wavefunction, 
is an exponential form,
\[\ket{\Psi_{\mathrm{CC}}} = \exp(T_1 + T_2 + \cdots + T_n) \ket{\Phi} \]
Or simply, 
\[\ket{\Psi_{\mathrm{CC}}} = \exp(T) \ket{\Phi} \]

Each of the wavefunctions, in their limit, approach the exact same wavefunction.
However, perturbation theory is an odd duck, since it always has an infinite number of
    contributions to the wavefunction and energy, 
    whereas the CC and CI wavefunctions for a given $n$-electron system 
    terminate at $n$ terms (you can't have a quadruple excitation in a 3-electron system). 

\section{Coupled Cluster Theory}
Our coupled cluster wavefunction from before is 
\[\ket{\Psi_{\mathrm{CC}}} = \exp(T) \ket{\Phi} \]
\[\ket{\Psi_{\mathrm{CC}}} = \exp(T_1 + T_2 + \cdots + T_n) \ket{\Phi} \]
where $T$ is the ``cluster operator'' which is a sum over
$k$-electron excitation operators
\[T = T_1 + T_2 + T_3 + \cdots \]
\[T_k = \left(\frac{1}{k!}\right)^2 t_{a_1 \cdots a_k}^{i_1 \cdots i_k } \tilde{a}_{i_1 \cdots i_k}^{a_1 \cdots a_k } \]
where $t_{a_1 \cdots a_k}^{i_1 \cdots i_k }$ are the coupled cluster \textit{amplitudes}.
It is worth briefly pointing out some buried complexity here. Noting that the function $e^T$ can be expanded according to it's Taylor series, the full form of the coupled cluster wavefunction
looks like the following, 
\[ \ket{\Psi_{\mathrm{CC}}} = \left[ 1 + (T_1 + T_2 + T_3 + T_4 + \cdots ) + \frac{1}{2} (T_1 + T_2 + T_3 + T_4 + \cdots)^2 +  \frac{1}{6} (T_1 + T_2 + T_3 + T_4 + \cdots)^3 \cdots \right] \ket{\Phi}  \]

Solving for the coupled cluster energy is done by doing the same thing we always do:
project our reference determinant $\bra{\Phi}$ on the left of both sides 
of the Schr{\"o}dinger equation.
Assuming intermediate normalization, we obtain
\[\braket{\Phi | \Hc | \Psi_{\mathrm{CC}}} = E_c \braket{\Phi|\Psi_{\mathrm{CC}}} \]
\[\braket{\Phi | \Hc \exp(T) | \Phi} = \Ec \]

Equations for the cluster amplitudes $t_{a b \cdots }^{i j \cdots}$
can be found by left-projecting the Schr{\"o}dinger equation by excited determinants
\[ \braket{ \Phi_{i j \cdots }^{a b \cdots} | \Hc \exp(T) | \Phi} = 
   \Ec \braket{\Phi_{i j \cdots }^{a b \cdots} | \exp(T) | \Phi} \]
where an equation involving $t_i^a$ would be obtained by projecting by $\bra{\Phi_{i}^{a}}$,
an equation for $t_{ij}^{ab}$ by projecting by $\bra{\Phi_{ij}^{ab}}$, and so on. 
To see this, one can take the right side of the above equation and expand the exponential $\exp(T) = 1 + T + \frac{T^2}{2!} + \cdots $, and only one term survives:
\[ \Ec \braket{\Phi_{i j \cdots }^{a b \cdots} | \exp(T) | \Phi} = \Ec t_{a b \cdots }^{i j \cdots}\]
Plugging in the exponential expansion of $\exp(T)$ into the energy equation
yields 
\[ \braket{\Phi | \Hc | \Phi} + \braket{\Phi| \Hc T| \Phi} + \braket{\Phi | \Hc \frac{T^2}{2!} | \Phi} = \Ec \]
These terms are the only ones that survive because higher powers of 
$T$ contain no terms for which one can obtain complete contractions. 
An analgous equation exists for the amplitudes.

These equations is not easily solvable. The trick used in traditional coupled cluster 
theory to make these equations more workable is to multiply by $\exp(-T)$ on both sides,
which effectively transforms our Hamiltonian $\exp(-T) H \exp(T)$.
We highlight the key results in blue. 
\[\braket{\Phi | \Hc | \Psi_{\mathrm{CC}}} = E_c \braket{\Phi|\Psi_{\mathrm{CC}}} \]
\[ \braket{\Phi | \exp(-T) \Hc \exp(T) | \Phi } = E \braket{\Phi | \exp(-T) \exp(T) | \Phi} \]
\[ \color{blue} \braket{\Phi | \exp(-T) \Hc \exp(T) | \Phi } = \Ec \]
\quad
\\
\[ \braket{ \Phi_{i j \cdots }^{a b \cdots} | \Hc \exp(T) | \Phi} = 
    \Ec \braket{\Phi_{i j \cdots }^{a b \cdots} | \exp(T) | \Phi}  \]
\[ \braket{ \Phi_{i j \cdots }^{a b \cdots} | \exp(-T) \Hc \exp(T) | \Phi} = 
    \Ec \braket{\Phi_{i j \cdots }^{a b \cdots} | \exp(-T) \exp(T) | \Phi}  \]
\[ \braket{ \Phi_{i j \cdots }^{a b \cdots} | \exp(-T) \Hc \exp(T) | \Phi} = 
    \Ec \braket{\Phi_{i j \cdots }^{a b \cdots} | \Phi}  \]
\[ \color{blue} \braket{ \Phi_{i j \cdots }^{a b \cdots} | \exp(-T) \Hc \exp(T) | \Phi} 
   =  0  \]

This transformation decouples the amplitude equation from the energy equation
by setting the right side of the amplitude equation to zero. 
In the language of diagrams, this transformation removes ``disconnected'' diagrams,
and $\Ec t_{ab \cdots}^{ij \cdots}$ is one example. 
This transformation also can be simplified further via the \textit{Hausdorff expansion}.

\subsection{The Hausdorff Expansion}
The Hausdorff expansion expresses our transformed Hamiltonian in
terms of a sum over nested commutators. As we will see, upon simplification it produces 
an expansion in powers of $T$ which naturally truncates, thus removing the infinite 
expansion due to the exponential $\exp(T)$. 
The Hausdorff expansion is,
\[ \exp(-T) \Hc \exp(T) = \Hc + [\Hc,T] + \frac{1}{2!} [[\Hc,T],T] + 
  \frac{1}{3!} [[[\Hc,T],T],T] +  \frac{1}{4!} [[[[\Hc,T],T],T],T] + \cdots   \]
This is simplified by evaluating the commutators using Wick's theorem.
Generally, if two strings of operators $Q$ and $Q'$ are normal ordered, 
and one of them has an even operator count, then 
\[ [Q, Q'] =  N[\fctr{}{Q}{}{Q} Q Q'] - N[\fctr{}{Q}{'}{Q} Q' Q ] \]
In this specific case, 
\[ [\Hc, T] = \Hc T - T\Hc = N[\Hc T] + N[\fctr{}{H}{{}_\mathrm{c}}{T}H_\mathrm{c}T]
  - N[T \Hc] - N[\fctr{}{T}{}{H} T H_\mathrm{c}] \]
\[ [\Hc, T] =  N[\fctr{}{H}{{}_\mathrm{c}}{T}H_\mathrm{c}T] - N[\fctr{}{T}{}{H} T H_\mathrm{c}] \]
The second term above is actually always 0, for any contraction.  
The reason is that $T$ is composed entirely of quasi-particle creation operators, 
which will be on the left for any possible contraction in $N[\fctr{}{T}{}{H} T H_\mathrm{c}]$.
The only nonzero contraction involving quasi-particle operators occurs
when the creation operator is on the \textit{right}, $ \fctr{}{b}{{}_q}{b} b_p b_q^\dagger = \delta_{pq} $.
Therefore, any possible contraction $N[\fctr{}{T}{}{H} T H_\mathrm{c}]$ must always be zero. 
We are left with the following, 
\[ [\Hc, T] =  N[\fctr{}{H}{{}_\mathrm{c}}{T}H_\mathrm{c}T] \]
For the nested commutators in the Hausdorff expansion, 
since $T$ has no contractions with operators to its right, 
we find that the only terms which survive are ones in which \textit{every single
cluster operator $T$ is contracted with the Hamiltonian at least once}.
In the language of diagrams, we call these terms \textit{connected},
since all excitation operators are connected to the Hamiltonian through a contraction.
We label such terms with a subscript capital C.
\[
  [\,\cdot\,,T]^n
=
N[
  \fctr[1.5ex]{}{H}{{}_\mathrm{c}TT\cdots}{T}
  \fctr[1.0ex]{}{H}{{}_\mathrm{c}T}{T}
  \fctr[0.5ex]{}{H}{{}_\mathrm{c}}{T}
  H_\mathrm{c} TT\cdots T
]
=
  (H_\mathrm{c} T^n)_\mathrm{C}
\]
The restriction to only connected terms naturally eliminates many terms in the Hausdorff expansion. 
Furthermore, since $\Hc$ is a linear combination of one-electron and two-electron operators, 
    the components of $\Hc$ can contract with at most two and four $T$'s, respectively. 
Since every $T$ has to be connected to $\Hc$, we are only left with, 
connected terms up to the fourth power in $T$:
\[ (\Hc \exp(T))_{\mathrm{C}} = (\Hc + \Hc T + \frac{1}{2!} \Hc T^2 +
  \frac{1}{3!} \Hc T^3 +  \frac{1}{4!} \Hc T^4)_{\mathrm{C}} \]
To convince yourself of this, you can write out the full form 
of $T = T_1 + T_2 + T_3 + \cdots $, re-express each $T_k$ according to their
definition, take various powers of the expansion $T^2, T^3, T^4, T^5$, 
and try contracting a few terms with the Hamiltonian. 
You will find that most terms in $T^4$ cannot be such that every $T_k$ forms a contraction
with $\Hc$, and no terms in $T^5$ can have each cluster operator contract with $\Hc$.
There isn't enough ``room'' in $\Hc$ for that many contractions.

To summarize, we initially replaced $\Hc \exp(T)$ in our energy and amplitude equations
with $ \exp(-T) \Hc \exp(T)$. We then used the Hausdorff expansion.
Upon simplifying, we found that $\exp(-T) \Hc \exp(T) = (\Hc \exp(T))_{\mathrm{C}}$.
This tells us that only ``connected'' terms survive, 
    and all disconnected terms necessarily cancel out. 
Furthermore, our Hamiltonian only allows contractions with up to 4 $T$'s,
so the Hausdorff expansion naturally truncates after $T^4$.
This gives us our final results for the coupled cluster energy and amplitude equations, 

\[\braket{\Phi |\Hc \exp(T) | \Phi }_\mathrm{C} = \Ec \]
\[\braket{ \Phi_{i j \cdots }^{a b \cdots} |\Hc \exp(T) | \Phi}_\mathrm{C} 
   =  0  \]
Note that these are the same results as the equations previously derived (in blue),
except we have used the result of the Hausdorff expansion to replace 
 $\exp(-T) \Hc \exp(T)$ with just connected terms $(\Hc \exp(T))_{\mathrm{C}}$.
Practically, when one uses the equations above algebraically, 
one should look for complete contractions in which the Hamiltonian
has at least one contraction with every cluster operator on its right.

\subsection{Excitation Levels}
There is an alternative to considering which terms form complete contractions when
evaluating expectation values. 
The \textit{excitation level} of a series of operators is equal 
to the net number of particles or quasi-particles it creates, divided by 2.
The excitation levels of $T_1$, $T_2$, and $T_3$ are 1, 2, and 3.
$\tilde{a}_{abcd}^{ijkl}$, which would perhaps appear in a bra as
$\bra{\Phi_{ijkl}^{abcd}}$, would have an excitation level of -4.
\textbf{The total excitation level of a term in an expectation value must balance out
to zero}, otherwise the term is zero. 
Thus, this is another way of detecting which terms cannot form complete contractions.
What is the excitation level of terms in our Hamiltonian $\Hc$?
Excitation operators in the Hamiltonian are made up of general indices $p,q,r,s$ (not necessarily occupied or virtual), so the excitation level of the Hamiltonian is variable.
The one-electron part of the Hamiltonian can have an excitation level of 0, +1, or -1. 
The two-electron part of the Hamiltonian can have an excitation level of 0, +1 , -1, +2, or -2.  
To see why, recall that the one-electron operator is $h_p^q \tilde{a}_q^p$, which
can take on virtual and occupied indices in four ways: 
$h_a^b \tilde{a}_b^a, h_a^i \tilde{a}_i^a, h_i^a \tilde{a}_a^i, h_i^j \tilde{a}_j^i$.
In the $ h_a^i \tilde{a}_i^a$ case, it converts to two quasi-particle 
creation operators $b_a^\dagger b_i^\dagger$. According to our definition of excitation
level, this has an exciation level of +1. In the case of $ h_i^a \tilde{a}_a^i $,
we have $b_i b_a $, two quasi-particle annihilation operators, so that the excitation
level is -1. 
The other two cases have excitation levels of 0.
Similar arguments can be made for the two-electron cases involving $\tilde{a}_{rs}^{pq}$.

By analyzing excitation levels, we can decide if expectation values are 0 or not.
Examples,
\begin{itemize}
\item The following expectation value is nonzero. Each $T_1$ has an excitation
level of +1, $\tilde{a}_{ab}^{ij}$ has an excitation level of -2, and the Hamiltonian
excitation level can be chosen to be 0 so that 1 + 1 - 2 + 0 = 0.
\[ \braket{\Phi_{ij}^{ab} | \Hc T_1^2 | \Phi} \] 
\item The following expectation value is 0. $T_4$ has an excitation level of +4,
$\tilde{a}_{a}^{i}$ has an excitation level of -1, and we can choose 
the Hamiltonian to have an excitation level of -2, at best. -1 + -2 + 4 = 1.
\[ \braket{\Phi_i^a | \Hc T_4 | \Phi} \] 
\end{itemize}
One could come up with similar conclusions by looking at what complete contractions
can be formed, but it is often much faster to count excitation levels.

When evaluating coupled cluster expectation values, if the excitation level 
of a term is not 0, the term can be eliminated.
However, just because the excitation level is 0 does not mean the term is automatically considered.
From the Hausdorff expansion, we also must have that the Hamiltonian is connected to each
cluster operator in a given term. If the excitation levels balance to zero and the Hamiltonian
can be connected (contracted) to all cluster operators in the term, the term is nonzero.
The following example has an excitation level of 0, but there is no way to connect $\Hc$ to the cluster
operator $T_2$, so it goes to zero.
\[\braket{\Phi_{ijkl}^{abcd}| \Hc T_2 | \Phi } \]

\subsection{The Amplitude Equations}
The cleanest way to solve for the cluster amplitudes 
in our amplitude equation 
\[\braket{ \Phi_{i j \cdots }^{a b \cdots} |\Hc \exp(T) | \Phi}_\mathrm{C} = 0\]
requires an odd-looking partitioning of our Hamiltonian 
into two parts: the diagonal part of the Fock operator, and everything else. 
\[H_0 = f_p^p \tilde{a}_p^p \]
\[\Vc = \Hc - H_0 = f_p^q (1 - \delta_p^q) \tilde{a}_q^p + \frac{1}{4} \bar{g}_{pq}^{rs}
 \tilde{a}_{rs}^{pq} \]
$H_0$ can be absorbed into an excited determinant to give back
a term which depends on the orbital energies,
\begin{align}
  H_0\Phi_{i_1\cd i_k}^{a_1\cd a_k}
=
  \mathcal{E}_{i_1\cd i_k}^{a_1\cd a_k}
  \Phi_{i_1\cd i_k}^{a_1\cd a_k}
&&
  \mathcal{E}_{q_1\cd q_k}^{p_1\cd p_k}
\equiv
  \sum_{r=1}^k
  f_{p_r}^{p_r}
-
  \sum_{r=1}^k
  f_{q_r}^{q_r}
\end{align}
$\mathcal{E}_{i_1\cd i_k}^{a_1\cd a_k}$ is essentially the change 
in the sum of the orbital energies from the reference state (Hartree-Fock) for the given 
excited determinant. A single  
$\mathcal{E}_{ij}^{ab}$ is $\mathcal{E}_{ij}^{ab} = \epsilon_a + \epsilon_b - \epsilon_i - \epsilon_j$. 
Using the above equations in conjunction with our amplitude equations,
\[\braket{ \Phi_{i j \cdots }^{a b \cdots} |\Hc \exp(T) | \Phi}_\mathrm{C} 
   =  0  \]
We plug in our partitioned Hamiltonian $\Hc = H_0 + \Vc$,
\[\braket{ \Phi_{i j \cdots }^{a b \cdots} |H_0 \exp(T) | \Phi}_\mathrm{C} 
 + \braket{ \Phi_{i j \cdots }^{a b \cdots} |\Vc \exp(T) | \Phi}_\mathrm{C}
   =  0  \]
The left term evalutes to 
\[ \braket{ \Phi_{i j \cdots }^{a b \cdots} |H_0 \exp(T) | \Phi}_\mathrm{C} = 
   \braket{ \Phi_{i j \cdots }^{a b \cdots} |H_0 T | \Phi} = 
\mathcal{E}_{ij \cd}^{ab \cd} \braket{\Phi_{i j \cdots }^{a b \cdots} |T|\Phi} =
\mathcal{E}_{ij \cd}^{ab \cd} t_{ab \cd}^{ij \cd} \]
where we have used the fact that only $T$ in $\exp(T) = 1 + T + \frac{1}{2}T^2 \cdots $
can create connected terms with $H_0$, since it is a one-electron operator.
Now our amplitude equation reads as
\[ \braket{ \Phi_{i j \cdots }^{a b \cdots} |\Vc \exp(T) | \Phi}_\mathrm{C}
  + \mathcal{E}_{ij \cd}^{ab \cd} t_{ab \cd}^{ij \cd}   =  0  \]
Since $\mathcal{E}_{ij \cd}^{ab \cd}$ is just a linear combination of orbital
energies, we can absorb a minus sign by swapping upper and lower indices, $-\mathcal{E}_{ij \cd}^{ab \cd} = \mathcal{E}_{ab \cd}^{ij \cd}$,  which gives us the following 
clean equation for solving for the amplitudes,  
\[ t_{ab \cd}^{ij \cd} = \left(\mathcal{E}_{ab \cd}^{ij \cd}\right)^{-1} 
        \braket{ \Phi_{i j \cdots }^{a b \cdots} |\Vc \exp(T) | \Phi}_\mathrm{C}  \]
   
The amplitude equations must be solved iteratively, since $t_{ab \cd}^{ij \cd}$ is
also embedded in $\exp(T)$.

\subsection{The CCSDTQ amplitude equations} 
If we truncate the cluster operator $T = T_1 + T_2 + T_3 + T_4$,
then $\exp(T) = \exp(T_1 + T_2 + T_3 + T_4)$. 
Not every operator in the exponential expansion of $\exp(T_1 + T_2 + T_3 + T_4)$ 
is relevant;
disconnected terms are omitted, as well as terms for which the excitation level
of the expectation value does not balance out to 0.
After careful consideration, one can use the amplitude equation we derived above
to obtain amplitude equations for CCSDTQ:
\begin{align*}
  t_a^i
=& (\mc{E}_a^i)^{-1}
  \braket{\Phi_i^a| V_\mr{c} (1 + T_1 + T_2 + T_3 + \tfrac{1}{2} T_1^2 +T_1T_2 + \tfrac{1}{3!} T_1^3)|\Phi}_\mr{C}
\\[3pt]
  t_{ab}^{ij} =& (\mc{E}_{ab}^{ij})^{-1}
  \braket{\Phi_{ij}^{ab} | V_\mr{c} ( 1 + T_1 + T_2 +  T_3 + T_4 + \tfrac{1}{2}T_1^2 +  T_1T_2 +  T_1T_3 +  \tfrac{1}{2} T_2^2
    + \tfrac{1}{3!} T_1^3 +  \tfrac{1}{2} T_1^2T_2 + \tfrac{1}{4!} T_1^4 ) | \Phi}_\mr{C} 
\\[3pt] 
  t_{abc}^{ijk}
=& (\mc{E}_{abc}^{ijk})^{-1}
  \bra{\Phi_{ijk}^{abc}} V_\mr{c}
    (T_2 + T_3 + T_4 + T_1T_2 + T_1T_3 + \tfrac{1}{2} T_2^2 +T_1T_4 + T_2T_3 + \tfrac{1}{2}T_1^2T_2 + \tfrac{1}{2}T_1^2T_3 + \tfrac{1}{2}T_1T_2^2
    + \tfrac{1}{3!}T_1^3T_2) \ket{\Phi}_\mr{C}
\\[3pt]
  t_{abcd}^{ijkl}
=&
  (\mc{E}_{abcd}^{ijkl})^{-1} \bra{\Phi_{ijkl}^{abcd}} V_\mr{c}
    (T_3 + T_4 + T_1T_3 + \tfrac{1}{2} T_2^2 + T_1T_4 + T_2T_3 + T_2T_4 + \tfrac{1}{2} T_3^2 +
      \tfrac{1}{2} T_1^2T_3  + \tfrac{1}{2} T_1T_2^2 + \tfrac{1}{2} T_1^2T_4
\\
&
\hphantom{
  (\mc{E}_{abcd}^{ijkl})^{-1}
  \bra{\Phi_{ijkl}^{abcd}}
    V_\mr{c}
    (
      T_3\
 }
    + T_1T_2T_3 + \tfrac{1}{3!} T_2^3 +\tfrac{1}{3!}T_1^3T_3 + \tfrac{1}{2!2!}T_1^2T_2^2)
  \ket{\Phi}_\mr{C}
\end{align*}


\section{On the connection between Coupled Cluster and Configuration Interaction}
Our CI wavefunction is  
\[ \ket{\Psi_{\mathrm{CI}}} = (1  + C_1 + C_2 + \cdots) \ket{\Phi}  \]
Our CC wavefunction is 
\[ \ket{\Psi_{\mathrm{CC}}} = \exp(T_1 + T_2 + \cdots) \ket{\Phi}  \]
\[ \ket{\Psi_{\mathrm{CC}}} = \left[ 1 + (T_1 + T_2 + T_3 + T_4 + \cdots ) + \frac{1}{2} (T_1 + T_2 + T_3 + T_4 + \cdots)^2 +  \frac{1}{6} (T_1 + T_2 + T_3 + T_4 + \cdots)^3 \cdots \right] \ket{\Phi}  \]

Noting that each $T_k$ ($C_k$) is just made up of scalar cluster amplitudes 
$t_{ab \cdots}^{ij \cdots}$ (CI coefficients $c_{ab \cdots}^{ij \cdots}$) 
and a Phi-normal-ordered excitation operator,  we can compare these wavefunction 
expansions term by term and find the following:

\begin{align*}
C_1 &= T_1 \\
C_2 &= T_2 + \frac{1}{2} T_1^2 \\
C_3 &= T_3 + T_1 T_2 + \frac{1}{6} T_1^3 \\
C_4 &= T_4 + \frac{1}{2}T_2^2 + T_1 T_3 +  \frac{1}{2} T_1^2 T_2 + \frac{1}{24} T_1^4 \\
\end{align*}

CI does not discriminate between different kinds of $n$-tuple excitations,
but in coupled cluster thoery, we can sub-classify various $n$-tuple excitations 
in different ways. 
Consider the quadruple excitation terms.
The physical interpretation of $T_4$ as an excitation event on some 
quantum mechanical state is 4 electrons coming together at the same location
and, upon interacting, they all have to simultaneously excite to a new state.
$\frac{1}{2}T_2^2$, on the other hand, corresponds to an excitation event 
in which a pair of 2 electrons which are close enough to locally interact
each excite to a new state, while simultaneously another distinct pair of 
    electrons does the same.
Which of these physical events ($T_4$  or $\frac{1}{2}T_2^2$) are more likely 
    to occur in a molecule? 
Almost certainly the latter, since electrons tend to avoid each other, but they also
    tend to form electron pairs! 
We conclude that $\frac{1}{2}T_2^2$ is a more important contribution to our wavefunction
than $T_4$.

Now consider the CCSD wavefunction,
\[ \ket{\Psi_{\mathrm{CCSD}}} = \exp(T_1 + T_2) \ket{\Phi} \] 
\[ \ket{\Psi_{\mathrm{CCSD}}} = (1 + T_1 + T_2 + \frac{1}{2} T_1^2 + T_1 T_2 + 
 \frac{1}{6} T_1^2 + \frac{1}{2} T_2^2 + \frac{1}{2} T_1^2 T_2 + \frac{1}{24} T_1^4 + ...)    \ket{\Phi} \] 

We obtain all single and double excitations according to CI, but we also
obtain some triple excitations ($T_1 T_2$, $\frac{1}{6} T_1^3$) and some quadruple 
excitations ($\frac{1}{2} T_2^2 + \frac{1}{2} T_1^2 T_2 + \frac{1}{24} T_1^4 $).
This is why its called coupled cluster; higher order excitations 
(quadruple, $\frac{1}{2}T_2^2$) are included even in low-order truncation
of the scheme (CCSD) via \textit{coupling} two double excitations together.
To put it another way, in CCSD, we obtain some information about quadruple excitations 
by considering two clusters which each doubly excite simultaneously. 
Not only that, but CCSD is giving us a quadruple excitation of relatively high importance,
according to our physical argument earlier. 

It is clear that CCSD is more accurate than CISD.
However, coupled cluster gets even better at higher orders. 
Looking at our expressions for $C_1, C_2, C_3, C_4$ above,
we see that the majority of $C_4$ is made up of terms which 
occur at CCSDT; expanding the CCSDT exponential 
$\exp(T_1 + T_2 + T_3)$ will give you all of the terms in $C_4$ except 
$T_4$ (!!!).
Thus, CCSDT is way more accurate than CISDT, and it's even 
almost as accurate correlation-wise as CISDTQ, since all that is missing is $T_4$.
Nevermind the fact  that coupled cluster is size-consistent, which is 
probably making up for more accuracy than the relative importance of $T_4$ in some cases.
In this view, it is even proper to say that CCSDT is \textit{more} accurate than CISDTQ.


\subsection{Deriving CC from CI}
Starting from full-CI,
\[ \ket{\Psi} = \lp 1 + C_1 + C_2 + C_3 + \cdots + C_n \rp \ket{\Phi} \]
\[ \ket{\Psi} = \lp 1 + C \rp \ket{\Phi} \]
We can define the cluster operator $T = T_1 + T_2 + \cdots$ as $T = \ln(1+C)$ where $C = C_1 + C_2 + \cdots$.
This way, $1 + C = e^T$, so it must be the case 
\[\ket{\Psi} = e^T \ket{\Phi} \]
which is our coupled cluster wavefunction.

We can compare coupled cluster and CI operators using the Taylor expansion for $\ln (1 + z)$
\[ \ln (1 + z) = \sum_{m=1}^{\infty} \frac{ (-1)^{m+1}  z^m}{ m} \]
This implies that for $T$ we have 
\[ T = \ln (1 + C) = \sum_{m=1}^{\infty} \frac{(-1)^{m+1}  C^m}{m} \]
This series always converges, since we always have a finite number of electrons in a system.
Expanding the series we get,
\[T = C - \frac{C^2}{2} + \frac{C^3}{3} + \cdots \]
\[T = C_1 + C_2 + C_3 + \cdots -  \frac{1}{2} ( C_1 + C_2 + C_3 + \cdots)^2 +  
  \frac{1}{3} (C_1 + C_2 + C_3 + \cdots)^3 +  \cdots
\]



\section{On the connection between Coupled Cluster and Perturbation Theory}
TODO; diagrams pre-requisite.




\end{document}

%%%%%%%%%%%%%%%%%%%%%%%%%%%%%%%%%%%%%%%%%%%%%%%%%%%%%%%%%%%%%%%%%%%%%%%%%%%%%%%%%%
% This work is licensed under the Creative Commons Attribution 4.0 International %
% License. To view a copy of this license, visit                                 %
% http://creativecommons.org/licenses/by/4.0/.                                   %
%%%%%%%%%%%%%%%%%%%%%%%%%%%%%%%%%%%%%%%%%%%%%%%%%%%%%%%%%%%%%%%%%%%%%%%%%%%%%%%%%%
\documentclass[11pt]{article}
\usepackage[cm]{fullpage}
%%AVC PACKAGES
\usepackage{avcgreek}
\usepackage{avcfonts}
\usepackage{avcmath}
\usepackage[numberby=section,skip=9pt plus 2pt minus 7pt]{avcthm}
\usepackage{qcmacros}
\usepackage{goldstone}
%%MACROS FOR THIS DOCUMENT
\numberwithin{equation}{section}
\usepackage[
  margin=1.5cm,
  includefoot,
  footskip=30pt,
  headsep=0.2cm,headheight=1.3cm
]{geometry}
\usepackage{fancyhdr}
\pagestyle{fancy}
\fancyhf{}
\fancyhead[LE,RO]{Quiz 9, Handout 1: Response theory}
\fancyfoot[CE,CO]{\thepage}
\usepackage{url}
\makeatother
\newcommand{\resolventline}[2][1]{
  \tikz[overlay]{
      \draw[thick,flexdotted] (0,-1ex) to ++(0,#1*4.5ex) node[above,inner sep=1pt] {#2};
  }
}
\usepackage{accents}
\newcommand{\oc}[1]{\ensuremath{\accentset{\circ}{#1}}}
\newcommand{\wtl}[1]{\ensuremath{\widetilde{#1}}}
\usepackage{multicol}

\begin{document}

\setlength{\abovedisplayskip}{5pt}
\setlength{\belowdisplayskip}{5pt}


\setcounter{section}{8}
\section{Response theory}

\begin{rmk}
In the presence of a time-varying field, a molecule's electronic wavefunction is no longer simply an eigenfunction of the Hamiltonian.
Instead, its electronic structure is described by the \textit{time-dependent Schr\"odinger equation}
\begin{align}
\label{eq:schrodinger-equation}
  H(t)
  \Y(t)
=
  i
  \pd{\Y(t)}{t}
&&
  H(t)
=
  H
+
  V(t)
\end{align}
where $H$ is the usual electronic Hamiltonian and $V(t)$ is an \textit{interaction Hamiltonian} describing the energetic influence of the field.
A general series solution to equation~\ref{eq:schrodinger-equation}, known as the \textit{Dyson series}, is derived in \cref{appendix:dyson-series}.
The interaction Hamiltonian can be expressed as a sum over one-electron operators $V_\b$, representing the electronic degrees of freedom which couple to the external field, scaled by \textit{time-envelopes} $f_\b(t)$ which control the strength of the applied field over time.
\begin{align}
\label{eq:interaction-time-envelopes}
  V(t)
=
\ts{
  \sum_\b
  V_\b
  f_\b(t)
}
\end{align}
One of the most important examples is the Hamiltonian of a dipole in an electric field, which is discussed in \cref{ex:dipole-approximation} below.
The zeroth order solutions of equation~\ref{eq:schrodinger-equation} are termed \textit{stationary states}, which have the following form.\footnotemark
\footnotetext{
When $\bm{f}=\bo{0}$, the Hamiltonian loses its time-dependence and we can write $\left.\Y(t)\right|_{\bm{f}=\bo{0}}=\f(t)\Y$ where $\f(t)$ is independent of the electronic coordinates.
Substituting this into eq~\ref{eq:schrodinger-equation} and rearranging gives
$
  H\Y/\Y
=
  i\dot{\f}(t)/\f(t)
$, which equals a constant $E$ since each side depends in different variables.
Therefore, $H\Y=E\Y$ and $i\dot{\f}(t)=E\f(t)$.  Integrating the latter gives $\f(t)=e^{-iEt}$.
}
\begin{align}
  \left.
  \Y(t)
  \right|_{\bm{f}=\bo{0}}
=
  e^{-iE_kt}
  \Y_k
&&
  H\Y_k
=
  E_k\Y_k
\end{align}
As a boundary condition we assume that $V(t)$ vanishes in the past, where $\Y(t)$ is initially in the ground stationary state.
\begin{align}
\label{eq:boundary-conditions}
  \lim_{t\rightarrow-\infty}
  f_\b(t)
=
  0
&&
  \lim_{t\rightarrow-\infty}
  e^{+iHt}
  \Y(t)
=
  \Y_0
\end{align}
This limiting behavior can be enforced by introducing a complex shift in the frequency domain of $f_\b(t)$'s Fourier expansion.\footnotemark
\footnotetext{
  This is a slightly unusual convention for the Fourier transform.
  A useful mnemonic for checking these is
  $
    \int_{-\infty}^\infty
    dk\,
    e^{ikx}
  =
    2\pi\,
    \d(x)
  $.
}
\begin{align}
\label{eq:frequency-envelopes}
  f_\b(t)
=
  \int_{-\infty}^\infty
  d\w\,
  f_\b(\w_\ev)
  e^{-i\w_\ev t}
&&
  f_\b(\w_\ev)
\equiv
  (2\pi)^{-1}
  \int_{-\infty}^\infty
  dt\,
  f_\b(t)
  e^{+i\w_\ev t}
&&
  \w_\ev
\equiv
  \w
+
  i\ev
&&
  \ev
=
  |\ev|
\end{align}
This has the effect of scaling the time envelope by a damping factor $e^{\ev t}$.
For sufficiently small $\ev$, this scaled envelope will match the original one to arbitrary precision in an arbitrarily wide window about the time origin.
The fact that the interaction Hamiltonian and the coupling operators $\{V_\b\}$ are Hermitian implies the following identities.
\begin{align}
  f_\b^*(t)
=
  f_\b(t)
&&
  f_\b^*(\w_{\ev})
=
  f_\b(-\w_{-\ev})
\end{align}
\end{rmk}


\begin{ex}
\label{ex:dipole-approximation}
The dominant coupling of an electronic system to an external electric or magnetic field is mediated through its dipoles, leading to \textit{the dipole approximation}.
Quantizing the classical formulae for these interaction energies gives
\begin{align}
\begin{array}{r@{\ }l@{\ }l@{\hspace{2cm}}r@{\ }l@{\hspace{2cm}}r@{\ }l}
  V_{\bo{E}}(t)
\approx&
-\,
  \bm{\mu}\cdot\bo{E}(t)
&=
-
  \sum_\b
  \mu_\b
  \mc{E}_\b(t)
&
  \bm{\mu}
&=
  \sum_{pq}
  \ip{\y_p|\op{\bm{\mu}}\,|\y_q}
  a_p\dg a_q
&
  \op{\bm{\mu}}
&=
-
  \op{\bo{r}}
\\
  V_{\bo{B}}(t)
\approx&
-
  \bm{m}\cdot\bo{B}(t)
&=
-
  \sum_\b
  m_\b
  \mc{B}_\b(t)
&
  \bm{m}
&=
  \sum_{pq}
  \ip{\y_p|\op{\bm{m}}|\y_q}
  a_p\dg a_q
&
  \op{\bm{m}}
&=
-
  \tfr{1}{2}
  (
    \op{\bm{l}}
  +
    2\,
    \op{\bo{s}}
  )
\end{array}
\end{align}
where $\op{\bm{\mu}}$ and $\op{\bm{m}}$ are the first-quantized electric and magnetic dipole operators.\footnote{
More generally, these expressions are
$
  \op{\bm{\mu}}
=
  q_e\,\op{\bo{r}}
$,
where $q_e=-e$ is the charge of an electron,
and
$
  \op{\bm{m}}
=
  \mu_\mr{B}
  (
    g_l\,
    \op{\bm{l}}
  +
    g_{\mr{s}}\,
    \op{\bo{s}}
  )
$
where $\mu_{\mr{B}}=\tfr{1}{2}\cdot\tfr{e\hbar}{m_e}$ is the Bohr magneton and $g_l=-1$, $g_\mr{s}=-2$ are the spin and orbital \textit{$g$-factors}.
Note that the exact $g_\mr{s}$ actually deviates very slightly from $2$ due to effects arising in quantum field theory.
The orbital angular momentum operator is given by $\op{\bm{l}}=\op{\bo{r}}\times\op{\bo{p}}$ and $\op{\bo{s}}$ is the intrinsic spin angular momentum operator.
}
The leading terms neglected by the dipole approximation are quadratic in the field amplitudes.
These weaker interactions are mediated through the higher moments (quadrupole, octupole, etc.) of the charge and current distributions and may become important in symmetric molecules where certain dipole interactions are ``symmetry forbidden''.
\end{ex}





\begin{dfn}
\thmtitle{Quasi-energy}
\begin{align}
  \Y(t)
=
  e^{-i\th(t)}
  \bar{\Y}(t)
&&
  \left.
  \th(t)
  \right|_{\bm{f}=\bo{0}}
=
  E_0
  t
&&
  \lim_{t\rightarrow-\infty}
  \bar{\Y}(t)
=
  \Y_0
\end{align}
\begin{align}
  (
    H(t)
  -
    i
    \tpd{}{t}
  )
  \bar{\Y}(t)
=
  \dot{\th}(t)
  \bar{\Y}(t)
\end{align}
\begin{align}
  \dot{\th}(t)
=
  \int_0^t
  dt'
  \ip{\bar{\Y}(t')|
    H(t')
  -
    i\tpd{}{t'}
  |\bar{\Y}(t')}
\end{align}
\begin{align}
  \br{\d\bar{\Y}(t)}
    H(t)
  -
    i
    \tpd{}{t}
  \kt{\bar{\Y}(t)}
=
  \dot{\th}(t)
  \ip{\d\bar{\Y}(t)|\bar{\Y}(t)}
\end{align}
\begin{align}
  \ip{\d\bar{\Y}(t)|\bar{\Y}(t)}
+
  \ip{\bar{\Y}(t)|\d\bar{\Y}(t)}
=
  0
\end{align}
\begin{align}
  \d
  \ip{\bar{\Y}(t)|
    H(t)
  -
    i
    \tpd{}{t}
  |\bar{\Y}(t)}
+
  i
  \tpd{}{t}
  \ip{\bar{\Y}(t)|\d\bar{\Y}(t)}
=
  0
\end{align}
\end{dfn}


\newpage
\appendix

\section{Dyson series}
\label{appendix:dyson-series}

\begin{dfn}
\thmtitle{Time-evolution operator}
If we know the wavefunction at a particular time $t_0$, we can express the wavefunction at any other time as a unitary transformation of this initial state, $\Y(t)=U(t,t_0)\Y(t_0)$.
This unitary transformation is called the \textit{time-evolution operator}.
\end{dfn}


\begin{dfn}
\thmtitle{Interaction picture}
The \textit{interaction picture} results from to the following similarity transformation.
\begin{align}
  \tl{\Th}(t)
\equiv
  e^{+iHt}
  \Th(t)
&&
  \tl{W}(t)
\equiv
  e^{+iHt}
  W(t)
  e^{-iHt}
\end{align}
Expanding the Schr\"odinger equation in the interaction picture yields the the \textit{Schwinger-Tomonaga equation}.
\begin{align}
\label{eq:schwinger-tomonaga}
  \tl{V}(t)
  \tl{\Y}(t)
=
  i
  \pd{\tl{\Y}(t)}{t}
\end{align}
Multiplying both sides by $-i$ and integrating from $t_0$ to $t$ yields a recursive equation for the time-evolution operator
\begin{align}
\label{eq:integrated-schwinger-tomonaga-equation}
  \tl{\Y}(t)
-
  \tl{\Y}(t_0)
=
-
  i
  \int_{t_0}^t
  dt'\,
  \tl{V}(t')
  \tl{\Y}(t')
&&
\implies
&&
  \tl{U}(t,t_0)
=
  1
-
  i
  \int_{t_0}^t
  dt'
  \tl{V}(t')\,
  \tl{U}(t',t_0)
\end{align}
and infinite recursion of this identity leads to the following expansion.
\begin{align}
\label{eq:time-evolution-infinite-recursion}
  \tl{U}(t,t_0)
=
  \sum_{n=0}^\infty
  (-i)^n
  \int_{t_0}^t
  dt_1
  \int_{t_0}^{t_1}
  dt_2
  \cd
  \int_{t_0}^{t_{n-1}}
  dt_n
  \,
  \tl{V}(t_1)
  \cd
  \tl{V}(t_n)
\end{align}
\end{dfn}

\begin{dfn}
\label{dfn:time-ordering}
\thmtitle{Time-ordering}
Let 
$
  \tl{q}_1(t_1)\cd \tl{q}_n(t_n)
$
be a string of particle-hole operators in the interaction picture.\footnotemark
\footnotetext{
As in
$
  \tl{q}(t)
\equiv
  e^{+iHt}
  q
  e^{-iHt}
$
for some
$q\in\{a_p\}\cup\{a_p\dg\}$.
}
The \textit{time-ordering map} takes this string into
$
  \mc{T}\{
  \tl{q}_1(t_1)\cd \tl{q}_n(t_n)
  \}
\equiv
  \e_\pi \,\tl{q}_{\pi(1)}(t_{\pi(1)})\cd \tl{q}_{\pi(n)}(t_{\pi(n)})
$,
where $\pi\in\mr{S}_n$ is a permutation that puts the time arguments in chronological order, $t_{\pi(1)}>\cd>t_{\pi(n)}$.
\end{dfn}

\begin{ntt}
Let us define the following notation for multivariate integrals by analogy with multi-index summations.\footnotemark
\footnotetext{
Compare these integrals to
$
  \sum_{i_1i_2i_3\cd}^{\{n_0,\ld,n\}}
$
and
$
  \sum_{i_1>i_2>i_3\cd}^{\{n_0,\ld,n\}}
$.
The subscript defines the summation variables, along with any conditions restricting their values, and the superscript indicates the allowed range of values for each variable.
}
\begin{align}
  \int_{t_1t_2t_3\ld}^{[t_0,t]}
  dt_1dt_2dt_3\cd
\equiv
  \int_{t_0}^t
  dt_1
  \int_{t_0}^t
  dt_2
  \int_{t_0}^t
  dt_3
  \cd
&&
  \int_{t_1>t_2>t_3>\cd}^{[t_0,t]}
  dt_1dt_2dt_3\cd
\equiv
  \int_{t_0}^t
  dt_1
  \int_{t_0}^{t_1}
  dt_2
  \int_{t_0}^{t_2}
  dt_3
  \cd
\end{align}
This notation should elucidate the following identity, 
which breaks an unrestricted integral into all possible chronologies.\footnotemark
\footnote{
  The corresponding summation identity would be
$
  \sum_{i_1\neq i_2\neq i_3\neq\cd}^{\{n_0,n\}}
=
  \sum_{\pi}^{\mr{S}_n}
  \sum_{i_{\pi(1)}> i_{\pi(2)}> i_{\pi(3)}>\cd}^{\{n_0,n\}}
$.
  The unrestricted integral is equivalent to an integral over $t_1\neq t_2\neq t_3\neq\cd$ because individual integrand values have ``measure zero'':
$
  \int_{t_j}^{t_j}
  dt_i
=
  0
$.
}
\begin{align}
\label{eq:integral-identity}
  \int_{t_1\cd t_n}^{[t_0,t]}
  dt_1\cd t_n\,
  f(t_1\cd t_n)
=
  \sum_\pi^{\mr{S}_n}
  \int_{t_{\pi(1)}>\ld>t_{\pi(n)}}^{[t_0,t]}
  dt_1\cd t_n\,
  f(t_1\cd t_n)
\end{align}
\end{ntt}


\begin{prop}
\thmtitle{The Dyson series}
\thmstatement{
If $\tl{V}(t)$ is particle-number consering, then
$
  \tl{U}(t,t_0)
=
  \mc{T}\{
    e^{
    -
      i
      \int_{t_0}^t
      dt'\,
      \tl{V}(t')
    }
  \}
$.
}\vspace{5pt}
\thmproof{
   Expanding the time-ordered exponential in a Taylor series and applying equation~\ref{eq:integral-identity} gives the following
\begin{align}
  \sum_{n=0}^\infty
  \fr{(-i)^n}{n!}
  \int_{t_1\cd t_n}^{[t_0,t]}
  dt_1\cd dt_n\,
  \mc{T}\{
    \tl{V}(t_1)
    \cd
    \tl{V}(t_n)
  \}
=
  \sum_{n=0}^\infty
  \fr{(-i)^n}{n!}
  \sum_\pi^{\mr{S}_n}
  \int_{t_{\pi(1)}>\ld>t_{\pi(n)}}^{[t_0,t]}
  dt_1\cd dt_n\,
  \mc{T}\{
    \tl{V}(t_1)
    \cd
    \tl{V}(t_n)
  \}
\end{align}
which simplifies to equation~\ref{eq:time-evolution-infinite-recursion} because all $n!$ terms in the sum over chronologies are equal by \cref{dfn:time-ordering}.
}
\end{prop}

\begin{rmk}
Assuming the boundary conditions of eq~\ref{eq:boundary-conditions}, the Dyson series for the wavefunction is
\begin{align}
\label{eq:wavefunction-dyson-series}
  \tl{\Y}(t)
=
  \lim_{t_0\rightarrow-\infty}
  \tl{U}(t,t_0)
  \Y(t_0)
=
  \sum_{n=0}^\infty
  \fr{(-i)^n}{n!}
  \int_{\mb{R}^n}
  dt_1\cd dt_n\,
  \th(t-t_1)
  \cd
  \th(t-t_n)\,
  \mc{T}\{
    \tl{V}(t_1)
    \cd
    \tl{V}(t_n)
  \}
  \Y_0
\end{align}
where
$
  \th(x)
=
  \int_{-\infty}^x
  dx'
  \d(x')
$
is the Heaviside step function, which here enforces the upper limits of integration.
\end{rmk}



\newpage
\section{Response functions}
\label{appendix:response-functions}


\begin{dfn}
\thmtitle{Response functions}
Any quantity $X(t)$ which depends on the time-envelopes $\{f_\b(t)\}$ can be expanded in a Taylor series.
The expansion coefficients in this series are called the \textit{response functions} of $X(t)$.
\begin{align}
\label{eq:general-perturbation-expansion}
  X(t)
=
  \sum_{n=0}^\infty
  \fr{1}{n!}
  \sum_{\b_1,\ld,\b_n}
  \int_{\mb{R}^n}
  dt_1\cd t_n\,
  f_{\b_1}(t_1)
  \cd
  f_{\b_n}(t_n)\,
  X^{\b_1\cd \b_n}_{t;t_1\cd\,t_n}
&&
  X^{\b_1\cd \b_n}_{t;t_1\cd\,t_n}
\equiv
  \left.
  \fd{^n
    X(t)
  }{
    f_{\b_1}(t_1)
    \cd
    df_{\b_n}(t_n)
  }
  \right|_{\bm{f}=\bo{0}}
\end{align}
\end{dfn}

\begin{ex}
Substituting equation~\ref{eq:interaction-time-envelopes} into equation~\ref{eq:wavefunction-dyson-series} and comparing the result to equation~\ref{eq:general-perturbation-expansion} implies the following.
\begin{align}
\label{eq:general-wavefunction-response}
  \tl{\Y}^{\b_1\cd \b_n}_{t; t_1\cd t_n}
=
  (-i)^n
  \th(t-t_1)
  \cd
  \th(t-t_n)\,
  \mc{T}\{
    \tl{V}_{\b_1}(t_1)
  \cd
    \tl{V}_{\b_n}(t_n)
  \}
  \Y_0
\end{align}
Defining $\ta_i\equiv t_i-t$, we find that wavefunction responses transform as follows when we move the time origin to $t$.\footnotemark
\footnotetext{
  This follows from $\th(t-t_i)=\th(0-\ta_i)$ and
  $
    \tl{V}_{\b_i}(\ta_i)
  =
    e^{-iHt}\tl{V}_{\b_i}(t_i)e^{+iHt}
  \implies
    \tl{V}_{\b_1}(\ta_1)
    \cd
    \tl{V}_{\b_n}(\ta_n)
  =
    e^{-iHt}
    \tl{V}_{\b_1}(t_1)
    \cd
    \tl{V}_{\b_n}(t_n)
    e^{+iHt}
  $.
}
\begin{align}
\label{eq:wavefunction-time-shift}
  \tl{\Y}^{\b_1\cd \b_n}_{0; \ta_1\cd \ta_n}
=
  e^{-i(H - E_0)t}\,
  \tl{\Y}^{\b_1\cd \b_n}_{t; t_1\cd t_n}
\end{align}
\end{ex}


\begin{dfn}
\thmtitle{Property response functions}
Response functions for the expectation value of an observable property $W$ are usually denoted with the following double-brackets notation.
\begin{align}
  \iip{\tl{W}(t); \tl{V}_{\b_1}(t_1),\ld,\tl{V}_{\b_n}(t_n)}
\equiv
  \left.
  \fd{^n
    \ip{\Y(t)|W|\Y(t)}
  }{
    f_{\b_1}(t_1)
    \cd
    df_{\b_n}(t_n)
  }
  \right|_{\bm{f}=\bo{0}}
\end{align}%
In some contexts, these \textit{property response functions} are known as \textit{retarded propagators} or \textit{retarded Green's functions}.

\end{dfn}

\begin{ex}
Substituting the response-function expansion of the wavefunction into $\ip{\Y(t)|W|\Y(t)}=\ip{\tl{\Y}(t)|\tl{W}(t)|\tl{\Y}(t)}$ and grouping powers of $\bm{f}$ gives the following expression for property response functions.
\begin{align}
\label{eq:general-linear-response}
  \iip{\tl{W}(t); \tl{V}_{\b_1}(t_1),\ld,\tl{V}_{\b_n}(t_n)}
=
  \sum_{p=0}^n
  \fr{1}{p!(n-p)!}
  \sum_\pi^{\mr{S}_n}
  \ip{
    \tl{\Y}_{t;t_{\pi(1)}\cd t_{\pi(p)}}
           ^{\b_{\pi(1)}\cd\b_{\pi(p)}}
  |
    \tl{W}(t)
  |
    \tl{\Y}_{t;t_{\pi(p+1)}\cd t_{\pi(n)}}
           ^{\b_{\pi(p+1)}\cd \b_{\pi(n)}}
  }
\end{align}
Using equation~\ref{eq:wavefunction-time-shift} and $\tl{W}(t)=e^{-iHt}\tl{W}(0)e^{+iHt}$, we can show that the property responses are invariant to time translation.
\begin{align}
  \iip{\tl{W}(0); \tl{V}_{\b_1}(\ta_1),\ld,\tl{V}_{\b_n}(\ta_n)}
=
  \iip{\tl{W}(t); \tl{V}_{\b_1}(t_1),\ld,\tl{V}_{\b_n}(t_n)}
\end{align}
\end{ex}

\begin{prop}
\label{prop:linear-response-commutator-expression}
\thmtitle{Linear property response function}
\thmstatement{
$\ds{
  \iip{\tl{W}(t);\tl{V}_\b(t')}
=
-
  i
  \th(t-t')
  \ip{\Y_0|[\tl{W}(t), \tl{V}_\b(t')]|\Y_0}
}$
}\vspace{6pt}
\thmproof{
This follows from equations~\ref{eq:general-wavefunction-response} and \ref{eq:general-linear-response} with $n=1$.
}
\end{prop}

\begin{samepage}
\begin{cor}
\thmstatement{
Defining $\w_k\equiv E_k-E_0$ and $\ta\equiv t'-t$, the linear property reponse can be expressed as follows.
\begin{align*}
  \iip{\tl{W}(t);\tl{V}_\b(t')}
=
-
  i
  \th(-\ta)
  \sum_{k=0}^\infty
  (
    e^{+i\w_k \ta}
    \ip{\Y_0|W|\Y_k}
    \ip{\Y_k|V_\b|\Y_0}
  -
    e^{-i\w_k \ta}
    \ip{\Y_0|V_\b|\Y_k}
    \ip{\Y_k|W|\Y_0}
  )
\end{align*}
}\thmproof{
Expanding the interaction-picture operators of \cref{prop:linear-response-commutator-expression} in the Schr\"odinger picture yields the following
\begin{align}
  \iip{\tl{W}(t);\tl{V}_\b(t')}
=&\
-
  i
  \th(t-t')
  (
    \ip{\Y_0|We^{-i(H-E_0)(t-t')}V_\b |\Y_0}
  -
    \ip{\Y_0|V_\b e^{-i(H-E_0)(t'-t)}W|\Y_0}
  )
\end{align}
since $H\Y_0=E_0\Y_0$.
The proposition follows from a spectral resolution of $e^{\mp(H-E_0)(t-t')}$ in each term.
}
\end{cor}
\end{samepage}

\begin{dfn}
\thmtitle{Response functions (frequency domain)}
The frequency-domain response functions of $X(t)$ at $t=0$ are defined as $\ev$-shifted Fourier transforms of the time-domain response functions with respect to $\ta_1,\ld,\ta_n$.
\begin{align}
  X^{\b_1\cd \b_n}_{0;\ta_1\cd \ta_n}
=
  (2\pi)^{-n}
  \int_{\mb{R}^n}
  d\w_1\cd d\w_n\,
  X^{\b_1\cd \b_n}_{\w_{\ev,1}\cd\w_{\ev,n}}
  e^{+i\sum_j\w_{\ev,j}\ta_j}
&&
  X^{\b_1\cd \b_n}_{\w_{\ev,1}\cd\w_{\ev,n}}
\equiv
  \int_{\mb{R}^n}
  d\ta_1\cd d\ta_n\,
  X^{\b_1\cd \b_n}_{0;\ta_1\cd \ta_n}
  e^{-i\sum_j\w_{\ev,j}\ta_j}
\end{align}
From equations~\ref{eq:frequency-envelopes} and~\ref{eq:general-perturbation-expansion}, we find that these are coefficients in the frequency-envelope Taylor expansion of $X(0)$.
\begin{align}
  X(0)
=
  \sum_{n=0}^\infty
  \fr{1}{n!}
  \sum_{\b_1\cd\b_n}
  \int_{\mb{R}^n}
  d\w_1\cd d\w_n\,
  f_{\b_1}(\w_{\ev,1})\cd
  f_{\b_n}(\w_{\ev,n})\,
  X^{\b_1\cd \b_n}_{\w_{\ev,1}\cd \w_{\ev,n}}
&&
  X^{\b_1\cd \b_n}_{\w_{\ev,1}\cd \w_{\ev,n}}
=
  \left.
  \fd{^n\,X(0)}{f_{\b_1}(\w_{\ev,1})\cd df_{\b_n}(\w_{\ev,n})}
  \right|_{\bm{f}=\bo{0}}
\end{align}
Property response functions in the frequency domain are denoted by $\iip{W; V_{\b_1},\ld,V_{\b_n}}_{\w_{\ev,1}\cd\w_{\ev,n}}$, which can be written as a Fourier transform of
$
  \iip{\tl{W}(t); \tl{V}_{\b_1}(t_1),\ld,\tl{V}_{\b_n}(t_n)}
$
itself due to its translational invariance.
\Cref{prop:property-response-frequency-expansion} shows that these frequency-domain functions can be used to expand $\ip{\Y(t)|W|\Y(t)}$ away from the time origin.
\end{dfn}


\begin{prop}
\label{prop:property-response-frequency-expansion}
\thmstatement{
The expectation value of an observable $W$ at time $t$ is given by the following.
\begin{align*}
  \ip{\Y(t)|W|\Y(t)}
=
  \sum_{n=0}^\infty
  \fr{1}{n!}
  \sum_{\b_1,\ld,\b_n}
  \int_{\mb{R}^n}
  d\w_1\cd d\w_n\,
  f_{\b_1}(\w_{\ev,1})
  \cd
  f_{\b_n}(\w_{\ev,n})\,
  \iip{W; V_{\b_1},\ld,V_{\b_n}}_{\w_{\ev,1}\cd\w_{\ev,n}}\,
  e^{-i\sum_j\w_{\ev,j}t}
\end{align*}
}%
\thmproof{
This follows from substituting equation~\ref{eq:frequency-envelopes} into the time-envelope expansion and inserting $e^{-i\sum_j\w_{\ev,j}t} e^{+i\sum_j\w_{\ev,j}t}$.
}
\end{prop}


\begin{rmk}
\begin{align*}
\end{align*}
\end{rmk}

\begin{align}
  \iip{\tl{W}(t);\tl{V}_\b(t')}
=&\
  \sum_{k=0}^\infty
  (
    g_k^+(\ta)
    \ip{\Y_0|W|\Y_k}
    \ip{\Y_k|V_\b|\Y_0}
  -
    g_k^-(\ta)
    \ip{\Y_0|V_\b|\Y_k}
    \ip{\Y_k|W|\Y_0}
  )
&&
  g_k^{\pm}(\ta)
\equiv
-
  i
  \th(-\ta)
    e^{\pm i\w_k \ta}
\end{align}
\begin{align}
  g_k^\pm(\w_\ev)
=&\
  \int_{-\infty}^\infty
  d\ta\,
  g_k^\pm(\ta)
  e^{-i\w_\ev \ta}
=
-
  i
  \int_{-\infty}^0
  d\ta\,
  e^{-i(\w_\ev \mp \w_k) \ta}
=
  \fr{
    1
  }{
    \w_\ev \mp \w_k
  }
\\
  g_k^{\pm}(\ta)
=&\
  \fr{1}{2\pi}
  \int_{-\infty}^{\infty}
  d\w\,
  g_k^\pm(\w_\ev)
  e^{+i\w_\ev \ta}
=
  \fr{1}{2\pi}
  \int_{-\infty}^{\infty}
  d\w\,
  \fr{
    e^{+i\w_\ev \ta}
  }{
    \w_\ev \mp \w_k
  }
\end{align}

\newpage
\section{Complex analysis}

\begin{dfn}
\thmtitle{Continuity}
A complex-valued function $f$ is said to be \textit{continuous} at $z\in\mb{C}$ if for any positive real number $\ev$ we can choose a radius $\d>0$ such that all complex values $z'$ within $\d$ of $z$ satisfy $|f(z')-f(z)|<\ev$.
That is, we can always choose a circle small enough that all function values lie within some threshold.
\end{dfn}

\begin{dfn}
\label{dfn:holomorphic-function}
\thmtitle{Holomorphic function}
The function $f(z)$ is \textit{differentiable} at $z$ if the following limit exists and
has the same value with $h$ approaching from any direction in the complex plane.
\begin{align}
  \pd{f(z)}{z}
\equiv
  \lim_{h\rightarrow0}
  \fr{f(z+h)-f(z)}{h}
\end{align}
A \textit{holomorphic function} is a complex-valued function which is differentiable everywhere on $\mb{C}$.
\end{dfn}

\begin{dfn}
\thmtitle{Wirtinger derivatives}
Denoting the real and imaginary components of $z$ by $x$ and $y$ we find
\begin{align}
  z
=
  x
+
  iy
&&
\implies
&&
  dz
=
  dx
+
  idy
,\ \ 
  dz^*
=
  dx
-
  idy
&&
\implies
&&
  dx
=
  \tfr{1}{2}
  \pr{
    dz
  +
    dz^*
  }
,\ \ 
  dy
=
  \tfr{1}{2i}
  \pr{
    dz
  -
    dz^*
  }
\end{align}
by adding and subtracting differentials.
Comparing these to the total derivative expansion for each variable, we find
\begin{align}
  \pd{x}{z}
=
  \fr{1}{2}
&&
  \pd{x}{z^*}
=
  \fr{1}{2}
&&
  \pd{y}{z}
=
  \fr{1}{2i}
&&
  \pd{y}{z^*}
=
-
  \fr{1}{2i}
\end{align}
which can lead to the following formulas for derivatives with respect to $z$ and $z^*$, known as \textit{Wirtinger derivatives}.
\begin{align}
  \pd{}{z}
=
  \pd{x}{z}
  \pd{}{x}
+
  \pd{y}{z}
  \pd{}{y}
=
  \fr{1}{2}
  \pr{
    \pd{}{x}
  -
    i
    \pd{}{y}
  }
&&
  \pd{}{z^*}
=
  \pd{x}{z^*}
  \pd{}{x}
+
  \pd{y}{z^*}
  \pd{}{y}
=
  \fr{1}{2}
  \pr{
    \pd{}{x}
  +
    i
    \pd{}{y}
  }
\end{align}
These can be used to show that $\pd{z^*}{z}=\pd{z}{z^*}=0$, confirming that $z$ and $z^*$ are independent variables.
\end{dfn}

\begin{prop}
\thmstatement{
  The function $f$ is differentiable at $z$ if and only if
  $
    \dpd{f(z)}{z^*}
  =
    0
  $.
}
\thmproof{
  Let $z=x+iy$ and assume the derivatives with respect to $x$ and $y$ exist.
  Then we can express $f(z+h)-f(h)$ as a bivariate Taylor expansion in $\mr{Re}(h)$ and $\mr{Im}(h)$, whose linear term is given by the following.
\begin{align*}
  \pd{f(z)}{x}
  \mr{Re}(h)
+
  \pd{f(z)}{y}
  \mr{Im}(h)
=
  \pd{f(z)}{x}
  \fr{
    h + h^*
  }{
    2
  }
+
  \pd{f(z)}{y}
  \fr{
    h - h^*
  }{
    2i
  }
\end{align*}
  Dividing this expression by $h$ and taking the limit as $h\rightarrow 0$ gives the complex derivative of $f$ at $z$.
\begin{align*}
  \lim_{h\rightarrow0}
  \fr{f(z+h)-f(z)}{h}
=
  \fr{1}{2}
  \pr{
    \pd{f(z)}{x}
  +
    \fr{1}{i}
    \pd{f(z)}{y}
  }
+
  \fr{1}{2}
  \pr{
    \pd{f(z)}{x}
  -
    \fr{1}{i}
    \pd{f(z)}{y}
  }
  \lim_{h\rightarrow h^*}
  \fr{h^*}{h}
\end{align*}
  If $h$ approaches along the real axis, the limit of $h^*/h$ is $+1$.
  If $h$ approaches along the imaginary axis, the limit of $h^*/h$ is $-1$.
  Therefore, $f$ is differentiable if and only if
  $
    \fr{1}{2}
    \pr{
      \pd{f(z)}{x}
    -
      \fr{1}{i}
      \pd{f(z)}{y}
    }
  =
    0
  $,
  which is equivalent to
  $
    \pd{f(z)}{z^*}
  =
    0
  $.
}
\end{prop}

\begin{ntt}
\thmtitle{Complex integration}
The notation
$
  \int_\g
  dz\,
  f(z)
$
denotes the line integral of $f$ over a path $\g$ in the complex plane, which is known as \textit{contour integration}.
The notation
$
  \oint_\g
  dz\,
  f(z)
$
means that $\g$ is a closed and counterclockwise. 
\end{ntt}

\begin{prop}
\thmstatement{
  If $\g$ is a circular path containing the point $z$, then
$\ds{
  \oint_\g
  dz'\,
  \fr{1}{z'-z}
=
  2\pi i
}$.
}
\thmproof{
  We can assume without loss of generality that $z$ is at the origin and parametrize the path as $z'(\th)=re^{i\th}$.
  Using
  $
    dz'(\th)
  =
    ire^{i\th}
    d\th
  $,
  the integrand simplifies to
  $
    dz'(\th)/
    z'(\th)
  =
    id\th
  $.
  Integrating from $0$ to $2\pi$ completes the proof.
}
\end{prop}




\newpage
\section{Differential geometry}

\begin{dfn}
\thmtitle{Topological space}
Let $\mc{O}_X$ be a collection of subsets from the set $X$.
This collection is a \textit{topology} if%
\\[5pt]
\begin{tabular}{@{\hspace{\parindent}}r@{\ \ }p{0.5\linewidth}p{0.4\linewidth}}
  1.
&
  $\mc{O}_X$ contains the empty set and $X$ itself.
&
  $\O\in \mc{O}_X,\ X\in \mc{O}_X$
\\[5pt]
  2.
&
  Any finite or infinite union of sets in $\mc{O}_X$ is in $\mc{O}_X$.
&
  $\ds{
    \mc{S}
  \subseteq
    \mc{O}_X
  \implies
    \cup_{O\in \mc{S}}
    O
  \in
    \mc{O}_X
  }$
\\[5pt]
  3.
&
  Any finite intersection of sets in $\mc{O}_X$ is in $\mc{O}_X$.
&
  $
    \{O_1,\ld,O_n\}
  \subseteq
    \mc{O}_X
  \implies
    O_1\cap\cd\cap O_n
  \in
    \mc{O}_X
  $
\end{tabular}\\[5pt]
in which case we call $(X,\mc{O}_X)$ a \textit{topological space}.
The elements of $X$ are called \textit{points} and the elements of $\mc{O}_X$ are called \textit{open subsets}.
The set complements of open subsets are \textit{closed subsets}.
An arbitrary subset $N\subseteq X$, not necessarily open, qualifies as a \textit{neighborhood of the point $p$} if it contains an open subset containing $p$.\footnotemark
\footnotetext{
  That is, there is some $O\in \mc{O}_X$ such that $p\in O\subseteq V$.
}
\end{dfn}

\begin{dfn}
Let $V$ be an open or closed subset of $X$.
Then any point $p$ in the space falls into one of three categories.%
\\[5pt]
\begin{tabular}{@{\hspace{\parindent}}r@{\ \ }p{0.6\linewidth}p{0.35\linewidth}}
  1.
&
  $V$ contains an open subset containing $p$.
&
  $
  \exists\,
    O
  \in
    \mc{O}_X
  :
    p
  \in
    O
  \subset
    V
  $
\\[5pt]
  2.
&
  The complement of $V$ contains an open subset containing $p$.
&
  $
  \exists\,
    O
  \in
    \mc{O}_X
  :
    p
  \in
    O
  \subset
    X\setminus V
  $
\\[5pt]
  3.
&
  Any open subset containing $p$ intersects both $V$ and its complement.
&
  $
    p
  \in
    O
  \in
    \mc{O}_X
  \implies
  \left\{
  \begin{array}{@{}l@{}}
    O
  \cap
    V
  \neq
    \O
  \ \text{ and}
  \\
    O
  \cap
    X\setminus V
  \neq
    \O
  \end{array}
  \right\}
  $
\end{tabular}\\[5pt]
Points in the first and second categories constitute the \textit{interior} and the \textit{exterior} of $V$.
Points in the third category constitute the \textit{boundary} of $V$, which is denoted $\pt V$.
An open subset is equal to its interior set, whereas a closed subset is given by the union of its interior and boundary sets.
\end{dfn}

\begin{dfn}
\thmtitle{Hausdorff condition}
Two points $p$ and $p'$ are \textit{distinct} if $p\neq p'$.
Two points are \textit{separated by neighborhoods} if they have any neighborhoods that are disjoint from each other.
A topological space in which all distinct points are separated by neighborhoods is said to satisfy the \textit{Hausdorff condition}.
\end{dfn}

\begin{dfn}
\thmtitle{Base}
A collection of open subsets $\mc{B}\subset \mc{O}_X$ is termed a \textit{base} for the topology if every member of $\mc{O}_X$ can be written as a union of the elements of $\mc{B}$.
We say that $\mc{O}_X$ is the topology \textit{generated} by $\mc{B}$.
\end{dfn}

\begin{dfn}
\thmtitle{Euclidean space}
The \textit{open ball of radius $r>0$ at $\bo{x}\in\mb{R}^n$} is defined as $\mb{B}^n_r(\bo{x})\equiv\{\bo{x}'\in\mb{R}^n\,|\,\|\bo{x}'-\bo{x}\|<r\}$, which is an open subset of $\mb{R}^n$.
The \textit{Euclidean topology} on $\mb{R}^n$ is the topology generated from the set of open balls.
\begin{align}
  \mc{O}_{\mb{R}^n}
\equiv
  \{
    \cup_{O\in \mc{S}}
    O
  \,|\,
    \mc{S}
  \subseteq
    \mc{B}
  \}
&&
  \mc{B}
=
  \{
    \mb{B}^n_r(\bo{x})
  \,|\,
    r>0,
    \bo{x}\in\mb{R}^n
  \}
\end{align}
The resulting topological space, $(\mb{R}^n, \mc{O}_{\mb{R}^n})$, is called \textit{Euclidean $n$-space}.
\end{dfn}


\begin{dfn}
\thmtitle{Homeomorphism}
An invertible map $\mu:X\rightarrow X'$ that takes open subsets of $X$ into open subsets of $X'$ and vice versa is called \textit{bicontinuous}.
A bicontinuous map whose image covers the codomain is called a \textit{homeomorphism}.
\end{dfn}

\begin{dfn}
\thmtitle{Local homeomorphism}
A \textit{local homeomorphism} is map $\la:X\rightarrow X'$ for which every point in $X$ is a member of at least one open subset $O$ whose restriction $\la|_O$ is a homeomorphism onto an open subset of $X'$.
\end{dfn}

\begin{dfn}
\thmtitle{Topological manifold}
An \textit{$n$-dimensional topological manifold} is a topological space  $(M, \mc{O}_M)$ that is locally homeomorphic to Euclidean $n$-space and satisfies the Hausdorff condition.
\end{dfn}

\begin{prop}
\thmstatement{
  Euclidean $n$-space is a topological manifold.
}
\thmproof{
  Since we are dealing with a Euclidean space, we only need to show that it satisfies the Hausdorff condition.
  By identity of indiscernibles,\footnotemark
\footnotetext{
  \url{https://en.wikipedia.org/wiki/Metric_(mathematics)}
}
the points $\bo{x}$ and $\bo{x}'$ are distinct if and only if $r=\|\bo{x}-\bo{x}'\|/2$ is a positive number.
  If so, then $\mb{B}^n_r(\bo{x})$ and $\mb{B}^n_r(\bo{x}')$ are disjoint open subsets containing each and they are separated by neighborhoods.
}
\end{prop}

\begin{dfn}
\thmtitle{Chart}
A \textit{chart} $(O, \xi)$ consists of a map to Euclidean space $\xi:M\rightarrow\mb{R}^n$ together with an open subset $O$ on which this map is a homeomorphism.
This is also called a \textit{local coordinate frame} because it relates each point $p$ in a region of the manifold to a coordinate vector $\xi(p)=\bo{x}$ in $\mb{R}^n$.
\end{dfn}

\begin{dfn}
\thmtitle{Transition map}
A transition map defines the conversion between overlapping coordinate frames.
If $(O, \xi)$ and $(O', \xi')$ are overlapping charts, their transition map is defined by $\ta(\bo{x})=\xi'(\xi^{-1}(\bo{x}))$, which takes the image of their overlap under one chart, $\xi(O\cap O')\subseteq \mb{R}^n$, into the image of their overlap under the second chart, $\xi'(O\cap O')\subseteq \mb{R}^n$.
\end{dfn}

\begin{dfn}
\thmtitle{Differentiability class}
Consider a function or mapping that is compatible with differential calculus.
If its $k\eth$ derivatives are continuous, we say that the function belongs to \textit{differentiability class $C^k$}.
\end{dfn}


\begin{dfn}
\thmtitle{Differential structure}
An \textit{atlas} is a collection of charts $\mc{A}\equiv\{(O_\a, \xi_\a)\}$ covering a manifold,
$
  \cup_{O\in\mc{A}}
  O
=
  M
$.
We say that $\mc{A}$ is $C^k$-differentiable if all of its transition maps are.
Two $C^k$-differentiable atlases are considered \textit{compatible}, $\mc{A}\sim\mc{A}'$, if their union is also $C^k$ differentiable.
The equivalence class $[\mc{A}]\equiv\{\mc{A}'\,|\,\mc{A}\sim\mc{A}'\}$ of atlases compatible with a $C^k$ atlas $\mc{A}$ constitutes a \textit{$C^k$ differential structure}.
Each differential structure on a manifold is associated with a unique \textit{maximal atlas} containing all of the members of its equivalence class.
\end{dfn}

\begin{samepage}
\begin{dfn}
\thmtitle{Differentiable manifold}
A manifold with a $C^k$ differential structure is called a \textit{$C^k$-differentiable manifold}.
A real-valued function on the manifold $f:M\rightarrow\mb{R}$ is categorized as $C^{k'}$ differentiable according to the lowest differentiability class of $f(\xi^{-1}(\bo{x}))$ for the charts in its differential structure.
The set of manifold functions in a given differentiability class is denoted $C^{k'}(M)$.
Infinitely differentiable manifolds and manifold functions are called \textit{smooth}.
\end{dfn}
\end{samepage}


\begin{samepage}
\begin{dfn}
\thmtitle{Tangent space}
A \textit{tangent at $p$} is a real functional on $C^\infty(M)$ with the following characteristics.
\begin{align}
\label{eq:tangent-properties}
  \mc{D}_p
:
  C^\infty(M)
\rightarrow
  \mb{R}
&&
  \mc{D}_p(c\,f + c'\,f')
=
  c\,\mc{D}_p(f)
+
  d\,\mc{D}_p(g)
&&
  \mc{D}_p(ff')
=
  \mc{D}_p(f)\cdot f'(p)
+
  f(p) \cdot \mc{D}_p(f')
\end{align}
In words, tangents are linear functionals on $C^\infty(M)$ that satisfy the product rule of differential calculus, which is called the \textit{product rule of derivations}.
The set of tangents at a point constitutes its \textit{tangent space}, which is denoted $T_pM$.
\end{dfn}
\end{samepage}

\begin{ex}
A general tangent vector at a point $p$ can be defined as the \textit{directional derivative} along a path $\g:\mb{R}\rightarrow M$ containing the point.
If $\g(t)=p$, the directional derivative along $\g$ at $p$ is the functional
$
  (\pt_\g)_p
:
  C^\infty
\rightarrow
  \mb{R}
$
defined by
\begin{align}
  (\pt_\g)_p(f)
\equiv
  \pd{}{t}
  (f\circ \g)(t)
=
  \pd{f(g(t))}{t}
\end{align}
which can be evaluated by ordinary differential calculus because the composed function $(f\circ \g)$ is a real-valued function of a real parameter.
On an $n$-manifold one can always find a set of paths $\{\g_i:\mb{R}\rightarrow M\,|\,\g_i(t)=p\}$ such that
$
  \{(\pt_{\g_1})_p,\ld,(\pt_{\g_n})_p\}
$
is linearly independent, and therefore serves as a basis for $T_pM$.
\end{ex}

\begin{ex}
Directional derivatives along the coordinate axes provide the \textit{standard basis} for Euclidean tangent spaces.
\begin{align}
\label{eq:coordinate-directional-derivative}
  T_\bo{x}\mb{R}^n
=
  \mr{span}
  \{
    (\pt_{x_1})_{\bo{x}}
  ,
    \ld
  ,
    (\pt_{x_n})_{\bo{x}}
  \}
&&
  (\pt_{x_i})_{\bo{x}}(f)
\equiv
  \pr{\pd{f}{x_i}}_\bo{x}
\end{align}.
\end{ex}

\begin{dfn}
\thmtitle{Cotangent space}
The \textit{cotangent space at $p$}, denoted $T_p^*M$, is the space of linear functionals on~$T_pM$.
\begin{align}
  \a_p
:
  T_pM
\rightarrow
  \mb{R}
&&
  \a_p
  (c\,\mc{D}_p + c'\,\mc{D}_p')
=
  c\,
  \a_p(\mc{D}_p)
+
  c'\,
  \a_p(\mc{D}_p')
\end{align}
That is, each cotangent $\a_p$ takes a tangent at $p$ as its argument and spits out a real number, 
\end{dfn}

\begin{ex}
If
$
  \{(\pt_{\g_1})_p,\ld,(\pt_{\g_n})_p\}
$
\end{ex}

\begin{dfn}
\label{eq:differential-map-on-0-forms}
\thmtitle{Differential map}
The \textit{differential map}, $(d\,\cdot)_p:C^\infty(M)\rightarrow T_p^*(M)$, takes smooth functions into cotangents at $p$ according the the following rule:
$(df)_p(\mc{D}_p)\equiv\mc{D}_p(f)$.
In words, the differential of a function at $p$ maps tangents into their value on on that function at $p$.
From the properties of tangents, one can show that this map is linear,
$
  d(cf + c'f')_p
=
  c(df)_p
+
  c'(df')_p
$
, and satisfies its own form of product rule,
$
  d(fg)
=
  (df)_p\cdot g(p)
+
  f(p)\cdot (dg)_p
$.\footnotemark
\footnotetext{
  Pointwise multiplication of cotangents and functions is defined by $(g(p)\cdot(df)_p)(\mc{D}_p)\equiv g(p)\cdot(df)_p(\mc{D}_p)$.
}
\end{dfn}

\begin{dfn}
\label{eq:exact-differential}
\thmtitle{Exact differential}
A cotangent $\a_p\in T_p^*M$ which can be written as the differential of a function, $\a_p=(df)_p$, is called an \textit{exact differential}.
All other cotangents are \textit{inexact differentials}.
\end{dfn}


\begin{ex}
As an example of a cotangent, consider the coordinate differential $(dx_i)_\bo{x}.$
Acting on $(\pt_{x_j})_\bo{x}$ gives
\begin{align}
\label{eq:coordinate-independence}
  (dx_i)_\bo{x}
  ((\pt_{x_j})_\bo{x})
=
  \pr{\pd{x_i}{x_j}}_\bo{x}
=
  \d_{ij}
\end{align}
which, by linearity, can be used to determine its action on other tangents in $T_\bo{x}\mb{R}^n$.
The set $\{(dx_i)_\bo{x}\,|\,1\leq i\leq n\}$  of coordinate differentials makes up the \textit{standard basis} for $T_\bo{x}^*\mb{R}^n$.
\end{ex}


\begin{ex}
\label{ex:tangent-cotangent-expansions}
General tangents and cotangents $\bo{x}\in\mb{R}^n$ are linear combinations of coordinate derivatives and differentials.
\begin{align}
\label{eq:tangent-cotangent-expansions}
  \mc{D}_\bo{x}
=
\ts{
  \sum_i
  D_\bo{x}^i\,
  (\pt_{x_i})_\bo{x}
}
&&
  \a_\bo{x}
=
\ts{
  \sum_i
  A_\bo{x}^i\,
  (dx_i)_\bo{x}
}
\end{align}
The value of a cotangent on a given tangent vector can therefore be expressed in terms of their expansion coefficients as
\begin{align}
\label{eq:cotangent-tangent-dot-product-expansion}
\ts{
  \a_\bo{x}(\mc{D}_\bo{x})
=
  \sum_{ij}
  A_\bo{x}^i
  D_\bo{x}^j\,
  (dx_i)_\bo{x}((\pt_{x_j})_\bo{x})
=
  \sum_i
  A_\bo{x}^i
  D_\bo{x}^i
}
\end{align}
where the second equality follows from equation~\ref{eq:coordinate-independence}.
\end{ex}


\begin{prop}
\thmstatement{
The cotangent $\a_\bo{x}$ in $T_\bo{x}^*\mb{R}^n$ is an exact differential at if and only if its expansion coefficients in the standard basis have the form $A_\bo{x}^i=(\pd{f}{x_i})_\bo{x}$ for some smooth function $f$ in $C^\infty(\mb{R}^n)$.
}\vspace{5pt}
\thmproof{
  Writing the expansion coefficients as
  $
    A_\bo{x}^i
  =
    (\pt_{x_i})_\bo{x}(f)
  $,
  and expanding $\a_\bo{x}(\mc{D}_\bo{x})$ according to equation~\ref{eq:cotangent-tangent-dot-product-expansion} gives
\begin{align}
\ts{
  \a_\bo{x}(\mc{D}_\bo{x})
=
  \sum_i
  (\pt_{x_i})_\bo{x}(f)\,
  D_\bo{x}^i
=
  \mc{D}_\bo{x}(f)
}
\end{align}
  where we have used the tangent expansion from equation~\ref{eq:tangent-cotangent-expansions} in the second step.
  According to the definition of the differential map this equals
  $
  (df)_\bo{x}(\mc{D}_\bo{x})
  $,
  which shows that
  $
    \a_\bo{x}
  =
    (df)_\bo{x}
  $
  is an exact differential.
}
\end{prop}


\begin{samepage}
\begin{dfn}
\thmtitle{Exterior power}
The \textit{$k\eth$ exterior power} of a vector space $V$ is $\La^k(V)\equiv\{\sum v_1\wedge\cd\wedge v_k\,|\,v_i\in V\}$, where the \textit{wedge product} $\wedge$ is an associative bilinear product that is antisymmetric in its arguments.
\begin{align}
  (v\wedge w)\wedge x
=
  v\wedge (w\wedge x)
&&
  c\,(v\wedge w)
=
  (cv)\wedge w
&&
  (cv + c'v')\wedge w
=
  c\,v\wedge w
+
  c'\,v'\wedge w
&&
  v\wedge w
=
-
  w\wedge v
\end{align}
More generally, if $\a$ is in $\La^k(V)$ and $\b$ is in $\La^{k'}(V)$, we have
$
  \a
  \wedge
  \b
=
  (-)^{k\cdot k'}
  \b
  \wedge
  \a
$.
The $0\eth$ exterior power of $V$ is its underlying scalar field,
$
  \La^0(V)
\equiv
  \mb{R}
$,
whose wedge product is equivalent to scalar multiplication,
$
  c\wedge\a
=
  \a\wedge c
=
  c\,\a
$.
\end{dfn}
\end{samepage}

\begin{dfn}
\thmtitle{Differential $k$-form}
A \textit{differential $k$-form at a point $p$} is an element of the $k\eth$ exterior power of its cotangent space, $\La^k(T_p^*M)$.
The dimension of this space is
$
  {n\choose k}
=
  \fr{n!}{k!(n-k)!}
$,
where $n$ is the dimension of its underlying manifold.
A $k$-form can be expanded in terms of the unique wedge products of the cotangent basis
$\{(\e_1)_p, \ld, (\e_n)_p\}$
as
\begin{align}
\ts{
  \a_p
=
  \sum_{i_1<\cd<i_k}\,
  \a_p^{i_1\cd i_k}
  \,
  (\e_{i_1})_p
  \wedge
  \cd
  \wedge
  (\e_{i_k})_p
}
\end{align}
where $\a_p^{i_1\cd i_k}$ is a real-valued array whose indices range from $1$ to $n$ and the wedge products of $(\e_i)_p$'s constitute a basis for the space of $k$-forms.
Keep in mind that the cotangent vectors are functionals on the tangent space, so each $k$-form eats $k$ tangents $\mc{D}_{1,p},\ld,\mc{D}_{k,p}$ in $T_pM$ and spits out a real number,
$
  \a_p(
    \mc{D}_{1,p}
  ,
    \ld
  ,
    \mc{D}_{k,p}
  )
\in
  \mb{R}
$.
\end{dfn}

\begin{rmk}
It's worth highlighting some special types of $k$-forms.
First of all, note that the 1-forms are simply cotangents.
The 0-forms comprise the one-dimensional space of scalars $\La^0(T_p^* M)=\mb{R}$.
Treating $p$ as a variable, we can think of 0-forms as point values of smooth functions on the manifold: $f_p=f(p)$ for some $f:M\rightarrow\mb{R}$.
Finally, note that on an $n$-dimensional manifold the $n$-forms live in a one-dimensional space as well.
These \textit{top-dimensional forms} are scalar multiples of a single basis vector $(\e_1)_p\wedge\cd\wedge(\e_n)_p$.
\end{rmk}

\begin{ex}
As an example of a $k$-form, consider $\a_\bo{x}$ in $T_\bo{x}^*\mb{R}^n$.
Expanding this quantity in terms of coordinate differentials, we can identify its coordinate array as values of $\a_\bo{x}$ on the coordinate differentials
\begin{align}
  \a_\bo{x}
=
  \sum_{i_1<\cd<i_k}
  \a_\bo{x}^{i_1\cd i_k}\,
  (dx_{i_1})_\bo{x}
  \wedge
  \cd
  \wedge
  (dx_{i_k})_\bo{x}
&&
\implies
&&
  \a_\bo{x}
  (
    (\pt_{x_{i_1}})_\bo{x}
  ,
    \ld
  ,
    (\pt_{x_{i_k}})_\bo{x}
  )
=
  \a_\bo{x}^{i_1\cd i_k}
\end{align}
which follows from equation~\ref{eq:coordinate-independence}.
This can be used to express the value of $\a_\bo{x}$ on a general set of tangents as
\begin{align}
  \a_\bo{x}
  (
    \mc{D}_{1, \bo{x}}
  ,
    \ld
  ,
    \mc{D}_{k, \bo{x}}
  )
=
  \sum_{i_1<\cd<i_k}
  \a_\bo{x}^{i_1\cd i_k}\,
  D_{1, \bo{x}}^{i_1}
  \cd
  D_{k, \bo{x}}^{i_k}
\end{align}
in terms of expansion coefficients with respect to the coordinate derivatives.
\end{ex}

\begin{dfn}
\thmtitle{Exterior derivative}
The \textit{exterior derivative} is a map $\mr{d}:\La^k(T_p^*M)\rightarrow\La^{k+1}(T_p^*M)$ characterized by
\begin{align}
  \mr{d}f_p
=
  (df)_p
&&
  \mr{d}^2\a_p
=
  0
&&
  \mr{d}(\a_p\wedge\b_p)
=
  \mr{d}\a_p\wedge \b_p
+
  (-)^k
  \a_p\wedge \mr{d}\b_p
\end{align}
where $\a_p$ is a $k$-form.
The first property says that the exterior derivative takes 0-forms into their differentials.
A $k$-form that can be expressed as an exterior derivative of a $(k-1)$-form is called an \textit{exact form}, so the second property says that exact forms have vanishing exterior derivatives.
The third property is called the \textit{product rule of antiderivations}.
\end{dfn}

\begin{dfn}
\thmtitle{Partitions of unity}
A set of smooth functions $\y\in\Up\subset C^\infty(M)$ qualifies as a \textit{partition of unity} if
\begin{align}
  0
\leq
  \y(p)
\leq
  1
&&
  \sum_{\y\in\Up}
  \y(p)
=
  1
&&
\exists\,
  O
\ni
  p
:
  |\{
    \y\in\Up
  \,|\,
    \y(p')
  \neq
    0
  ,\,
    p'
  \in
    O
  \}|
=
  n
\in
  \mb{N}
\end{align}
for every point $p$ on the manifold.
In words, the third condition says that there is an open neighborhood of $p$ on which all but a finite number of functions in the partition are zero.
\end{dfn}

\begin{dfn}
\thmtitle{Pushforward}
The \textit{pushforward} of a map $\varphi:M\rightarrow M'$ is a mapping
$
  \varphi_*
:
  T_pM
\rightarrow
  T_{\varphi(p)}M'
$
between their tangent spaces
which takes the tangent $\mc{D}_p$ at the point $p$ on the original manifold into a tangent at $\varphi(p)$ that acts on smooth functions $f'$ on the target manifold as
$
  (\varphi_*\mc{D}_p)_{\varphi(p)}(f')
\equiv
  \mc{D}_p(f'\circ\varphi)
$
where
$\circ$
denotes function composition.
\end{dfn}

\begin{ex}

\end{ex}

\begin{thm}
\thmtitle{Stokes' theorem}
\thmstatement{
Let $M$ be a be a differentiable $n$-manifold with boundary $\pt M$, and let $\a$ be an exact $n$-form on the manifold, $\a=\mr{d}\w$.
Then the integral of $\a$ equals the integral of $\w$ on the manifold boundary. 
\begin{align*}
  \int_{M}
  \mr{d}\w
=
  \oint_{\pt M}
  \w
\end{align*}
}
\end{thm}

\begin{ex}
\thmtitle{Green's theorem}
\begin{align}
  \w
=
  \w_1
  \mr{d}x_1
+
  \w_2
  \mr{d}x_2
&&
  \mr{d}\w
=
  \mr{d}\w_1\wedge\mr{d}x_1
+
  \mr{d}\w_2\wedge\mr{d}x_2
&&
  \mr{d}\w_i
=
  \pd{\w_i}{x_1}
  \mr{d}x_1
+
  \pd{\w_i}{x_2}
  \mr{d}x_2
&&
  \mr{d}\w
=
  \pr{
    \pd{\w_2}{x_1}
  -
    \pd{\w_1}{x_2}
  }
  \mr{d}x_1\wedge\mr{d}x_2
\end{align}
\begin{align}
  \int_M
  \mr{d}x_1\wedge\mr{d}x_2
  \pr{
    \pd{\w_2}{x_1}
  -
    \pd{\w_1}{x_2}
  }
=
  \oint_{\pt M}
  \pr{
    \mr{d}x_1
    \w_1
  +
    \mr{d}x_2
    \w_2
  }
\end{align}
\end{ex}

\begin{ex}
\thmtitle{Kelvin-Stokes theorem}
\begin{align}
  \w
=
  \w_1
  \mr{d}x_1
+
  \w_2
  \mr{d}x_2
+
  \w_3
  \mr{d}x_3
\end{align}
\begin{align}
  \mr{d}\w
=&\
  \mr{d}\w_1
  \wedge
  \mr{d}x_1
+
  \mr{d}\w_2
  \wedge
  \mr{d}x_2
+
  \mr{d}\w_3
  \wedge
  \mr{d}x_3
\\=&\
  \pd{\w_1}{x_2}
  \mr{d}x_2
  \wedge
  \mr{d}x_1
+
  \pd{\w_1}{x_3}
  \mr{d}x_3
  \wedge
  \mr{d}x_1
+
  \pd{\w_2}{x_1}
  \mr{d}x_1
  \wedge
  \mr{d}x_2
+
  \pd{\w_2}{x_3}
  \mr{d}x_3
  \wedge
  \mr{d}x_2
+
  \pd{\w_3}{x_1}
  \mr{d}x_1
  \wedge
  \mr{d}x_3
+
  \pd{\w_3}{x_2}
  \mr{d}x_2
  \wedge
  \mr{d}x_3
\\=&\
  \mr{d}x_2
  \wedge
  \mr{d}x_3
  \pr{
    \pd{\w_3}{x_2}
  -
    \pd{\w_2}{x_3}
  }
+
  \mr{d}x_1
  \wedge
  \mr{d}x_2
  \pr{
    \pd{\w_2}{x_1}
  -
    \pd{\w_1}{x_2}
  }
+
  \mr{d}x_1
  \wedge
  \mr{d}x_3
  \pr{
    \pd{\w_3}{x_1}
  -
    \pd{\w_1}{x_3}
  }
\end{align}
\end{ex}


\newpage

\begin{ex}
A vector field $\bo{v}: M\rightarrow\mb{R}^n$ with elements  $\bo{e}_i\cdot\bo{v}(p)=v_i(p)$ is described by an $(n-1)$-form
\begin{align}
  \w_{\bo{v}(p)}
=
  \sum_{i=1}^n
  (-)^{i-1}
  v_i\,
  dx_1
  \wedge
  \cd
  \wedge
  \cancel{
    dx_i
    \wedge
  }
  \cd
  \wedge
  dx_n
\end{align}
The exterior derivative of this form is the divergence
\begin{align}
  \mr{d}
  \w_{\bo{v}(p)}
=
  \sum_{i=1}^n
  \mr{d}x_1
  \wedge
  \cd
  \wedge
  \mr{d}v_i
  \wedge
  \cd
  \wedge
  \mr{d}x_n
=
  \pr{
    \sum_{i=1}^n
    \pd{v_i}{x_i}
  }
  \mr{d}x_1
  \wedge
  \cd
  \wedge
  \mr{d}x_n
=
  (\nabla\cdot\bo{v})\,\,
  \mr{d}^n\bo{x}
\end{align}
This vector field is also associated with a $1$-form
\begin{align}
  \h_{\bo{v}(p)}
=
  \sum_{i=1}^n
  v_i\,
  \mr{d}x_i
\end{align}
whose exterior derivative as follows.
\begin{align}
  \mr{d}\h_{\bo{v}(p)}
=
  \sum_j
  \mr{d}v_j\wedge\mr{d}x_j
=
  \sum_{i\neq j}
  \pd{v_j}{x_i}\,
  \mr{d}x_i
  \wedge
  \mr{d}x_j
=
  \sum_{i<j}
  \pr{
    \pd{v_j}{x_i}
  -
    \pd{v_i}{x_j}
  }
  \mr{d}x_i
  \wedge
  \mr{d}x_j
\end{align}
For $n=3$, this is equal to the curl.
\begin{align*}
  \mr{d}\h_{\bo{v}(p)}
=&\
  \pr{
    \pd{v_3}{x_2}
  -
    \pd{v_2}{x_3}
  }\,
  \mr{d}x_2
  \wedge
  \mr{d}x_3
+
  \pr{
    \pd{v_3}{x_1}
  -
    \pd{v_1}{x_3}
  }\,
  \mr{d}x_1
  \wedge
  \mr{d}x_3
+
  \pr{
    \pd{v_2}{x_1}
  -
    \pd{v_1}{x_3}
  }\,
  \mr{d}x_1
  \wedge
  \mr{d}x_2
\\=&\
  (-)^{1-1}
  (\nabla\times\bo{v})_1\,
  \cancel{
    \mr{d}x_1
  }
  \wedge
  \mr{d}x_2
  \wedge
  \mr{d}x_3
+
  (-)^{2-1}
  (\nabla\times\bo{v})_2\,
  \mr{d}x_1
  \wedge
  \cancel{
    \mr{d}x_2
  }
  \wedge
  \mr{d}x_3
+
  (-)^{3-1}
  (\nabla\times\bo{v})_3\,
  \mr{d}x_1
  \wedge
  \mr{d}x_2
  \wedge
  \cancel{
    \mr{d}x_3
  }
\\=&\
  \w_{\nabla\times\bo{v}(p)}
\end{align*}
\end{ex}

\end{document}
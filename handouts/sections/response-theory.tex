\chapter{Response theory}

\begin{rmk}
In the presence of a time-varying field, a molecule's electronic wavefunction is no longer simply an eigenfunction of the Hamiltonian.
Instead, its electronic structure is described by the \textit{time-dependent Schr\"odinger equation}
\begin{align}
\label{eq:schrodinger-equation}
  H(t)
  \Y(t)
=
  i
  \pd{\Y(t)}{t}
&&
  H(t)
=
  H
+
  V(t)
\end{align}
where $H$ is the usual electronic Hamiltonian and $V(t)$ is an \textit{interaction Hamiltonian} describing the energetic influence of the field.
A general series solution to equation~\ref{eq:schrodinger-equation}, known as the \textit{Dyson series}, is derived in \cref{appendix:dyson-series}.
The interaction Hamiltonian can be expressed as a sum over one-electron operators $V_\beta$, representing the electronic degrees of freedom which couple to the external field, scaled by \emph{time-envelopes} $f_\beta(t)$ which control the strength of the applied field over time.
\begin{align}
\label{eq:interaction-time-envelopes}
  V(t)
=
\ts{
  \sum_\beta
  V_\beta
  f_\beta(t)
}
\end{align}
One of the most important examples is the Hamiltonian of a dipole in an electric field, which is discussed in \cref{ex:dipole-approximation} below.
The zeroth order solutions of equation~\ref{eq:schrodinger-equation} are termed \textit{stationary states}, which have the following form.\footnotemark
\footnotetext{
When $\bm{f}=\bo{0}$, the Hamiltonian loses its time-dependence and we can write $\left.\Y(t)\right|_{\bm{f}=\bo{0}}=\f(t)\Y$ where $\f(t)$ is independent of the electronic coordinates.
Substituting this into eq~\ref{eq:schrodinger-equation} and rearranging gives
$
  H\Y/\Y
=
  i\dot{\f}(t)/\f(t)
$, which equals a constant $E$ since each side depends in different variables.
Therefore, $H\Y=E\Y$ and $i\dot{\f}(t)=E\f(t)$.  Integrating the latter gives $\f(t)=e^{-iEt}$.
}
\begin{align}
  \left.
  \Y(t)
  \right|_{\bm{f}=\bo{0}}
=
  e^{-iE_kt}
  \Y_k
&&
  H\Y_k
=
  E_k\Y_k
\end{align}
As a boundary condition we assume that $V(t)$ vanishes in the past, where $\Y(t)$ is initially in the ground stationary state.
\begin{align}
\label{eq:boundary-conditions}
  \lim_{t\rightarrow-\infty}
  f_\beta(t)
=
  0
&&
  \lim_{t\rightarrow-\infty}
  e^{+iHt}
  \Psi(t)
=
  \Psi_0
\end{align}
This limiting behavior can be enforced by introducing a complex shift in the frequency domain of $f_\beta(t)$'s Fourier expansion.\footnotemark
\footnotetext{
  This is a slightly unusual convention for the Fourier transform.
  A useful mnemonic for checking these is
  $
    \int_{-\infty}^\infty
    dk\,
    e^{ikx}
  =
    2\pi\,
    \d(x)
  $.
}
\begin{align}
\label{eq:frequency-envelopes}
  f_\beta(t)
=
  \int_{-\infty}^\infty
  d\w\,
  f_\beta(\w_\ev)
  e^{-i\w_\ev t}
&&
  f_\beta(\w_\ev)
\equiv
  (2\pi)^{-1}
  \int_{-\infty}^\infty
  dt\,
  f_\beta(t)
  e^{+i\w_\ev t}
&&
  \w_\ev
\equiv
  \w
+
  i\ev
&&
  \ev
=
  |\ev|
\end{align}
This has the effect of scaling the time envelope by a damping factor $e^{\ev t}$.
For sufficiently small $\ev$, this scaled envelope will match the original one to arbitrary precision in an arbitrarily wide window about the time origin.
The fact that the interaction Hamiltonian and the coupling operators $\{V_\beta\}$ are Hermitian implies the following identities.
\begin{align}
  f_\beta^*(t)
=
  f_\beta(t)
&&
  f_\beta^*(\w_{\ev})
=
  f_\beta(-\w_{-\ev})
\end{align}
\end{rmk}


\begin{ex}
\label{ex:dipole-approximation}
The dominant coupling of an electronic system to an external electric or magnetic field is mediated through its dipoles, leading to \textit{the dipole approximation}.
Quantizing the classical formulae for these interaction energies gives
\begin{align}
\begin{array}{r@{\ }l@{\ }l@{\hspace{2cm}}r@{\ }l@{\hspace{2cm}}r@{\ }l}
  V_{\bo{E}}(t)
\approx&
-\,
  \bm{\mu}\cdot\bo{E}(t)
&=
-
  \sum_\beta
  \mu_\beta
  \mc{E}_\beta(t)
&
  \bm{\mu}
&=
  \sum_{pq}
  \ip{\y_p|\op{\bm{\mu}}\,|\y_q}
  a_p\dg a_q
&
  \op{\bm{\mu}}
&=
-
  \op{\bo{r}}
\\
  V_{\bo{B}}(t)
\approx&
-
  \bm{m}\cdot\bo{B}(t)
&=
-
  \sum_\beta
  m_\beta
  \mc{B}_\beta(t)
&
  \bm{m}
&=
  \sum_{pq}
  \ip{\y_p|\op{\bm{m}}|\y_q}
  a_p\dg a_q
&
  \op{\bm{m}}
&=
-
  \tfr{1}{2}
  (
    \op{\bm{l}}
  +
    2\,
    \op{\bo{s}}
  )
\end{array}
\end{align}
where $\op{\bm{\mu}}$ and $\op{\bm{m}}$ are the first-quantized electric and magnetic dipole operators.\footnote{
More generally, these expressions are
$
  \op{\bm{\mu}}
=
  q_e\,\op{\bo{r}}
$,
where $q_e=-e$ is the charge of an electron,
and
$
  \op{\bm{m}}
=
  \mu_{\text{B}}
  (
    g_l\,
    \op{\bm{l}}
  +
    g_{\text{s}}\,
    \op{\bo{s}}
  )
$
where $\mu_{\text{B}}=\tfr{1}{2}\cdot\tfr{e\hbar}{m_e}$ is the Bohr magneton and $g_l=-1$, $g_{\text{s}}=-2$ are the spin and orbital \textit{$g$-factors}.
Note that the exact $g_{\text{s}}$ actually deviates very slightly from $2$ due to effects arising in quantum field theory.
The orbital angular momentum operator is given by $\op{\bm{l}}=\op{\bo{r}}\times\op{\bo{p}}$ and $\op{\bo{s}}$ is the intrinsic spin angular momentum operator.
}
The leading terms neglected by the dipole approximation are quadratic in the field amplitudes.
These weaker interactions are mediated through the higher moments (quadrupole, octupole, etc.) of the charge and current distributions and may become important in symmetric molecules where certain dipole interactions are ``symmetry forbidden''.
\end{ex}



% % ERROR

% \begin{dfn}
% \thmtitle{Quasi-energy}
% \begin{align}
%   \Psi(t)
% =
%   e^{-i\th(t)}
%   \bar{\Psi}(t)
% &&
%   \left.
%   \th(t)
%   \right|_{\bm{f}=\bo{0}}
% =
%   E_0
%   t
% &&
%   \lim_{t\rightarrow-\infty}
%   \bar{\Psi}(t)
% =
%   \Psi_0
% \end{align}
% \begin{align}
%   (
%     H(t)
%   -
%     i
%     \tpd{}{t}
%   )
%   \bar{\Psi}(t)
% =
%   \dot{\th}(t)
%   \bar{\Psi}(t)
% \end{align}
% \begin{align}
%   \dot{\th}(t)
% =
%   \int_0^t
%   dt'
%   \ip{\bar{\Psi}(t')|
%     H(t')
%   -
%     i\tpd{}{t'}
%   |\bar{\Psi}(t')}
% \end{align}
% \begin{align}
%   \br{\d\bar{\Psi}(t)}
%     H(t)
%   -
%     i
%     \tpd{}{t}
%   \kt{\bar{\Psi}(t)}
% =
%   \dot{\th}(t)
%   \ip{\d\bar{\Psi}(t)|\bar{\Psi}(t)}
% \end{align}
% \begin{align}
%   \ip{\d\bar{\Psi}(t)|\bar{\Psi}(t)}
% +
%   \ip{\bar{\Psi}(t)|\d\bar{\Psi}(t)}
% =
%   0
% \end{align}
% \begin{align}
%   \d
%   \ip{\bar{\Psi}(t)|
%     H(t)
%   -
%     i
%     \tpd{}{t}
%   |\bar{\Psi}(t)}
% +
%   i
%   \tpd{}{t}
%   \ip{\bar{\Psi}(t)|\d\bar{\Psi}(t)}
% =
%   0
% \end{align}
% \end{dfn}

\chapter{Proof of the Linked-Diagram Theorem}\label{app:linked-diagram-theorem}

\begin{ntt}\label{ntt:operator-combinations}
Let
``$Y^m$ choose $Z^k$'', denoted ${}^mC_k(Y:Z)$,
refer to a sum over the $m$ choose $k$ permutations of $Y^{m-k}Z^k$,\,\footnote{For example,
$
  {}^4C_2(Y:Z)
=
  Y^2Z^2
+
  YZYZ
+
  YZ^2Y
+
  ZY^2Z
+
  ZYZY
+
  Z^2Y^2
$.
}
where $Y$ and $Z$ are operators that may or may not commute.\,\footnote{
  If they do commute, then ${}^mC_k(Y:Z)={n\choose k}Y^{m-k}Z^k$.
}
This defines a generalization of the binomial theorem.
\begin{align}
\label{eq:generalized-binomial-theorem}
  (
    Y
  +
    Z
  )^m
=
  \sum_{k=0}^m
  {}^mC_k(Y:Z)
\end{align}
Furthermore, let
$
  {}^mC(Y:Z_1,\ld,Z_k)
$
be a sum over permutations of
$
  Y^{m-k}
  Z_1\cd Z_k
$ that preserve the ordering of the $Z_i$'s.\,\footnote{
  For example,
$
  {}^4C(Y:Z_1,Z_2)
=
  Y^2Z_1Z_2
+
  YZ_1YZ_2
+
  YZ_1Z_2Y
+
  Z_1Y^2Z_2
+
  Z_1YZ_2Y
+
  Z_1Z_2Y^2
$.
}
When all of the $Z_i$'s equal $Z$, we can write
$
  {}^mC(Y:Z_1,\ld,Z_k)
=
  {}^mC_k(Y:Z)
$.
\end{ntt}

\begin{prop}
\label{prop:wavefunction-infinite-recursion}
\thmstatement{
$\ds{
  \Y(\la)
=
  \sum_{m=0}^\infty
  \pr{
    R_0
    (\la V_{\text{c}} - E(\la))
  }^m
  \F
}$
}
\thmproof{
  This follows by infinite recursion of equation~\ref{eq:lambda-dependent-recursive-series} with the assumption
  $\ds{
  \lim_{m\rightarrow\infty}
  \pr{
    R_0
    (\la V_{\text{c}} - E(\la))
  }^m
  \Y(\la)
  =
    0
  }$.
}
\end{prop}

\begin{dfn}\label{dfn:integer-compositions}
\thmtitle{Integer compositions}
The \textit{compositions} of an integer $m$ are the ways of writing $m$ as a sum of positive integers.
The full set of integer compositions of $m$ is given by
$
  \mc{C}(m)
=
  \mc{C}_1(m)
  \cup
  \mc{C}_2(m)
  \cup
  \cd
  \cup
  \mc{C}_m(m)
$
where
$
  \mc{C}_k(m)
=
  \{
    (r_1,\ld,r_k)\in\mb{N}_0^k
  \,|\,
    r_1+\cd+r_k
  =
    m
  \}
$
are the integer compositions of $m$ into $k$ parts.
\end{dfn}

\begin{lem}
\label{lem:energy-substitution-proof}
\thmtitle{The Energy Substitution Lemma}
\thmstatement{
  $\Y\ord{m}$ equals the sum of a ``principal term''
  $(R_0V_{\text{c}})^m\F$
  plus all possible substitutions of adjacent factors $(R_0V_{\text{c}})^{r_i}$ in the principal term by $R_0E_{\text{c}}\ord{r_i}$.
  Each term in the sum is weighted by a sign factor $(-)^k$, where $k$ is the number of substitutions.
}\vspace{5pt}
\thmproof{
Using equation~\ref{eq:generalized-binomial-theorem} and a double sum identity\footnote{
  Reverse double-sum reduction:
  $\ds{
    \sum_{m=0}^\infty
    \sum_{k=0}^m
    t_{m-k,k}
  =
    \sum_{k'=0}^\infty
    \sum_{k=0}^\infty
    t_{k',k}
  }$.
  See
  \url{http://functions.wolfram.com/GeneralIdentities/12/}.
} in the infinite recursion formula for $\Y(\la)$ gives the following.
{\footnotesize
\begin{align*}
  \Y(\la)
=
  \sum_{m=0}^\infty
  (
    R_0
    (
      \la V_{\text{c}}
    -
      E(\la)
    )
  )^m
  \F
=
  \sum_{m=0}^\infty
  \sum_{k=0}^m
  \la^{m-k}
  (-)^k\,\,
  {}^mC_k
  (
    R_0V_{\text{c}}:
    R_0E(\la)
  )
  \F
=
  \sum_{k'=0}^\infty
  \sum_{k=0}^\infty
  \la^{k'}
  (-)^{k}\,\,
  {}^{k'+k}\hspace{-1pt}C_k
  (
    R_0V_{\text{c}}:
    R_0E(\la)
  )
  \F
\end{align*}}%
The $k'=0$ term has no operators separating $\F$ from the resolvent and vanishes.
Taylor expansion of the energies gives
{\footnotesize
\begin{align*}
  \Y(\la)
=&\
  \sum_{k=0}^\infty
  \sum_{k'=1}^\infty
  {\sum_{p_1=1}^\infty}
  \cd
  {\sum_{p_k=1}^\infty}
  \la^{k' + p_1 + \cd + p_k}
  (-)^{k}\,\,
  {}^{k'+k}\hspace{-1pt}C
  (
    R_0V_{\text{c}}:
    R_0E_{\text{c}}\ord{p_1},\ld,
    R_0E_{\text{c}}\ord{p_k}
  )
  \F
\\
=&\
  \sum_{m=1}^\infty
  \sum_{k=0}^{m-1}
  \sum_{(r_1,\ld,r_{k+1})}^{\mc{C}_{k+1}(m)}
  \la^m
  (-)^k\,\,
  {}^{k+r_1}\hspace{-1pt}C
  (
    R_0V_{\text{c}}:
    R_0E_{\text{c}}\ord{r_2},\ld,
    R_0E_{\text{c}}\ord{r_{k+1}}
  )
  \F
\end{align*}}%
where we have grouped powers of $\la$ using a multi-sum reduction.
Writing the inner sums as a sum over $\mc{C}(m)$ we find
\begin{align}
  \Y\ord{m}
=
  \left.
  \fr{1}{m!}
  \pd{^m\Y(\la)}{\la^m}
  \right|_{\la=0}
=
  \sum_{(r_1,\ld,r_{k+1})}^{\mc{C}(m)}
  (-)^k\,\,
  {}^{k+r_1}\hspace{-1pt}C
  (
    R_0V_{\text{c}}:
    R_0E_{\text{c}}\ord{r_2},\ld,
    R_0E_{\text{c}}\ord{r_{k+1}}
  )
  \F
\end{align}
which, given \cref{ntt:operator-combinations} and definition~\ref{dfn:integer-compositions}, is an algebraic statement of the proposition, completing the proof.
}
\end{lem}

\begin{thm}
\thmtitle{The Bracketing Theorem}
\thmstatement{
  $\Y\ord{m}$ equals the principal term plus all possible insertions of nested brackets into the principal term.
  Each term in the sum is weighted by $(-)^k$ where $k$ is the total number of brackets.
}\vspace{5pt}
\thmproof{
  The proposition holds for $m=1$ because
  $\Y\ord{1}=R_0V_{\text{c}}\F$ and there are no possible bracketings.
  Assume it holds for $m-1$.
  Then by the energy substitution lemma it also holds for $m$ because $E_{\text{c}}\ord{r_i}$
  equals
  $\ip{\F|V_{\text{c}}|\Y\ord{r_i}}$
  which, by our inductive assumption, equals
  $\ip{V_{\text{c}}(R_0V_{\text{c}})^{r_i}}$
  plus all nested bracketings weighted by appropriate sign factors.
}
\end{thm}

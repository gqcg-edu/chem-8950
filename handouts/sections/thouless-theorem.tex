\chapter{The Thouless theorem}
\label{appendix:thouless}

\begin{ntt}
\label{ntt:orbital-transformation}
Let $\bm{\y}=[\y_p]$ be a row vector of orthonormal spin-orbitals, composed of occupied and virtual blocks, $\bm{\y}=[\bm{\y}_{\text{o}}\ \bm{\y}_{\text{v}}]$, with respect to a reference determinant, $\F$.
Other spin-orbital bases relate to this one via
$
  \kt{\bm{\y}'}
=
  \kt{\bm{\y}}\,
  \bo{U}
$,
which is a unitary transformation if the primed orbitals are orthonormal.
Let $\F'$ be the \textit{transformed reference determinant}, constructed from the first $n$ orbitals in $\bm{\y}'$.
Then the occupied and virtual orbitals of the transformed space are given by
\begin{align}
\label{eq:block-transformation}
  \kt{\bm{\y}'_{\text{o}}}
=
  \kt{\bm{\y}_{\text{o}}}\,
  \bo{U}_{\text{oo}}
+
  \kt{\bm{\y}_{\text{v}}}\,
  \bo{U}_{\text{vo}}
&&
  \kt{\bm{\y}'_{\text{v}}}
=
  \kt{\bm{\y}_{\text{o}}}\,
  \bo{U}_{\text{ov}}
+
  \kt{\bm{\y}_{\text{v}}}\,
  \bo{U}_{\text{vv}}
\end{align}
in terms of the occupied and virtual blocks of $\bo{U}$.
This kind of transformation is sometimes called an \textit{orbital rotation}.
\end{ntt}


\begin{thm}
\label{thm:thouless}
\thmtitle{The Thouless theorem}
\begin{enumerate}
\item
\label{item:thouless-part-1}
\thmstatement{
  The function
  $e^{T_1}\F$
  is an intermediately normalized determinant
  $
    \tfr{1}{\sqrt{n!}}
    \text{det}(\widetilde{\y}_1\cd\widetilde{\y}_n)
  $
  with orbitals
  $
    \widetilde{\y}_i
  =
    \y_i
  +
    \sum_a
    \y_a
    t_a^i
  $.
}
\thmproof{
  Intermediate normalization follows from
  $
    \ip{\F|e^{T_1}\F}
  =
    1
  $.
  This function has the form of a determinant
\begin{align*}
  e^{T_1}
  \kt{\F}
=
  e^{\sum_{a}t_a^1 a_1^a + \cd + \sum_a t_a^n a_n^a}
  a_1\dg
  \cd
  a_n\dg
  \kt{\vac}
=
  \widetilde{a}_1\dg
  \cd
  \widetilde{a}_n\dg
  \kt{\vac}
=
  \kt{\widetilde{\F}}
&&
  \widetilde{a}_i\dg
\equiv
  \text{exp}(\ts{\sum_{a}t_a^i a_i^a})\,
  a_i\dg
\end{align*}
  since $\sum_a t_a^ia_i^a$ commutes with all creation operators except $a_i\dg$.
  The transformed orbitals are given by
\begin{align*}
  \kt{\widetilde{\y}_i}
=
  \widetilde{a}_i\dg
  \kt{\vac}
=
  \text{exp}(\ts{\sum_{a}t_a^i a_a\dg a_i})\,
  a_i\dg
  \kt{\vac}
=
  (\ts{
    1
  +
    \sum_a
    t_a^i
    a_a\dg
    a_i
  })\,
    a_i\dg
  \kt{\vac}
=
  \kt{\y_i}
+
  \ts{\sum_a}
  t_a^i
  \kt{\y_a}
\end{align*}
 using $a_i^2=0$ and $a_ia_i\dg\kt{\vac}=\kt{\vac}$.
}


\item
\thmstatement{
  Any intermediately normalized determinant
  $
    \widetilde{\F}
  =
    \tfr{1}{\sqrt{n!}}
    \text{det}(\widetilde{\y}_1\cd\widetilde{\y}_n)
  $
  can be written as $e^{T_1}\,\F$.
}
\thmproof{
  Intermediate normalization implies that $\widetilde{\F}$ has non-zero overlap with the reference determinant.
  Therefore, $\widetilde{\F}$ can be written as
  $\F'/\ip{\F|\F'}$
  where $\F'$ is a Slater determinant.
  The normalization factor is given by
\begin{align*}
\ts{
  \ip{\F|\F'}
=
  \tfr{1}{n!}
  \sum_{\pi,\si}^{\text{S}_n}
  \e_\pi
  \e_\si
  \ip{\y_{\pi(1)}|\y'_{\si(1)}}
  \cd
  \ip{\y_{\pi(n)}|\y'_{\si(n)}}
=
  \sum_{\si}^{\text{S}_n}
  \e_\si
  \ip{\y_{1}|\y'_{\si(1)}}
  \cd
  \ip{\y_{n}|\y'_{\si(n)}}
=
  \text{det}(\bo{U}_{\text{oo}})\,\,.
}
\end{align*}
  Therefore,
  $
    \widetilde{\F}
  =
    \F'/
    \text{det}(\bo{U}_{\text{oo}})
  =
    \F'\,
    \text{det}(\bo{U}_{\text{oo}}^{-1})
  $
  and the rows of $\widetilde{\F}$ are given by the following vector\footnote{
    The second equality follows from expanding $\kt{\bm{\y}_{\text{o}}'}$ according to eq~\ref{eq:block-transformation}.
  }
\begin{align*}
  \kt{\bm{\widetilde{\y}}_{\text{o}}}
=
  \kt{\bm{\y}'_{\text{o}}}\,
  \bo{U}_{\text{oo}}^{-1}
=
  \kt{\bm{\y}_{\text{o}}}\,
+
  \kt{\bm{\y}_{\text{v}}}\,
  \bo{U}_{\text{vo}}
  \bo{U}_{\text{oo}}^{-1}
\end{align*}
  with elements
  $
    \widetilde{\y}_i
  =
    \y_i
  +
    \sum_a
    \y_a\,
    (\bo{U}_{\text{vo}}\bo{U}_{\text{oo}}^{-1})_{ai}
  $.
  Referring back to part one,
  $
    \widetilde{\F}
  =
    e^{T_1}\F
  $
  with
  $
    t_a^i
  =
    (\bo{U}_{\text{vo}}\bo{U}_{\text{oo}}^{-1})_{ai}
  $.
}
\end{enumerate}
\end{thm}



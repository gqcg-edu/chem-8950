\chapter{Orbital rotations}
\label{appendix:orbital-rotations}

\begin{dfn}
\label{dfn:normal-matrix}
\thmtitle{Normal matrix}
A square matrix satisfying $\bo{N}\dg\bo{N}=\bo{N}\bo{N}\dg$ is termed \textit{normal}.
Several important kinds of matrices meet this criterion:
\textit{Hermitian matrices}, $\bo{H}\dg=\bo{H}$;
\textit{anti-Hermitian matrices}, $\bo{A}\dg=-\bo{A}$;
and
\textit{unitary matrices}, $\bo{U}\dg=\bo{U}^{-1}$.
Note that Hermitian and anti-Hermitian matrices can always be written as $\bo{X}+\bo{X}\dg$ and $\bo{X}-\bo{X}\dg$.
\end{dfn}

\begin{rmk}
\label{rmk:spectral-theorem}
The spectral theorem\footnote{See \url{https://en.wikipedia.org/wiki/Spectral_theorem}} for normal matrices says that $\bo{N}=\bo{V}\widetilde{\bo{N}}\bo{V}\dg$ where $\bo{V}$ is unitary and $\widetilde{\bo{N}}$ is diagonal.
A direct corollary\footnote{Since there exists a basis in which $\bo{N}$ is diagonal, statements about $\bo{N}$ translate into statements about its eigenvalues.} is that the eigenvalues of Hermitian, anti-Hermitian, and unitary matrices can be written as follows.
\begin{align}
  h^*
=
  h
\implies
  h
=
  \f
&&
  a^*
=
-
  a
\implies
  a
=
  i\f
&&
  u^*
=
  u^{-1}
\implies
  u
=
  e^{i\f}
&&
  \f
\in
  \mb{R}
\end{align}
In words, Hermitian eigenvalues are real, anti-Hermitian eigenvalues are pure imaginary, and unitary eigenvalues lie on the unit circle.
Note that unitary eigenvalues have the form $u=\text{exp}(a)$ where $a$ is an anti-Hermitian eigenvalue.
This implies that any unitary matrix $\bo{U}$ can be written as $\text{exp}(\bo{A})$, where $\bo{A}$ is anti-Hermitian.
\end{rmk}

\begin{rmk}
\label{rmk:spin-orbital-transformation-law}
According to \cref{dfn:normal-matrix} and \cref{rmk:spectral-theorem}, unitary transformations of the spin-orbitals can be parametrized as
\begin{align}
\label{eq:spin-orbital-transformation}
  \y_p'
=
  \sum_q
  \y_q
  (\text{exp}(\bo{X} - \bo{X}\dg))_{qp}
\end{align}
in terms of a square matrix $\bo{X}$.
The form of this parametrization leads to redundancies.
In particular, notice that $(\bo{X})_{pq}\equiv z\,\delta_{pp'}\delta_{qq'}$ generates the same transformation as $(\bo{X})_{pq}\equiv -z^*\,\delta_{pq'}\delta_{qp'}$.
These redundancies are eliminated by setting the upper or lower triangle of $\bo{X}$ to zero.
The creation operators for these orbitals are given by
$
  a_p^{\prime\,\dagger}
=
  \sum_q
  a_q\dg
  (\text{exp}(\bo{X} - \bo{X}\dg))_{qp}
$.
\end{rmk}


\begin{prop}
\label{prop:creation-operator-similarity-transform}
\thmstatement{
The identity\ \
$\ds{
  \text{exp}(G)\,a_p\dg\,\text{exp}(-G)
=
  \sum_q
  a_q\dg\,
  (\text{exp}(\bo{G}))_{qp}
}$\
holds for any
$
  G
=
  \sum_{pq}
  (\bo{G})_{pq}\,
  a_p\dg a_q
$.
}\vspace{3pt}
\thmproof{
  This follows from
  $
    [G,\cdot\,]^m(a_p\dg)
  =
    \sum_q
    a_q\dg
    (\bo{G}^m)_{qp}
  $,
  which we will prove by induction.
  For $m=0$ the statement is trivially true.
  If we assume it holds for $m$, then the following shows that it also holds for $m+1$,\footnote{
  The second equality here follows from expanding $G$ and using
$
  [a_r\dg a_s, a_q\dg]
=
  \no{
    a_r\dg
    \ctr{}{a}{_s}{}
    a_s a_q\dg
  }
=
  a_r\dg\,
  \delta_{sq}
$.
}
\begin{align*}
\ts{
  [G,\cdot\,]^{m+1}(a_p\dg)
=
  \sum_q
  [G, a_q\dg]\,
  (\bo{G}^m)_{qp}
=
  \sum_{qr}
  a_r\dg
  (\bo{G})_{rq}
  (\bo{G}^m)_{qp}
=
  \sum_r
  a_r\dg
  (\bo{G}^{m+1})_{rp}
}
\end{align*}
  which completes the induction.
  Substituting this result into the Hausdorff expansion of
$
  \text{exp}(G)\,a_p\dg\,\text{exp}(-G)
$
and  recognizing the Taylor expansion of $\text{exp}(\bo{G})$ completes the proof.
}
\end{prop}


\begin{rmk}
Given \cref{rmk:spin-orbital-transformation-law} and \cref{prop:creation-operator-similarity-transform}, the transformation of particle-hole operators can be expressed as
\begin{align}
\label{eq:operator-transformation}
\begin{array}{r@{\ }l}
  a_p^{\prime\,\dagger}
&=
  \text{exp}(X - X\dg)
  a_p\dg\,
  \text{exp}(X\dg - X)
\\[4pt]
  a_p^{\prime}
&=
  \text{exp}(X - X\dg)
  a_p\,
  \text{exp}(X\dg - X)
\end{array}
&&
  X
=
  \sum_{p>q}
  (\bo{X})_{pq}
  a_p\dg a_q
\end{align}
where the annihilation operator transformation is simply the adjoint of the one for creation operators.
\end{rmk}


\begin{rmk}
If $\Th'$ is obtained by replacing all of the orbitals in the basis expansion of $\Th\in\mc{F}$ with primed orbitals, then
\begin{align}
  \Th'
=
  \text{exp}(X - X\dg)
  \Th
\end{align}
which follows from substituting equation~\ref{eq:operator-transformation} into
$
  a_{p_1}^{\prime\,\dagger}
  \cd
  a_{p_n}^{\prime\,\dagger}
  \kt{\vac}
$
to prove that
$
  \kt{\F'_{(p_1\cd p_n)}}
=
  \text{exp}(X - X\dg)
  \kt{\F_{(p_1\cd p_n)}}
$\footnote{
  Note that
$
  \text{exp}(X\dg - X)
=
  \pr{\text{exp}(X - X\dg)}^{-1}
$
and
$
  \text{exp}(X\dg - X)
  \kt{\vac}
=
  \kt{\vac}
$.
}
for any basis state.
\end{rmk}

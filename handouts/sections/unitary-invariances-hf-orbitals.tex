\chapter{Unitary Invariances for Hartree-Fock Orbitals}\label{app:hartree-fock-orbital-invariance}


\paragraph{Orthonormality.}
By definition, unitary transformations preserve overlaps.
This can be verified as follows
\begin{align*}
  \ip{\tl\y_i|\tl\y_j}
=
\sum_{kl}
  U_{ki}U_{lj}^*
  \ip{\y_k|\y_l}
=
\sum_{kl}
  U_{ki}U_{lj}^*
  \delta_{kl}
=
\sum_k
  U_{ki}U_{kj}^*
=
  \delta_{ij}
\end{align*}
using $\sum_k U_{ki}U_{kj}^*=(\bo{U}\bo{U}\dg)_{ji}=(\bo{1})_{ji}=\d_{ji}$.

\paragraph{Fock operator.}
Only the Coulomb and exchange parts of the Fock operator depend on the orbital set.
For the Coulomb part, we have
{\small\begin{align*}
\sum_i
  \ip{\tl\y_i(2)|\op{g}(1,2)|\tl\y_i(2)}
=
\sum_{ijk}
  U_{ji}U_{ki}^*
  \ip{\y_j(2)|\op{g}(1,2)|\y_k(2)}
=
\sum_{jk}
  \delta_{jk}
  \ip{\y_j(2)|\op{g}(1,2)|\y_k(2)}
=
\sum_j
  \ip{\y_j(2)|\op{g}(1,2)|\y_j(2)}
\end{align*} \underline{}}%
using the fact that $\sum_i U_{ji}U_{ki}^*=\d_{jk}$.
For the exchange part, we have the same thing with a $\op{P}(1,2)$ sandwiched in there.

\paragraph{Hamiltonian expectation value.}
The vector notation $\bm\y$ for our orbitals allows us to express $\F$ and $\tl\F$ as
\begin{align*}
  \F(1,\ld,n)
=
  \tfrac{1}{\sqrt{n!}}
  |\bm\y(1)\cd\bm\y(n)|
%\sp\sp
  \tl\F(1,\ld,n)
=
  \tfrac{1}{\sqrt{n!}}
  |\tl{\bm\y}(1)\cd\tl{\bm\y}(n)|
\end{align*}
which, noting that the matrix $\ma{\tl{\bm\y}(1)\ \cd\ \tl{\bm\y}(n)}$ is simply
\begin{align*}
  \ma{\tl{\bm\y}(1)\ \cd\ \tl{\bm\y}(n)}
=
  \ma{\bo{U}\dg\bm\y(1)\ \cd\ \bo{U}\dg\bm\y(n)}
=
  \bo{U}\dg\ma{\bm\y(1)\ \cd\ \bm\y(n)}
\end{align*}
implies $\tl\F=\det(\bo{U}\dg)\F=\det(\bo{U})^*\F$.
Therefore, $\tl{\F}$ and $\F$ have the same energy expectation values.
\begin{align}
  \ip{\tl\F|\op{H}_e|\tl\F}
=
  \det(\bo{U}\bo{U}\dg)\ip{\F|\op{H}_e|\F}
=
  \ip{\F|\op{H}_e|\F}
\end{align}

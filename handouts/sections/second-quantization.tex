\chapter{Second Quantization}

\begin{dfn}\label{dfn:slater-determinant}
\thmtitle{Slater determinant}
A \textit{Slater determinant} is a normalized antisymmetric product of spin-orbitals
\begin{align}\label{eq:slater-determinant-position-representation}
  \F_{(p_1\cd p_n)}(1,\ld,n)
=
  \tfr{1}{\sqrt{n!}}
  \sum_{\pi}^{\mr{S}_n}
  \e_{\pi}
  \y_{p_{\pi(1)}}(1)\cd\y_{p_{\pi(n)}}(n)
\end{align}
where $\pi\in\mr{S}_n$ is a permutation of $(1,\cd, n)$ with signature $\e_{\pi}$.\footnote{The signature of a permutation is $(-)^{\text{\# transpositions}}$.  See \url{https://en.wikipedia.org/wiki/Symmetric_group} for more on $\mr{S}_n$.}
\end{dfn}

\section{Deriving the second-quantized Hamiltonian from first quantization}\label{ssec:direct-derivation-of-second-quantization}

Let $\mc{F}_n$ denote the span of $n$-electron determinants and consider the integral operator $\op{a}_p:\mc{F}_n\rightarrow \mc{F}_{n-1}$ given by
\begin{align}
  (\op{a}_p\Y)(2,\cd,n)
\equiv
  \sqrt{n}\int d(1) \y_p^*(1)\Y(1,2,\cd,n)\,.
\end{align}
This operator acts on Slater determinants as follows.
\begin{align}
  (\op{a}_p\F_{(p_1\cd p_n)})(2,\cd,n)
=
\left\{
\ar{
  (-)^{k-1}\F_{(p_1\cd \cancel{p_k}\cd p_n)}(2,\ld,n) & p=p_k\in(p_1\cd p_n)\\[5pt]
  0 & \text{otherwise}
}
\right.
\end{align}
In words, it deletes $\y_p$ from $\F_{(p_1\cd p_n)}$ if present, otherwise killing the determinant.
The restriction to an antisymmetric space makes these operators anticommute, $\op{a}_p\op{a}_q=-\op{a}_q\op{a}_p$, since it can be shown that for $\Y\in\mc{F}_n$
\begin{align*}
  \int d(1)d(2)\y_p^*(1)\y_q^*(2)\Y(1,2,\cd,n)
=&\
-
  \int d(1)d(2)\y_q^*(1)\y_p^*(2)\Y(1,2,\cd,n)
\end{align*}
by swapping integration variables.
These operators can be used to generate the following decompositions.\footnote{These follow from substituting in the definition of $\op{a}_p$ and applying resolution of the identity to each argument.}
\begin{align}
  \Y(1,\cd,n)
=&\
  \tfr{1}{\sqrt{n}}
  \sum_p^\infty\y_p(1)\pr{\op{a}_p\Y}(2,\cd,n)
\\=&\
  \tfr{1}{\sqrt{n(n-1)}}
  \sum_{pq}^\infty
  \y_p(1)\y_q(2)(\op{a}_q\op{a}_p\Y)(3,\cd,n)
\end{align}
Therefore, matrix elements of the electronic Hamiltonian with respect to $\Y,\Y'\in \mc{F}_n$ can be expressed as
\begin{align*}
  \ip{\Y|\op{H}\Y'}
=
  \sum_{i=1}^n\ip{\Y|\op{h}(i)\Y'}
+
  \sum_{i<j}^n\ip{\Y|\op{g}(i,j)\Y'}
=&\
  n\ip{\Y|\op{h}(1)\Y'}
+
  \tfr{n(n-1)}{2}
  \ip{\Y|\op{g}(1,2)\Y'}
\\=&\
  \sum_{pq}^\infty
  h_{pq}\ip{\op{a}_p\Y|\op{a}_q\Y'}
+
  \tfr{1}{2}
  \sum_{pqrs}^\infty
  \ip{pq|rs}\ip{\op{a}_q\op{a}_p\Y|\op{a}_s\op{a}_r\Y'}
\end{align*}
in terms of the usual one- and two-electron integrals.
Since $\Y$ and $\Y'$ are arbitrary elements of $\mc{F}_n$, this implies
\begin{align}\label{eq:second-quantized-hamiltonian}
  \left.
  \op{H}
  \right|_{\mc{F}_n}
=
  \sum_{pq}^\infty
  h_{pq}
  \op{a}_p\dg \op{a}_q
+
  \tfr{1}{2}
  \sum_{pqrs}^\infty
  \ip{pq|rs}
  \op{a}_p\dg\op{a}_q\dg\op{a}_s\op{a}_r
\end{align}
which is the \textit{second quantized} form of the Hamiltonian, as opposed to the \textit{first quantized} form which is not restricted to antisymmetric functions.
A defining feature of the second quantization formalism is that $\op{H}$ is independent of the number of electrons, because \cref{eq:second-quantized-hamiltonian} holds for all $n$.

\section{Formal treatment of second quantization}

\begin{dfn}\label{dfn:direct-sum-and-direct-product}
\thmtitle{Direct sums and products}
\begin{samepage}
The \textit{direct sum}, $\oplus$, and \textit{direct product}\footnote{Also known as a \textit{tensor product}}, $\otimes$, are operations defining two different ways of combining vector spaces.
Each operation takes a vector from one space and a vector the other to form an ordered pair, but they behave differently under vector addition and scalar multiplication.
In a direct sum space $V\oplus V'\equiv\{v\oplus v'\,|\,v\in V,\,v'\in V'\}$,
vector addition and scalar multiplication are defined by
\begin{align}
  v_1\oplus v_1'
+
  v_2\oplus v_2'
=
  (v_1 + v_2)
\oplus
  (v_1' + v_2')
&&
  c(v\oplus v')
=
  cv\oplus cv'\,,
\end{align}
whereas, in a direct product space $V\otimes V'\equiv\{\sum v\otimes v'\,|\,v\in V,\,v'\in V'\}$, they are defined as follows.
\begin{align}
  v_1\otimes v'
+
  v_2\otimes v'
=
  (v_1 + v_2)\otimes v'
&&
  v\otimes v_1'
+
  v\otimes v_2'
=
  v\otimes(v_1' + v_2')
&&
  c(v\otimes v')
=
  (cv)\otimes v'
=
  v\otimes(cv')
\end{align}
Note that $\oplus$ behaves like addition and $\otimes$ behaves like multiplication.
If $\{e_i\}$ and $\{e_{i'}'\}$ are basis sets for $V$ and $V'$, respectively, then $\{e_i\oplus0'\}\cup\{0\oplus e_{i'}'\}$ is a basis for their direct sum and $\{e_i\otimes e_{i'}'\}$ is a basis for their direct product.
The dimension of the direct sum space is the sum of their dimensions,
$\dim V+\dim V'$,
and that of the direct product space is the product of their dimensions,
$\dim V\cdot \,\dim V'$.
Finally, if $\ip{\cdot|\cdot}_V$ and $\ip{\cdot|\cdot}_{V'}$ are inner products on $V$ and $V'$, then the following are inner products on the combined spaces.
\begin{align}
  \ip{v\oplus v'|w\oplus w'}_{V\oplus V'}
\equiv
  \ip{v|w}_V
+
  \ip{v'|w'}_{V'}
&&
  \ip{v\otimes v'|w\otimes w'}_{V\otimes V'}
\equiv
  \ip{v|w}_V
\cdot
  \ip{v'|w'}_{V'}
\end{align}
\end{samepage}
\end{dfn}


\begin{dfn}
\thmtitle{Hilbert space}
If $\mc{H}$ is a one-electron Hilbert space spanned by a set of spin-orbitals $\{\y_p\}$, then $\mc{H}^{\otimes n}=\mc{H}\otimes\cd\otimes\mc{H}=\spn\{\y_{p_1}{}\otimes\cd\otimes\y_{p_n}\}$ is an \textit{$n$-electron Hilbert space}.\footnote{These basis vectors are abstract representations spin-orbital product functions, $\ip{1\otimes\cd\otimes n|\y_{p_1}\otimes\cd\otimes\y_{p_n}}=\y_{p_1}(1)\cd\y_{p_n}(n)$, which are known as \textit{Hartree products}.}
\end{dfn}


\begin{dfn}\label{dfn:fock-space}
\thmtitle{Fock space}
Let $\mc{F}_n(\mc{H})$ denote $\spn\{\F_{(p_1\cd p_n)}\}$,\footnote{%
These basis vectors are Slater determinants, abstracted from position space:
$
  \F_{(p_1\cd p_n)}
=
  \fr{1}{\sqrt{n!}}
  \sum_{\pi\in\mr{S}_n}
  \y_{p_{\pi(1)}}\otimes\cd\otimes
  \y_{p_{\pi(n)}}
$.
Equation~\ref{eq:slater-determinant-position-representation} corresponds to
$
  \ip{1\otimes\cd\otimes n|\F_{(p_1\cd p_n)}}= \F_{(p_1\cd p_n)}(1,\ld,n)
$.
}
the antisymmetric subspace of $\mc{H}^{\otimes n}$.
\textit{Fock space} is the union of these spaces, $\mc{F}(\mc{H})=\mc{F}_0(\mc{H})\oplus \mc{F}_1(\mc{H})\oplus \mc{F}_2(\mc{H})\oplus\cd\oplus \mc{F}_{\infty}(\mc{H})$, comprising all possible electronic wavefunctions.
\end{dfn}

\begin{dfn}\label{occupation-number-representation}
\thmtitle{Occupation vectors}
In the \textit{occupation number formalism}, Fock space basis states are represented as \textit{occupation vectors}.
These are denoted by a series of bits,
$
  \kt{\bo{n}}
\equiv
  \kt{n_1,n_2,n_3,\cd,n_\infty}
$,
where $n_p=1$ when $\y_p$ is occupied and $n_p=0$ when it isn't.
The fully unoccupied state is called the \textit{vacuum}, denoted $\kt{\vac}$, which spans $\mc{F}_0(\mc{H})$.
\end{dfn}

\begin{dfn}\label{dfn:particle-hole-operators}
\thmtitle{Particle-hole operators}
\textit{Particle-hole operators} change the occupation numbers of one-particle states.
The \textit{annihilation operator} of $\y_p$ is a linear mapping $a_p:\mc{F}_n(\mc{H})\rightarrow \mc{F}_{n-1}(\mc{H})$ defined by
\begin{align}\label{eq:occ-num-annihilation-operator-action}
  a_p\kt{\cd n_p\cd}
=
  (-)^{n_1+\cd+n_{p-1}}
  \kt{\cd n_p-1\cd}
\ \ \ \text{if $n_p=1$}
&&
  a_p\kt{\cd n_p\cd}
=
  0
\ \ \ \text{if $n_p=0$}
\end{align}
and the \textit{creation operator} of $\y_p$ is a linear mapping $c_p:\mc{F}_n(\mc{H})\rightarrow \mc{F}_{n+1}(\mc{H})$ defined by
\begin{align}\label{eq:occ-num-creation-operator-action}
  c_p\kt{\cd n_p\cd}
=
  (-)^{n_1+\cd+n_{p-1}}
  \kt{\cd n_p+1\cd}
\ \ \ \text{if $n_p=0$\ }
&&
  c_p\kt{\cd n_p\cd}
=
  0
\ \ \ \text{if $n_p=1$.}
\end{align}
\end{dfn}

\begin{prop}
\thmtitle{$c_p=a_p\dg$}
\thmstatement{Creation and annihilation operators of the same state $\y_p$ are adjoints of each other.}
\thmproof{
  $\ip{n_1'n_2'\cd|a_p[n_1n_2\cd]}$ vanishes unless $n_p'=0$, $n_p=1$, and $n_q'=n_q\ \forall q\neq p$.
  Likewise for $\ip{c_p[n_1'n_2'\cd]|n_1n_2\cd}$.
  Therefore, $\ip{\Y|a_p\Y'}=\ip{c_p\Y|\Y'}$ for all $\Y,\Y'\in \mc{F}(\mc{H})$ and $c_p=a_p\dg$ by the definition of adjoint.
}
\end{prop}

\begin{prop}\label{prop:particle-hole-operator-anticommutator}
\thmtitle{$[q,q']_+=\delta_{q'q\dg}$}
\thmstatement{Particle-hole operators $q$ and $q'$ anticommute unless $q'=q\dg$, for which $[q,q\dg]_+=1$.}~\footnote{These are anticommutator brackets, $[q, q']_+ \equiv qq' + q'q$.}
\thmproof{
  Let $q$ and $q'$ be arbitrary particle-hole operators acting on $\y_p$ and $\y_{p'}$, respectively.
  First, suppose $p\neq p'$. Then
  \begin{align*}
  &
    qq'\kt{\cd n_p\cd n_{p'}\cd}
  =
    (-)^{n_p+\sum_{r=p+1}^{p'}n_r}
    \kt{\cd\ol{n_p}\cd\ol{n_{p'}}\cd}
  \,\text{, and}
  \\
  &
    q'q\kt{\cd n_p\cd n_{p'}\cd}
  =
    (-)^{\ol{n_p}+\sum_{r=p+1}^{p'}n_r}
    \kt{\cd\ol{n_p}\cd\ol{n_{p'}}\cd}
  \end{align*}
  where $\ol{n_p}$ and $\ol{n_{p'}}$ are the occupations after applying $q$ and $q'$.
  Since $n_p$ and $\ol{n_p}$ differ by one, $qq'=-q'q$.
  The second case, $p=p'$, implies $q'\in\{q,q\dg\}$.
  If $q'=q$, then $qq'=-q'q=0$.
  If $q'=q\dg$, either $n_p=1\implies(a_p\dg a_p + a_pa_p\dg)\kt{\cd n_p\cd}=(1+0)\kt{\cd n_p\cd}$ or $n_p=0\implies(a_p\dg a_p + a_pa_p\dg)\kt{\cd n_p\cd}=(0+1)\kt{\cd n_p\cd}$.
  Either way, $q'=q\dg\implies(qq' + q'q)=1$.
}
\end{prop}

\begin{rmk}
\thmtitle{Relating the determinant and occupation number formalisms}
When $p_1<\cd<p_n$, $\F_{(p_1\cd p_n)}$ is equivalent to the occupation vector $\kt{\bo{n}_{(p_1\cd p_n)}}$ with ones at $p_1,\cd,p_n$.
Otherwise, this determinant is equivalent to $\e_{\pi}\kt{\bo{n}_{(p_1\cd p_n)}}$ for $\pi\in\mr{S}_n$ such that $p_{\pi(1)}<\cd<p_{\pi(n)}$.
The actions of $a_p$ and $a_p\dg$ on $\F_{(p_1\cd p_n)}$ are given by
\begin{align}\label{eq:abstract-annihilation-operator-action}
  a_p\F_{(p_1\cd p_n)}
=
  (-)^{k-1}\F_{(p_1\cd\cancel{p_k}\cd p_n)}
  \ \text{if $p=p_k\in(p_1\cd p_n)$}
&&
  a_p\F_{(p_1\cd p_n)}
=
  0
  \ \text{if $p\notin(p_1\cd p_n)$}
\\\label{eq:abstract-creation-operator-action}
  a_p\dg\F_{(p_1\cd p_n)}
=
  (-)^{k-1}\F_{(p_1\cd p_{k-1}pp_k\cd p_n)}
  \ \text{if $p\notin(p_1\cd p_n)$}
&&
  a_p\dg\F_{(p_1\cd p_n)}
=
  0
  \ \text{if $p\in(p_1\cd p_n)$}
\end{align}
which follows directly from \Cref{eq:occ-num-annihilation-operator-action,eq:occ-num-creation-operator-action} when $p_1<\cd<p_n$.
Other cases follow from the fact that any sign factors for permuting $(p_1\cd p_n)$ cancel on both sides of the equation, including the position of insertion or deletion, $p_k$, whose phase is tracked by $(-)^{k-1}$ on the right.
That is, both sides of the equation are antisymmetric to permutations of $(1\cd n)$.
Note that \Cref{eq:abstract-annihilation-operator-action} was also derived in \Cref{ssec:direct-derivation-of-second-quantization} using the position-space representation of $a_p$.
One advantage of the determinant basis is that, unlike occupation vectors, determinants translate directly into strings of creations operators
\begin{align}
  \kt{\F_{(p_1\cd p_n)}}
=
  a_{p_1}\dg\cd a_{p_n}\dg\kt{\vac}
\end{align}
without any phase ambiguity.
Together with the second quantized form of the electronic Hamiltonian, this boils much of the grunt work of electronic structure theory down to particle-hole operator algebra.
\end{rmk}

\begin{dfn}
\thmtitle{Excitation operators and excited determinants}
Operator strings of the form $a_{p_1}\dg\cd a_{p_m}\dg a_{q_m}\cd a_{q_1}$ are called \textit{excitation operators}.
For a given reference determinant $\F$, excited determinants can be constructed as
\begin{align}
  \F_{i_1\cd i_m}^{a_1\cd a_m}
=
  a_{a_1}\dg\cd a_{a_m}\dg a_{i_m}\cd a_{i_1}\F
=
  a_{a_1}\dg a_{i_1}\cd a_{a_m}\dg a_{i_m}\F
\end{align}
where $i_1,\cd,i_m$ are occupied and $a_1,\cd,a_m$ are virtual indices with respect to $\F$.
\end{dfn}

\begin{dfn}\label{dfn:particle-hole-isomorphism}
\thmtitle{Particle-hole isomorphism}
The \textit{particle-hole isomorphism} with respect to the reference determinant $\kt{\F}=\kt{\underset{\text{$n$ times}}{1\cd1}\,0\,0\cd}$ is a mapping $F(\mc{H})\rightarrow F(\mc{H})$ that inverts the bits occupied in $\F$:
\begin{align*}
  \kt{k_1\cd k_nk_{n+1}k_{n+2}\cd}
\mapsto
  \kt{\ol{k}_1\cd \ol{k}_nk_{n+1}k_{n+2}\cd}
  \text{where $\ol{k}_i=1-k_i$.}
\end{align*}
Physically, this corresponds to shift in perspective from the \textit{particle frame} to a \textit{quasiparticle frame}, where the first $n$ states are viewed as \textit{holes} rather than \textit{particles}.
This makes $\kt{\vac}\mapsto\kt{\underset{\text{$n$ times}}{\ol{1}\cd\ol{1}}\,0\,0\cd}$ a state of $n$ holes and no particles.  $\kt{\F}\mapsto\kt{\underset{\text{$n$ times}}{\ol{0}\cd\ol{0}}\,0\,0\cd}$ becomes the \textit{quasiparticle vacuum state}, in which all hole and particle states are unoccupied.
\end{dfn}

\begin{dfn}
\thmtitle{Quasiparticle creation and annihilation operators}
If we apply \Cref{dfn:particle-hole-operators} to the new quasiparticle Fock space, we end up with a new system of \textit{quasi-particle-hole operators} $\{b_p\}\cup\{b_p\dg\}$, related to the old set via
\begin{align}
&&
  a_i
\mapsto
  b_i\dg
&&
  a_i\dg
\mapsto
  b_i
&&
  a_a
\mapsto
  b_a
&&
  a_a\dg
\mapsto
  b_a\dg
\end{align}
where $i$ and $a$ are occupied and virtual indices with respect to the reference determinant $\F$.
$\{a_i\dg\}\cup\{a_a\}\mapsto\{b_p\}$ are therefore \textit{quasiparticle annihilation operators} and $\{a_i\}\cup\{a_a\dg\}\mapsto\{b_p\dg\}$ are \textit{quasiparticle creation operators}.
\end{dfn}

\begin{rmk}
The standard expression for the second-quantized Hamiltonian is
\begin{align}
  H
=
  \sum_{pq}
  h_{pq}\,
  a_p\dg a_q
+
  \tfr{1}{4}
  \sum_{pqrs}
  \ip{pq||rs}\,
  a_p\dg a_q\dg a_sa_r
\end{align}
where the summations run over the full set of spin-orbitals.
Note that this matches \Cref{eq:second-quantized-hamiltonian}, except that we have rearranged the second term to express it in terms of antisymmetrized integrals.
In terms of quasi-particle-hole operators,
\begin{align}
\nonumber
  H
=&\
  \sum_{ab}
  h_{ab}\,
  b_a\dg b_b
+
  \sum_{ai}
  h_{ai}\,
  b_a\dg b_i\dg
+
  \sum_{ia}
  h_{ia}\,
  b_i b_a
+
  \sum_{ij}
  h_{ij}
  b_i b_j\dg
\\\nonumber&\
+
  \tfr{1}{4}
  \sum_{abcd}
  \ip{ab||cd}\,
  b_a\dg b_b\dg b_d b_c
+
  \tfr{1}{2}
  \sum_{abci}
  \ip{ab||ci}\,
  b_a\dg b_b\dg b_i\dg b_c
+
  \tfr{1}{2}
  \sum_{aibc}
  \ip{ai||bc}\,
  b_a\dg b_i b_c b_b
+
  \tfr{1}{4}
  \sum_{abij}
  \ip{ab||ij}\,
  b_a\dg b_b\dg b_j\dg b_i\dg
+
  \sum_{aibj}
  \ip{ai||bj}\,
  b_a\dg b_i b_j\dg b_b
\\&\
+
  \tfr{1}{4}
  \sum_{ijab}
  \ip{ij||ab}\,
  b_i b_j b_b b_a
+
  \tfr{1}{2}
  \sum_{iajk}
  \ip{ia||jk}\,
  b_i b_a\dg b_k\dg b_j\dg
+
  \tfr{1}{2}
  \sum_{ijka}
  \ip{ij||ka}\,
  b_i b_j b_a b_k\dg
+
  \tfr{1}{4}
  \sum_{ijkl}
  \ip{ij||kl}\,
  b_i b_j b_l\dg b_k\dg
\end{align}
where we have split the full summations above into summations over occupied and virtual orbitals and grouped like terms.
\end{rmk}

\begin{dfn}
\thmtitle{Normal order}
A string $q_1\cd q_n$ of particle-hole operators is in \textit{normal order} when all of its creation operators sit to the left of its annihilation operators.
That is, when the string has the form $a_{p_1}\dg\cd a_{p_m}\dg a_{r_1}\cd a_{r_{m'}}$.
This guarantees that its vacuum expectation value vanishes, $\ip{\vac|q_1\cd q_n|\vac}=0$.
More generally, we say that $q_1\cd q_n$ is in \textit{$\F$-normal order} if it maps into a string of the form $b_{p_1}\dg\cd b_{p_m}\dg b_{r_1}\cd b_{r_{m'}}$ under particle-hole isomorphism referenced to $\F$, since this guarantees that
$\ip{\F|q_1\cd q_n|\F}=0$.
\end{dfn}

\begin{ex}
In second quantization, any operator string can be expanded as a linear combination of strings which are in normal order.
The expectation value of a string is always equal to the constant term in this expansion.
For example:
\begin{align*}
  a_p a_q\dg
=
-
  a_q\dg a_p
+
  \delta_{pq}
&\implies
  \ip{\vac|a_p a_q\dg|\vac}
=
  \delta_{pq}
\\
  a_p a_q a_s\dg a_r\dg
=
  a_r\dg a_s\dg a_q a_p
+
  \delta_{ps}
  a_r\dg a_q
-
  \delta_{pr}
  a_s\dg a_q
-
  \delta_{qs}
  a_r\dg a_p
+
  \delta_{qr}
  a_s\dg a_p
-
  \delta_{ps}\delta_{qr}
+
  \delta_{pr}\delta_{qs}
&\implies
  \ip{\vac|a_p a_q a_s\dg a_r\dg|\vac}
=
  \delta_{pr}\delta_{qs}
-
  \delta_{ps}\delta_{qr}
\end{align*}
where we have made repeated use of \Cref{prop:particle-hole-operator-anticommutator} to arrive at these expansions.
This strategy becomes unwieldy for expectation values of a reference determinant $\F$, in which case it is more convenient to use particle-hole isomorphism.
For example, consider the following matrix element of the core Hamiltonian.
\begin{align*}
  \sum_{pq}
  h_{pq}
  \ip{\F|a_p\dg a_q|\F_i^a}
=
  \sum_{bc}
  h_{bc}\,
  \cancel{\ip{\F|b_b\dg b_c b_a\dg b_i\dg|\F}}
+
  \sum_{bj}
  h_{bj}\,
  \cancel{\ip{\F|b_b\dg b_j\dg b_a\dg b_i\dg|\F}}
+
  \sum_{jb}
  h_{jb}\,
  \cancelto{\d_{ab}\d_{ij}}{\ip{\F|b_j b_b b_a\dg b_i\dg|\F}}
+
  \sum_{jk}
  h_{jk}
  \cancel{\ip{\F|b_j b_k\dg b_a\dg b_i\dg|\F}}
=
  h_{ia}
\end{align*}
Only the third term survives, because the others generate a ket state with a different number of quasi-particles from the bra state.
\end{ex}

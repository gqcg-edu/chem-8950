\chapter{Fundamental Lemma of Calculus of Variations}\label{app:fundamental-lemma-of-calculus-of-variations}

The \textit{Fundamental Lemma of Calculus of Variations}\footnote{\url{http://en.wikipedia.org/wiki/Fundamental_lemma_of_calculus_of_variations}} says that, for continuous functions, the condition
\begin{align}
  \int_{-\infty}^\infty dx f(x)\eta(x)
=
  0
\text{ for all $\eta(x)$}
\end{align}
holds only when $f(x)=0$ for all $x$.
We can see this by considering the case $\eta(x)=f(x)$.
Since $f(x)^2$ is nonnegative everywhere, the integral yields a positive number whenever $f(x)\neq 0$ on a finite range of $x$ values.

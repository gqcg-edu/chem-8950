\chapter{Traditional coupled-cluster theory}

\begin{dfn}\label{dfn:cc-effective-hamiltonian}
\thmtitle{Traditional coupled-cluster theory}
A \textit{wave operator} maps a determinant into a correlated wavefunction, $\Y=\W\F$.
The \textit{coupled-cluster Ansatz} is characterized by an exponential parametrization of the wave operator.
\begin{align}
\label{eq:cc-effective-hamiltonian}
  H_{\text{c}}\Y_{\text{CC}}
=
  E_{\text{c}}\Y_{\text{CC}}
&&
  \Y_{\text{CC}}
\equiv
  \text{exp}(T)
  \F
&&
  T
\equiv
  T_1
+
  T_2
+
  \cd
+
  T_n
&&
  T_k
\equiv
  (\tfr{1}{k!})^2
  t_{a_1\cd a_k}^{i_1\cd i_k}
  \tl{a}^{a_1\cd a_k}_{i_1\cd i_k}
\end{align}
The coupled-cluster Schr\"odinger equation can be projected onto the determinant basis to arrive at a series of equations
\begin{align}
\label{eq:projected-cc-equations}
  \ip{\F|H_{\text{c}}|\Y_{\text{CC}}}
=
  E_{\text{c}}
&&
  \ip{\F_{ij\cd}^{ab\cd}|H_{\text{c}}|\Y_{\text{CC}}}
=
  E_{\text{c}}
  \ip{\F_{ij\cd}^{ab\cd}|\Y_{\text{CC}}}
\end{align}
which specify the coupled-cluster energy and the \textit{amplitudes}, $t_{ab\cd}^{ij\cd}$.
A different approach, known as \textit{traditional coupled-cluster (TCC) theory}, first multiplies the Schr\"odinger equation on the left by the inverse of the wave operator
\begin{align}
  \ol{H}_{\text{c}}
  \F
=
  E_{\text{c}}
  \F
&&
  \ol{H}_{\text{c}}
\equiv
  \text{exp}(-T)
  H_{\text{c}}\,
  \text{exp}(T)
\end{align}
to define an \textit{effective Hamiltonian}, $\ol{H}_{\text{c}}$.
The eigenvalue of this similarity-transformed\footnote{See \url{https://en.wikipedia.org/wiki/Matrix_similarity}.} Hamiltonian is the exact correlation energy, $E_{\text{c}}$, but its eigenstate is the reference determinant, $\F$, rather than the correlated wavefunction.
Note that, unlike the true Hamiltonian, $\ol{H}_{\text{c}}$ is non-Hermitian.
Projection onto the determinant basis yields energy and amplitude equations
\begin{align}
\label{eq:traditional-cc-equations}
  \ip{\F|\ol{H}_{\text{c}}|\F}
=
  E_{\text{c}}
&&
  \ip{\F_{ij\cd}^{ab\cd}|\ol{H}_{\text{c}}|\F}
=
  0
\end{align}
which look similar to eq~\ref{eq:projected-cc-equations}, except that the right-hand side of the amplitude equations is now zero.
The next few results show that the TCC similarity transformation removes disconnected terms in eq~\ref{eq:projected-cc-equations}, of which $E_{\text{c}}t_{ab\cd}^{ij\cd}$ is an example.
\end{dfn}

\begin{thm}\label{thm:hausdorff}
\thmtitle{The Hausdorff Expansion}
\thmstatement{
$\ds{
  e^{- X}Ye^{X}
=
  \sum_{n=0}^\infty
  \fr{1}{n!}
  [\,\cdot\,, X]^n(Y)
}$
}\,\footnote{
$
  [\,\cdot\,, X]^n(Y)
$
denotes a nested commutator,
$
  [\cd[[Y,X],X]\cd,X]
$.
For $n=0$, we define
$
  [\,\cdot\,, X]^0(Y)
\equiv
  Y
$.
}\vspace{5pt}
\thmproof{
  This follows from
  $
    \pd{^n}{\la^n}
    e^{-\la X}Ye^{\la X}
  =
    [\,\cdot\,, X]^n(
    e^{-\la X}Ye^{\la X}
    )
  $,
  which we will prove by induction.
  This is trivially true for $n=0$.
  Assuming it holds for $n$, the following shows that it also holds for $n+1$, completing the induction.
\begin{align*}
  \tpd{^{n+1}}{\la^{n+1}}
  e^{-\la X}Ye^{\la X}
=
  [\,\cdot\,, X]^n(
  \tpd{}{\la}
  e^{-\la X}Ye^{\la X}
  )
=
  [\,\cdot\,, X]^n(
    e^{-\la X}Ye^{\la X} X
  -
    X e^{-\la X}Ye^{\la X}
  )
=
  [\,\cdot\,, X]^{n+1}(
    e^{-\la X}Ye^{\la X}
  )
\end{align*}
  Substituting this result into a Taylor expansion of
  $
    e^{-\la X}Ye^{\la X}
  $
  about
  $\la=0$
  evaluated at
  $\la=1$
  completes the proof.
}
\end{thm}

\begin{ex}
The Hausdorff expansion can be used to express the TCC effective Hamiltonian in powers of $T$.
\begin{align*}
  \ol{H}_{\text{c}}
=
  \text{exp}(-T)
  H_{\text{c}}\,
  \text{exp}(T)
=
  H_{\text{c}}
+
  [H_{\text{c}}, T]
+
  \tfr{1}{2!}
  [[H_{\text{c}}, T], T]
+
  \tfr{1}{3!}
  [[[H_{\text{c}}, T], T], T]
+
  \tfr{1}{4!}
  [[[[H_{\text{c}}, T], T], T], T]
+
  \cd
\end{align*}
This expansion can be further simplified by analyzing the commutators with $T$ using Wick's theorem.
\end{ex}

\begin{prop}\label{prop:wicks-theorem-for-commutators}
\thmstatement{
If $Q$ and $Q'$ are normal ordered and one of them has an even operator count,
$
  [Q,Q']
=
  \gno{
    \ctr{}{Q}{}{Q}
    QQ'
  }
-
  \gno{
    \ctr{}{Q}{'}{Q}
    Q'Q
  }
$.%
}\vspace{2pt}%
\thmproof{
By Wick's theorem,
$
  QQ'
-
  Q'Q
=
  \gno{QQ'}
+
  \gno{
    \ctr{}{Q}{}{Q}
    QQ'
  }
-
  \gno{Q'Q}
-
  \gno{
    \ctr{}{Q}{'}{Q}
    Q'Q
  }
$.
The proposition follows from the fact that
$
  \gno{QQ'}
=
  \gno{Q'Q}
$
when one of the strings contains an even number of operators.
}
\end{prop}

\begin{cor}
\label{cor:tcc-similarity-transformation-connected}
\thmstatement{
  TCC similarity-transformed operators,
  $
    \ol{W}
  \equiv
    \text{exp}(-T)
    W
    \text{exp}(T)
  $,
  can be evaluated as
  $
    \ol{W}
  =
    (
      W\,
      \text{exp}(T)
    )_{\text{C}}
  $,
  where the subscript $\text{C}$ denotes a restriction to connected diagrams.
}
\thmproof{
  \Cref{prop:wicks-theorem-for-commutators} implies
$
  [W,T]
=
\gno{
  \ctr{}{W}{}{T}
  WT
}
$
and, by straightforward induction,
$
  [\,\cdot\,,T]^n(W)
=
\gno{
  \ctr[1.4]{}{W}{TT\cd}{T}
  \ctr[0.7]{}{W}{T}{T}
  \ctr[0.0]{}{W}{}{T}
  WTT\cd T
}
=
  (WT^n)_{\text{C}}
$,
since
$T$
has no contractions with operators to its right.\footnote{This is easily seen from the diagram.  It comes from the fact that $T$ is composed entirely of quasi-particle creation operators.}
  Applying \cref{thm:hausdorff} to $\ol{W}$ and using this result completes the proof.
}
\end{cor}


\begin{rmk}\label{rmk:connected-expansion}
Applying \cref{cor:tcc-similarity-transformation-connected} to the coupled-cluster effective Hamiltonian gives the following expansion
\begin{align}
  \ol{H}_{\text{c}}
=
  (H_{\text{c}}\,
   \text{exp}(T))_{\text{C}}
=
  (
    H_{\text{c}}
  +
    H_{\text{c}}
    T
  +
    \tfr{1}{2!}
    H_{\text{c}}
    T^2
  +
    \tfr{1}{3!}
    H_{\text{c}}
    T^3
  +
    \tfr{1}{4!}
    H_{\text{c}}
    T^4
  )_{\text{C}}
\end{align}
which ends at the fourth power because $H_{\text{c}}$ is a linear combination of one- and two-particle operators, which can contract at most two and four $T$'s, respectively.
Using this result, the energy and amplitude equations are often written as
\begin{align}
\label{eq:traditional-cc-equations-2}
  \ip{\F|H_{\text{c}}\,\text{exp}(T)|\F}_{\text{C}}
=
  E_{\text{c}}
&&
  \ip{\F_{ij\cd}^{ab\cd}|H_{\text{c}}\,\text{exp}(T)|\F}_{\text{C}}
=
  0
\end{align}
where the subscript $\text{C}$ on the expectation value ket is shorthand for
$
  \ip{\F_{ij\cd}^{ab\cd}|(H_{\text{c}}\,\text{exp}(T))_{\text{C}}|\F}
$.
\end{rmk}


\begin{ntt}
The following is suggested notation for the diagonal and off-diagonal contributions to the Fock operator.
\begin{align}
\diagram{
  \draw (-0.5,0) node[circlex] (h) {} -- (0,0) node[ddot=white] (h1) {};
  \draw[->-] (h1) to ++(0,+0.5);
  \draw[-<-] (h1) to ++(0,-0.5);
}
=
\diagram{
  \draw (-0.5,0) node[circlez] (h) {} -- (0,0) node[ddot=white] (h1) {};
  \draw[->-] (h1) to ++(0,+0.5);
  \draw[-<-] (h1) to ++(0,-0.5);
}
+
\diagram{
  \draw (-0.5,0) node[circlep] (h) {} -- (0,0) node[ddot=white] (h1) {};
  \draw[->-] (h1) to ++(0,+0.5);
  \draw[-<-] (h1) to ++(0,-0.5);
}
&&
\diagram{
  \draw (-0.5,0) node[circlez] (h) {} -- (0,0) node[ddot=white] (h1) {};
  \draw[->-] (h1) to ++(0,+0.5);
  \draw[-<-] (h1) to ++(0,-0.5);
}
\equiv
  H_0
&&
\diagram{
  \draw (-0.5,0) node[circlep] (h) {} -- (0,0) node[ddot=white] (h1) {};
  \draw[->-] (h1) to ++(0,+0.5);
  \draw[-<-] (h1) to ++(0,-0.5);
}
\equiv
  f_p^q(1-\delta_p^q)
  \tl{a}^p_q
\end{align}
so that
$
  H
=
  E_{\text{ref}}
+
\diagram{
  \draw (-0.5,0) node[circlez] (h) {} -- (0,0) node[ddot=white] (h1) {};
  \draw[->-] (h1) to ++(0,+0.35);
  \draw[-<-] (h1) to ++(0,-0.35);
}
+
\diagram{
  \draw (-0.5,0) node[circlep] (h) {} -- (0,0) node[ddot=white] (h1) {};
  \draw[->-] (h1) to ++(0,+0.35);
  \draw[-<-] (h1) to ++(0,-0.35);
}
+
\diagram{
  \interaction{2}{g}{(0,0)}{ddot=white}{sawtooth};
  \draw[->-] (g1) to ++(0,+0.35);
  \draw[-<-] (g1) to ++(0,-0.35);
  \draw[->-] (g2) to ++(0,+0.35);
  \draw[-<-] (g2) to ++(0,-0.35);
}
$\
is the full electronic Hamiltonian
and
$
  V_{\text{c}}
=
\diagram{
  \draw (-0.5,0) node[circlep] (h) {} -- (0,0) node[ddot=white] (h1) {};
  \draw[->-] (h1) to ++(0,+0.35);
  \draw[-<-] (h1) to ++(0,-0.35);
}
+
\diagram{
  \interaction{2}{g}{(0,0)}{ddot=white}{sawtooth};
  \draw[->-] (g1) to ++(0,+0.35);
  \draw[-<-] (g1) to ++(0,-0.35);
  \draw[->-] (g2) to ++(0,+0.35);
  \draw[-<-] (g2) to ++(0,-0.35);
}
$.
Note that
\begin{align*}
\diagram{
  \draw (-0.5,0) node[circlez] (h) {} -- (0,0) node[ddot=white] (h1) {};
  \draw[->-] (h1) to ++(0,+0.5);
  \draw[-<-] (h1) to ++(0,-0.5);
}
=
\diagram{
  \draw (-0.5,0) node[circlez] (h) {} -- (0,0) node[ddot] (h1) {};
  \draw[->-] (h1) to ++(0,+0.5);
  \draw[-<-] (h1) to ++(0,-0.5);
}
+
\diagram{
  \draw (-0.5,0) node[circlez] (h) {} -- (0,0) node[ddot] (h1) {};
  \draw[-<-] (h1) to ++(0,+0.5);
  \draw[->-] (h1) to ++(0,-0.5);
}
&&
\diagram{
  \draw (-0.5,0) node[circlep] (h) {} -- (0,0) node[ddot=white] (h1) {};
  \draw[->-] (h1) to ++(0,+0.5);
  \draw[-<-] (h1) to ++(0,-0.5);
}
=
\diagram{
  \draw (-0.5,0) node[circlep] (h) {} -- (0,0) node[ddot] (h1) {};
  \draw[->-] (h1) to ++(0,+0.5);
  \draw[-<-] (h1) to ++(0,-0.5);
}
+
\diagram{
  \draw (-0.5,0) node[circlep] (h) {} -- (0,0) node[ddot] (h1) {};
  \draw[->-] (h1) to ++(-0.25,+0.5);
  \draw[-<-] (h1) to ++(+0.25,+0.5);
}
+
\diagram{
  \draw (-0.5,0) node[circlep] (h) {} -- (0,0) node[ddot] (h1) {};
  \draw[->-] (h1) to ++(-0.25,-0.5);
  \draw[-<-] (h1) to ++(+0.25,-0.5);
}
+
\diagram{
  \draw (-0.5,0) node[circlep] (h) {} -- (0,0) node[ddot] (h1) {};
  \draw[-<-] (h1) to ++(0,+0.5);
  \draw[->-] (h1) to ++(0,-0.5);
}
\end{align*}
where the excitation level $\pm1$ contributions to $H_0$ have been omitted because its interaction tensor is diagonal.
\end{ntt}


\begin{rmk}
It can be shown that the determinant basis forms an eigenbasis for the diagonal part of the Fock operator.\footnotemark
\begin{align}
  H_0\F_{i_1\cd i_k}^{a_1\cd a_k}
=
  \mc{E}_{i_1\cd i_k}^{a_1\cd a_k}
  \F_{i_1\cd i_k}^{a_1\cd a_k}
&&
  H_0
\equiv
  f_p^p\tl{a}^p_p
&&
  \mc{E}_{q_1\cd q_k}^{p_1\cd p_k}
\equiv
  \sum_{r=1}^k
  f_{p_r}^{p_r}
-
  \sum_{r=1}^k
  f_{q_r}^{q_r}
\end{align}
Noting that $H_0$ is Hermitian, this implies
$\ip{\F_{ij\cd}^{ab\cd}|H_0\,T|\F}=\mc{E}_{ij\cd}^{ab\cd}\ip{\F_{ij\cd}^{ab\cd}|T|\F}=\mc{E}_{ij\cd}^{ab\cd}t_{ab\cd}^{ij\cd}$.
This can be used to rearrange the amplitude equation in
(\ref{eq:traditional-cc-equations-2}) as follows, which defines the working
equations used to iteratively solve TCC.%
\footnote{%
    Note that $\ip{\F_{ij\cd}^{ab\cd}|H_0\,\exp(T)|\F}_{\text{C}}=\ip{\F_{ij\cd}^{ab\cd}|H_0\,T|\F}$.
}
\begin{align}
  t_{ab\cd}^{ij\cd}
=
  (\mc{E}_{ab\cd}^{ij\cd})^{-1}
  \ip{\F_{ab\cd}^{ij\cd}|V_{\text{c}}\,\text{exp}(T)|\F}_{\text{C}}
&&
  V_{\text{c}}
\equiv
  H_{\text{c}}
-
  H_0
=
  f_p^q(1-\delta_p^q)
  \tl{a}^p_q
+
  \tfr{1}{4}
  \ol{g}_{pq}^{rs}
  \tl{a}^{pq}_{rs}
\end{align}
In M\o ller-Plesset perturbation theory, $H_0$ is known the \textit{zeroth order Hamiltonian} and $V_{\text{c}}$ is the \textit{perturbation}.
These operators are also known as the \textit{model Hamiltonian} and \textit{fluctuation potential}, respectively.
\end{rmk}
\footnotetext{
The proof is as follows.
First, note that $a^p_p\F_\si = n_p^\si \F_\si$, where $n_p^\si$ is the occupation of $\y_p$ in $\F_\si$.
By Wick's theorem, $a^p_p=\tl{a}^p_p+n_p^{\text{ref}}$, where $n_p^{\text{ref}}$ denotes the occupation of $\y_p$ in $\F$.
Therefore, $\tl{a}^p_p\F_\si = (n_p^\si - n_p^{\text{ref}})\F_\si$ and $H_0\F_\si = \pr{\sum_{p\in\F_\si}f_p^p - \sum_{p\in\F}f_p^p}\F_\si$.
}


\begin{dfn}
\thmtitle{Excitation level}
The \textit{excitation level} of a graph equals the net number of particles or quasi-particles it creates, divided by two.
For example, the quasi-particle excitation levels of the $T_1$, $T_2$ and $T_3$ operators are $1$, $2$, and $3$, respectively, and that of $\tl{a}_{abcd}^{ijkl}$ is $-4$.
A convenient rule for evaluating reference expectation values is that the total excitation level of a closed graph must balance out to zero.
\end{dfn}

\begin{ex}
The excitation levels in the quasi-particle expansions of one- and two-particle operators are as follows.
\begin{align*}
\diagram{
  \draw (-0.5,0) node[circlep] (h) {} -- (0,0) node[ddot=white] (h1) {};
  \draw[->-] (h1) to ++(0,+0.5);
  \draw[-<-] (h1) to ++(0,-0.5);
  \node at (0,-0.8) {$(0)$};
}
=&\
\diagram{
  \draw (-0.5,0) node[circlep] (h) {} -- (0,0) node[ddot] (h1) {};
  \draw[->-] (h1) to ++(0,+0.5);
  \draw[-<-] (h1) to ++(0,-0.5);
  \node at (0,-0.8) {$(0)$};
}
+
\diagram{
  \draw (-0.5,0) node[circlep] (h) {} -- (0,0) node[ddot] (h1) {};
  \draw[->-] (h1) to ++(-0.25,+0.5);
  \draw[-<-] (h1) to ++(+0.25,+0.5);
  \node at (0,-0.8) {$(+1)$};
}
+
\diagram{
  \draw (-0.5,0) node[circlep] (h) {} -- (0,0) node[ddot] (h1) {};
  \draw[->-] (h1) to ++(-0.25,-0.5);
  \draw[-<-] (h1) to ++(+0.25,-0.5);
  \node at (0,-0.8) {$(-1)$};
}
+
\diagram{
  \draw (-0.5,0) node[circlep] (h) {} -- (0,0) node[ddot] (h1) {};
  \draw[->-] (h1) to ++(0,-0.5);
  \draw[-<-] (h1) to ++(0,+0.5);
  \node at (0,-0.8) {$(0)$};
}
\\[10pt]
\diagram{
  \interaction{2}{g}{(0,0)}{ddot=white}{sawtooth};
  \draw[->-] (g1) to ++(0,+0.5);
  \draw[-<-] (g1) to ++(0,-0.5);
  \draw[->-] (g2) to ++(0,+0.5);
  \draw[-<-] (g2) to ++(0,-0.5);
  \node at (0.5,-0.8) {$(0)$};
}
=&\
\diagram{
  \interaction{2}{g}{(0,0)}{ddot}{sawtooth};
  \draw[->-] (g1) to ++(0,+0.5);
  \draw[-<-] (g1) to ++(0,-0.5);
  \draw[->-] (g2) to ++(0,+0.5);
  \draw[-<-] (g2) to ++(0,-0.5);
  \node at (0.5,-0.8) {$(0)$};
}
+
\diagram{
  \interaction{2}{g}{(0,0)}{ddot}{sawtooth};
  \draw[->-] (g1) to ++(0,+0.5);
  \draw[-<-] (g1) to ++(0,-0.5);
  \draw[->-] (g2) to ++(-0.25,+0.5);
  \draw[-<-] (g2) to ++(+0.25,+0.5);
  \node at (0.5,-0.8) {$(+1)$};
}
+
\diagram{
  \interaction{2}{g}{(0,0)}{ddot}{sawtooth};
  \draw[->-] (g1) to ++(0,+0.5);
  \draw[-<-] (g1) to ++(0,-0.5);
  \draw[->-] (g2) to ++(-0.25,-0.5);
  \draw[-<-] (g2) to ++(+0.25,-0.5);
  \node at (0.5,-0.8) {$(-1)$};
}
+
\diagram{
  \interaction{2}{g}{(0,0)}{ddot}{sawtooth};
  \draw[->-] (g1) to ++(-0.25,+0.5);
  \draw[-<-] (g1) to ++(+0.25,+0.5);
  \draw[->-] (g2) to ++(-0.25,+0.5);
  \draw[-<-] (g2) to ++(+0.25,+0.5);
  \node at (0.5,-0.8) {$(+2)$};
}
+
\diagram{
  \interaction{2}{g}{(0,0)}{ddot}{sawtooth};
  \draw[->-] (g1) to ++(-0.25,-0.5);
  \draw[-<-] (g1) to ++(+0.25,-0.5);
  \draw[->-] (g2) to ++(-0.25,+0.5);
  \draw[-<-] (g2) to ++(+0.25,+0.5);
  \node at (0.5,-0.8) {$(0)$};
}
+
\diagram{
  \interaction{2}{g}{(0,0)}{ddot}{sawtooth};
  \draw[->-] (g1) to ++(-0.25,-0.5);
  \draw[-<-] (g1) to ++(+0.25,-0.5);
  \draw[->-] (g2) to ++(-0.25,-0.5);
  \draw[-<-] (g2) to ++(+0.25,-0.5);
  \node at (0.5,-0.8) {$(-2)$};
}
+
\diagram{
  \interaction{2}{g}{(0,0)}{ddot}{sawtooth};
  \draw[->-] (g1) to ++(0,-0.5);
  \draw[-<-] (g1) to ++(0,+0.5);
  \draw[->-] (g2) to ++(-0.25,+0.5);
  \draw[-<-] (g2) to ++(+0.25,+0.5);
  \node at (0.5,-0.8) {$(+1)$};
}
+
\diagram{
  \interaction{2}{g}{(0,0)}{ddot}{sawtooth};
  \draw[->-] (g1) to ++(0,-0.5);
  \draw[-<-] (g1) to ++(0,+0.5);
  \draw[->-] (g2) to ++(-0.25,-0.5);
  \draw[-<-] (g2) to ++(+0.25,-0.5);
  \node at (0.5,-0.8) {$(-1)$};
}
+
\diagram{
  \interaction{2}{g}{(0,0)}{ddot}{sawtooth};
  \draw[->-] (g1) to ++(0,-0.5);
  \draw[-<-] (g1) to ++(0,+0.5);
  \draw[->-] (g2) to ++(0,-0.5);
  \draw[-<-] (g2) to ++(0,+0.5);
  \node at (0.5,-0.8) {$(0)$};
}\,\,.
\end{align*}
\end{ex}


\begin{ex}
\thmtitle{The CCSDTQ equations}
Truncating the cluster operator at quadruples, $T\approx T_1 + T_2 + T_3 + T_4$, gives the CCSDTQ approximation.
The resulting singles, doubles, triples, and quadruples amplitude equations are given by\begin{align}
\nonumber
  t_a^i
=&\
  (\mc{E}_a^i)^{-1}
  \ip{\F_i^a|
    V_{\text{c}}
    (
      1
    +
      T_1
    +
      T_2
    +
      T_3
    +
      \tfr{1}{2}
      T_1^2
    +
      T_1T_2
    +
      \tfr{1}{3!}
      T_1^3
    )
  |\F}_{\text{C}}
\\[3pt]
\nonumber
  t_{ab}^{ij}
=&\
  (\mc{E}_{ab}^{ij})^{-1}
  \br{\F_{ij}^{ab}}
    V_{\text{c}}
    (
      1
    +
      T_1
    +
      T_2
    +
      T_3
    +
      T_4
    +
      \tfr{1}{2}
      T_1^2
    +
      T_1T_2
    +
      T_1T_3
    +
      \tfr{1}{2}
      T_2^2
    +
      \tfr{1}{3!}
      T_1^3
    +
      \tfr{1}{2}
      T_1^2T_2
    +
      \tfr{1}{4!}
      T_1^4
    )
  \kt{\F}_{\text{C}}
\\[3pt]
\nonumber
  t_{abc}^{ijk}
=&\
  (\mc{E}_{abc}^{ijk})^{-1}
  \br{\F_{ijk}^{abc}}
    V_{\text{c}}
    (
      T_2
    +
      T_3
    +
      T_4
    +
      T_1T_2
    +
      T_1T_3
    +
      \tfr{1}{2}
      T_2^2
    +
      T_1T_4
    +
      T_2T_3
    +
      \tfr{1}{2}
      T_1^2T_2
    +
      \tfr{1}{2}
      T_1^2T_3
    +
      \tfr{1}{2}
      T_1T_2^2
    +
      \tfr{1}{3!}
      T_1^3T_2
    )
  \kt{\F}_{\text{C}}
\\[3pt]
\nonumber
  t_{abcd}^{ijkl}
=&\
  (\mc{E}_{abcd}^{ijkl})^{-1}
  \br{\F_{ijkl}^{abcd}}
    V_{\text{c}}
    (
      T_3
    +
      T_4
    +
      T_1T_3
    +
      \tfr{1}{2}
      T_2^2
    +
      T_1T_4
    +
      T_2T_3
    +
      T_2T_4
    +
      \tfr{1}{2}
      T_3^2
    +
      \tfr{1}{2}
      T_1^2T_3
    +
      \tfr{1}{2}
      T_1T_2^2
    +
      \tfr{1}{2}
      T_1^2T_4
\\
\nonumber
&
\hphantom{
  (\mc{E}_{abcd}^{ijkl})^{-1}
  \br{\F_{ijkl}^{abcd}}
    V_{\text{c}}
    (
      T_3\
 }
    +
      T_1T_2T_3
    +
      \tfr{1}{3!}
      T_2^3
    +
      \tfr{1}{3!}
      T_1^3T_3
    +
      \tfr{1}{2!2!}
      T_1^2T_2^2
    )
  \kt{\F}_{\text{C}}
\end{align}
where several contributions to $\text{exp}(T_1+T_2+T_3+T_4)$ have been omitted either because the excitation levels do not balance or because they require one of the cluster operators to be disconnected from the Hamiltonian.
\end{ex}


\begin{dfn}
\thmtitle{Isomorphism}
For any invertible map $S:\ol{V}\rightarrow V$, we can express operators and vectors on $V$ as
\begin{align}
  A
=
  S
  \ol{A}
  S^{-1}
&&
  \kt{v}
=
  S
  \kt{\ol{v}}
&&
  \br{v}
=
  \br{\ol{v}}
  S^{-1}
\end{align}
in terms of operators and vectors on $\ol{V}$.
Note that the transformed bra and ket, $\br{\ol{v}}$ and $\kt{\ol{v}}$, corresponding to $v$ are not adjoints unless the transformation is unitary, $S^{-1}=S\dg$.
The similarity-transformed operator $\ol{A}$ retains all of the basis-independent properties of $A$, such as its trace, determinant, and eigenvalue spectrum, and its matrix elements satisfy $\ip{\ol{v}|\ol{A}|\ol{v}'}=\ip{v|A|v'}$.
More broadly, the invertibility of $S$ implies an \textit{isomorphism} between $V$ and $\ol{V}$, such that all statements about $V$ are in one-to-one correspondence with statements about $\ol{V}$ under this transformation.
\end{dfn}


\begin{rmk}
Since exponential operators are automatically invertible, the TCC wave operator defines a similarity transformation of Fock space into itself.
The image of the Schr\"odinger equation under this transformation is as follows
\begin{align}
\label{eq:similarity-transformed-schrodinger-equation}
  \ol{H}
  \kt{\ol{\Y}_k}
=
  E_k
  \kt{\ol{\Y}_k}
&&
  \br{\ol{\Y}_k}
  \ol{H}
=
  \br{\ol{\Y}_k}
  E_k
&&
\begin{array}{c@{\ }l}
  \ol{H}
&=
  \text{exp}(-T)
  H\,
  \text{exp}(T)
\\
  \kt{\ol{\Y}_k}
&=
  \text{exp}(-T)
  \kt{\Y_k}
\\
  \br{\ol{\Y}_k}
&=
  \br{\Y_k}\,
  \text{exp}(T)
\end{array}
\end{align}
where the $k\eth$ left and right eigenstates are not adjoints because $\text{exp}(T)$ is inherently non-unitary.
In TCC, $T$ is determined by the requirement that the ground-state right eigenvector of $\ol{H}$ be the reference determinant,
$
  \kt{\ol{\Y}_0}
\overset{!}{=}
  \kt{\F}
$.
\end{rmk}


\begin{dfn}
\thmtitle{Equation-of-motion coupled-cluster theory}
Expanding the left and right eigenstates of $\ol{H}$ in the determinant basis leads to the \textit{equation-of-motion (EOM) coupled-cluster equations}
\begin{align}
\begin{array}{r@{\ }lr@{\ }l}
  \ol{H}
  \,{}^kR
  \kt{\F}
&
=
  E_k
  \,{}^kR
  \kt{\F}
&
  R
&=
  R_0
+
  R_1
+
  \cd
+
  R_n
\\[7pt]
  \br{\F}
  \,{}^kL\,
  \ol{H}
&=
  \br{\F}
  \,{}^kL\,
  E_k
&
  L
&=
  L_0
+
  L_1
+
  \cd
+
  L_n
\end{array}
&&
  \ip{\F|
    {}^kL\,
    {}^lR
  |\F}
\overset{!}{=}
  \delta_{kl}
\end{align}
which are analogous to the configuration interaction eigenvalue equation, ${}^kR$ and ${}^kL$ being linear excitation and de-excitation operators analogous to $C$ and $C\dg$.
The condition on the right indicates that left and right eigenstates are chosen to form a \textit{biorthonormal system} by normalizing the overlap of the $k\eth$ left and right eigenfunctions to equal one.
Observable expectation values are given by
$
  \ip{\Y_k|W|\Y_k}
=
  \ip{\F|{}^kL\,\ol{W}\,{}^kR|\F}
$
in terms of left and right EOM coefficients.
Transition matrix elements are given by
$
  \ip{\Y_k|W|\Y_l}
=
  \ip{\F|{}^kL\,\ol{W}\,{}^lR|\F}
$.
\end{dfn}

\begin{dfn}
\thmtitle{The coupled-cluster Lagrangian}
The ground-state TCC equations are equivalent to requiring ${}^0R=1$, which implies ${}^0L_0=1$ from the biorthonormality condition.
Therefore, the ground-state left eigenvector has the form
\begin{align}
\label{eq:cc-ground-state-wave-operators}
  {}^0L
=
  1
+
  \La
&&
  \La
=
  \La_1
+
  \cd
+
  \La_n
&&
  \La_k
\equiv
  (\tfr{1}{k!})^2
  \la_{i_1\cd i_k}^{a_1\cd a_k}
  \tl{a}^{i_1\cd i_k}_{a_1\cd a_k}
\end{align}
and the ground-state energy can be written as follows, in an expression known as the \textit{coupled-cluster Lagrangian}.
\begin{align}
  \ip{\Y_0|H|\Y_0}
=
  \ip{\F|
    (
      1
    +
      \La
    )
    \ol{H}
  |\F}
\equiv
  \mc{L}(\bo{t},\bm{\la})
\end{align}
To see why this constitutes a Lagrangian, note that setting its gradient with respect to $\la_{i_1\cd i_k}^{a_1\cd a_k}$ equal to zero yields the TCC amplitude equations of eq~\ref{eq:traditional-cc-equations}.
If these are satisfied, the $\la$-dependent part of the equation vanishes and the Lagrangian returns the coupled-cluster energy:
$
  \ip{\F|\ol{H}|\F}
=
  E
$.
Therefore, the $\la$ coefficients can be viewed as Lagrange multipliers enforcing the TCC amplitude equations as a constraint.
\end{dfn}


\begin{dfn}
\thmtitle{The coupled-cluster lambda equations}
Setting the gradient of $\mc{L}$ with respect to $t_{a_1\cd a_k}^{i_1\cd i_k}$ equal to zero gives the \textit{coupled-cluster lambda equations}, which determine the Lagrange multpliers.
\begin{align}
  \br{\F}
  (
    1
  +
    \La
  )
  H_{\text{c}}\,
  \text{exp}(T)
  \kt{\F_{i_1\cd i_k}^{a_1\cd a_k}}_{\text{C}}
\overset{!}{=}
  0
\end{align}
The subscript $\text{C}$ now denotes that $H_{\text{c}}$ is connected both to the ket and to the $T$ operators.
This can be rearranged as\footnote{Note that $\ip{\F|\La H_0 T^p|\F_{i_1\cd i_k}^{a_1\cd a_k}}_{\text{C}}=0$ for $p\geq 1$, since the model Hamiltonian can only connect to one operator on either side.}
\begin{align}
  \la_{i_1\cd i_k}^{a_1\cd a_k}
=
  (\mc{E}_{a_1\cd a_k}^{i_1\cd i_k})^{-1}
  \br{\F}
  (
    1
  +
    \La
  )
  V_{\text{c}}\,
  \text{exp}(T)
  \kt{\F_{i_1\cd i_k}^{a_1\cd a_k}}_{\text{C}}
\end{align}
which sets up the iterative procedure for determining $\la_{i_1\cd i_k}^{a_1\cd a_k}$ from a given set of amplitudes.
\end{dfn}


\begin{ex}
\thmtitle{The CCSD lambda equations}
Truncating the wave operator at doubles, $T\approx T_1 + T_2$, gives the CCSD approximation.
The resulting singles and doubles lambda equations are given by the following
\begin{align*}
  \la_i^a
=&\
  (\mc{E}_a^i)^{-1}
  \br{\F}
    V_{\text{c}}\,
    (
      1
    +
      T_1
    )
  +
    \La_1
    V_{\text{c}}\,
    (
      1
    +
      T_1
    +
      T_2
    +
      \tfr{1}{2}
      T_1^2
    )
  +
    \La_2
    V_{\text{c}}\,
    (
      1
    +
      T_1
    +
      T_2
    +
      \tfr{1}{2}
      T_1^2
    +
      T_1
      T_2
    )
  \kt{\F_i^a}_{\text{C}}
\\
  \la_{ij}^{ab}
=&\
  (\mc{E}_{ab}^{ij})^{-1}
  \br{\F}
    V_{\text{c}}\,
  +
    \La_1
    V_{\text{c}}\,
    (
      1
    +
      T_1
    )
  +
    \La_2
    V_{\text{c}}\,
    (
      1
    +
      T_1
    +
      T_2
    +
      \tfr{1}{2}
      T_1^2
    )
  \kt{\F_{ij}^{ab}}_{\text{C}}
\end{align*}
where we have omitted any contributions to $\text{exp}(T_1 + T_2)$ that vanish.
\end{ex}

\begin{ex}
Assuming Brillouin's theorem holds, the CCD lambda equations are as follows.
\begin{align*}
  \la_{ij}^{ab}
  \mc{E}_{ab}^{ij}
=&\
  \ip{\F|
    V_{\text{c}}
  +
    \La_2
    V_{\text{c}}
  +
    \La_2 V_{\text{c}}T_2
  |\F_{ij}^{ab}}_{\text{C}}
\\=&\
\diagram{
  \interaction{2}{g}{(0,0.5)}{ddot}{sawtooth};
  \draw[->-] (g1) to ++(-0.25,-1) node[smalldot] {};
  \draw[-<-] (g1) to ++(+0.25,-1) node[smalldot] {};
  \draw[->-] (g2) to ++(-0.25,-1) node[smalldot] {};
  \draw[-<-] (g2) to ++(+0.25,-1) node[smalldot] {};
}
+
\diagram{
  \interaction{2}{t}{(0,0.5)}{ddot}{overhang};
  \draw[->-=0.65] (t1) to ++(-0.25,-1) node[smalldot] {};
  \draw[-<-=0.25,-<-=0.75]
      (t1)
    to
      node[ddot,midway] (g1) {}
    ++(+0.25,-1)
      node[smalldot] {};
  \draw[->-=0.65] (t2) to ++(-0.25,-1) node[smalldot] {};
  \draw[-<-=0.25,-<-=0.75]
      (t2)
    to
      node[ddot,midway] (g2) {}
    ++(+0.25,-1)
      node[smalldot] {};
   \draw[sawtooth] (g1) to (g2);
}
+
\diagram{
  \interaction{2}{t}{(0,0.5)}{ddot}{overhang};
  \draw[-<-=0.65] (t1) to ++(-0.25,-1) node[smalldot] {};
  \draw[->-=0.25,->-=0.75]
      (t1)
    to
      node[ddot,midway] (g1) {}
    ++(+0.25,-1)
      node[smalldot] {};
  \draw[-<-=0.65] (t2) to ++(-0.25,-1) node[smalldot] {};
  \draw[->-=0.25,->-=0.75]
      (t2)
    to
      node[ddot,midway] (g2) {}
    ++(+0.25,-1)
      node[smalldot] {};
   \draw[sawtooth] (g1) to (g2);
}
+
\diagram{
  \interaction{2}{t}{(0,0.5)}{ddot}{overhang};
  \interaction{2}{g}{(1,0.)}{ddot}{sawtooth};
  \draw[-<-] (t1) to ++(-0.25,-1) node[smalldot] {};
  \draw[->-] (t1) to ++(+0.25,-1) node[smalldot] {};
  \draw[->-,bend left=40 ] (g1) to (t2);
  \draw[-<-,bend right=40] (g1) to (t2);
  \draw[-<-] (g2) to ++(-0.25,-0.5) node[smalldot] {};
  \draw[->-] (g2) to ++(+0.25,-0.5) node[smalldot] {};
}
+
\diagram{
  \interaction{2}{l}{(0,0.5)}{ddot}{overhang};
  \interaction{2}{g}{(2,0.5)}{ddot}{sawtooth};
  \draw[overhang] (1.5,0) node[ddot] (t1) {} to ++(1.5,0) node[ddot] (t2) {};
  \draw[-<-=0.65] (l1) to ++(-0.25,-1) node[smalldot] {};
  \draw[->-=0.65] (l1) to ++(+0.25,-1) node[smalldot] {};
  \draw[-<-=0.65] (l2) to ++(-0.25,-1) node[smalldot] {};
  \draw[-<-] (t1) to (l2);
  \draw[->-] (t1) to (g1);
  \draw[->-=0.65] (g1) to ++(+0.25,-1) node[smalldot] {};
  \draw[-<-,bend left=40 ] (t2) to (g2);
  \draw[->-,bend right=40] (t2) to (g2);
}
+
\diagram{
  \interaction{2}{l}{(0,0.5)}{ddot}{overhang};
  \interaction{2}{g}{(2,0.5)}{ddot}{sawtooth};
  \draw[overhang] (1.5,0) node[ddot] (t1) {} to ++(1.5,0) node[ddot] (t2) {};
  \draw[->-=0.65] (l1) to ++(-0.25,-1) node[smalldot] {};
  \draw[-<-=0.65] (l1) to ++(+0.25,-1) node[smalldot] {};
  \draw[->-=0.65] (l2) to ++(-0.25,-1) node[smalldot] {};
  \draw[->-] (t1) to (l2);
  \draw[-<-] (t1) to (g1);
  \draw[-<-=0.65] (g1) to ++(+0.25,-1) node[smalldot] {};
  \draw[->-,bend left=40 ] (t2) to (g2);
  \draw[-<-,bend right=40] (t2) to (g2);
}
\\&\
+
\diagram{
  \interaction{2}{l}{(0,0.5)}{ddot}{overhang};
  \interaction{2}{g}{(2.2,0.5)}{ddot}{sawtooth};
  \interaction{2}{t}{(1.1,0.0)}{ddot}{overhang};
  \draw[->-=0.7] (l1) to ++(-0.4,-1) node[smalldot] {};
  \draw[-<-=0.3] (l1) to (t1);
  \draw[->-=0.7] (l2) to ++(-0.4,-1) node[smalldot] {};
  \draw[-<-=0.3] (l2) to (t2);
  \draw[->-=0.3] (g1) to (t1);
  \draw[-<-=0.7] (g1) to ++(+0.4,-1) node[smalldot] {};
  \draw[->-=0.3] (g2) to (t2);
  \draw[-<-=0.7] (g2) to ++(+0.4,-1) node[smalldot] {};
}
+
\diagram{
  \interaction{2}{l}{(0,0.5)}{ddot}{overhang};
  \interaction{2}{g}{(2,0.5)}{ddot}{sawtooth};
  \interaction{2}{t}{(1,0.0)}{ddot}{overhang};
  \draw[-<-=0.65] (l1) to ++(-0.25,-1) node[smalldot] {};
  \draw[->-=0.65] (l1) to ++(+0.25,-1) node[smalldot] {};
  \draw[->-,bend left=40 ] (t1) to (l2);
  \draw[-<-,bend right=40] (t1) to (l2);
  \draw[->-,bend left=40 ] (t2) to (g1);
  \draw[-<-,bend right=40] (t2) to (g1);
  \draw[-<-=0.65] (g2) to ++(-0.25,-1) node[smalldot] {};
  \draw[->-=0.65] (g2) to ++(+0.25,-1) node[smalldot] {};
}
+
\diagram{
  \interaction{2}{g}{(0,0.5)}{ddot}{sawtooth};
  \interaction{2}{l}{(2.2,0.5)}{ddot}{overhang};
  \interaction{2}{t}{(1.1,0.0)}{ddot}{overhang};
  \draw[->-=0.7] (g1) to ++(-0.4,-1) node[smalldot] {};
  \draw[-<-=0.3] (g1) to (t1);
  \draw[->-=0.7] (g2) to ++(-0.4,-1) node[smalldot] {};
  \draw[-<-=0.3] (g2) to (t2);
  \draw[->-=0.3] (l1) to (t1);
  \draw[-<-=0.7] (l1) to ++(+0.4,-1) node[smalldot] {};
  \draw[->-=0.3] (l2) to (t2);
  \draw[-<-=0.7] (l2) to ++(+0.4,-1) node[smalldot] {};
}
\\&\
+
\diagram{
  \interaction{2}{g}{(0,0.5)}{ddot}{sawtooth};
  \interaction{2}{l}{(2,0.5)}{ddot}{overhang};
  \draw[overhang] (1.5,0) node[ddot] (t1) {} to ++(1.5,0) node[ddot] (t2) {};
  \draw[-<-=0.65] (g1) to ++(-0.25,-1) node[smalldot] {};
  \draw[->-=0.65] (g1) to ++(+0.25,-1) node[smalldot] {};
  \draw[-<-=0.65] (g2) to ++(-0.25,-1) node[smalldot] {};
  \draw[-<-] (t1) to (g2);
  \draw[->-] (t1) to (l1);
  \draw[->-=0.65] (l1) to ++(+0.25,-1) node[smalldot] {};
  \draw[-<-,bend left=40 ] (t2) to (l2);
  \draw[->-,bend right=40] (t2) to (l2);
}
+
\diagram{
  \interaction{2}{g}{(0,0.5)}{ddot}{sawtooth};
  \interaction{2}{l}{(2,0.5)}{ddot}{overhang};
  \draw[overhang] (1.5,0) node[ddot] (t1) {} to ++(1.5,0) node[ddot] (t2) {};
  \draw[->-=0.65] (g1) to ++(-0.25,-1) node[smalldot] {};
  \draw[-<-=0.65] (g1) to ++(+0.25,-1) node[smalldot] {};
  \draw[->-=0.65] (g2) to ++(-0.25,-1) node[smalldot] {};
  \draw[->-] (t1) to (g2);
  \draw[-<-] (t1) to (l1);
  \draw[-<-=0.65] (l1) to ++(+0.25,-1) node[smalldot] {};
  \draw[->-,bend left=40 ] (t2) to (l2);
  \draw[-<-,bend right=40] (t2) to (l2);
}
\\=&\
  \ol{g}_{ij}^{ab}
+
  \tfr{1}{2}
  \la_{ij}^{cd}
  \ol{g}_{cd}^{ab}
+
  \tfr{1}{2}
  \la_{kl}^{ab}
  \ol{g}_{ij}^{kl}
+
  P_{(i/j)}^{(a/b)}
  \la_{ik}^{ac}
  \ol{g}_{cj}^{kb}
-
  \tfr{1}{2}
  P_{(i/j)}
  \la_{ik}^{ab}
  t_{cd}^{kl}
  \ol{g}_{jl}^{cd}
-
  \tfr{1}{2}
  P^{(a/b)}
  \la_{ij}^{ac}
  t_{cd}^{kl}
  \ol{g}_{kl}^{bd}
\\&\
+
  \tfr{1}{2^2}
  \la_{ij}^{cd}
  t_{cd}^{kl}
  \ol{g}_{kl}^{ab}
+
  P_{(i/j)}^{(a/b)}
  \la_{ik}^{ac}
  t_{cd}^{kl}
  \ol{g}_{lj}^{db}
+
  \tfr{1}{2^2}
  \ol{g}_{ij}^{cd}
  t_{cd}^{kl}
  \la_{kl}^{ab}
-
  \tfr{1}{2}
  P_{(i/j)}
  \ol{g}_{ik}^{ab}
  t_{cd}^{kl}
  \la_{jl}^{cd}
-
  \tfr{1}{2}
  P^{(a/b)}
  \ol{g}_{ij}^{ac}
  t_{cd}^{kl}
  \la_{kl}^{bd}
\end{align*}
\end{ex}

\begin{samepage}
\begin{rmk}
\thmtitle{The Hellmann-Feynman theorem}
If the Hamiltonian depends on a parameter $\xi$, such as a nuclear coordinate or an electric field strength, we can express the Schr\"odinger equation as a function of that parameter
\begin{align}
\label{eq:xi-dependent-schrodinger-equation}
  H(\xi)
  \Y(\xi)
=
  E(\xi)
  \Y(\xi)
&&
  \ip{\Y(\xi)|\Y(\xi)}
\overset{!}{=}
  1
\end{align}
Then the total derivative of the energy $E(\xi)=\ip{\Y(\xi)|H(\xi)|\Y(\xi)}$ with respect to $\xi$ is given by the following.
\begin{align}
  \fd{E(\xi)}{\xi}
=
  \ip{\Y(\xi)|
  \pd{H(\xi)}{\xi}
  |\Y(\xi)}
+
\cancel{
  \ip{\pd{\Y(\xi)}{\xi}|
  H(\xi)
  |\Y(\xi)}
+
  \ip{\Y(\xi)|
  H(\xi)
  |\pd{\Y(\xi)}{\xi}}
}
\end{align}
The second and third terms cancel by the \textit{Hellmann-Feynman theorem}, which one can prove by substituting the Schr\"odinger equation into both terms and employing the derivative of the normalization condition with respect to $\xi$.
In words, it says that the first derivative of the energy does not depend on the  ``response'' of the wavefunction.
More generally, for approximate methods, the wavefunction may be parametrized by a set of coefficients $\bo{c}$ which are \textit{stationary} in the sense that their energy gradient equals zero.
Denoting the remaining non-stationary coordinates by $\bo{p}$, we have
\begin{align}
  \fd{E(\xi)}{\xi}
=
  \pd{E(\xi)}{\xi}
+
  \cancel{
  \pd{E}{\bo{c}}
  }
  \cdot
  \fd{\bo{c}}{\xi}
+
  \pd{E}{\bo{p}}
  \cdot
  \fd{\bo{p}}{\xi}
&&
  \pd{E}{\xi}
=
  \ip{\Y(\bo{c},\bo{p})|
  \pd{H(\xi)}{\xi}
  |\Y(\bo{c},\bo{p})}
\end{align}
which applies to configuration interaction and other variational methods.
For most methods $\bo{p}$ will include parameters which determine the molecular orbital coefficients, since the shape of the Hartree-Fock orbitals depends on $\xi$.
When $\xi$ is a nuclear coordinate, the atomic-orbital basis functions themselves change\footnote{The parameters defining the basis functions are constant, but their centers of origin move with the nuclei.}
and we must include parameters to account for this as well.
The Hellmann-Feynman theorem does not apply to TCC energy, which is not stationary in any of its parameters.
However, it does apply to the coupled-cluster Lagrangian as follows.
\begin{align}
  \fd{\mc{L}(\xi)}{\xi}
=
  \pd{\mc{L}(\xi)}{\xi}
+
  \cancel{
  \pd{\mc{L}}{\bo{t}}
  }
  \cdot
  \fd{\bo{t}}{\xi}
+
  \cancel{
  \pd{\mc{L}}{\bm{\la}}
  }
  \cdot
  \fd{\bm{\la}}{\xi}
+
  \pd{\mc{L}}{\bo{p}}
  \cdot
  \fd{\bo{p}}{\xi}
&&
  \pd{\mc{L}(\xi)}{\xi}
=
  \ip{\F(\bo{p})|
    (
      1
    +
      \La
    )\,
    \pd{H(\xi)}{\xi}\,
    \text{exp}(T)
  |\F(\bo{p})}_{\text{C}}
\end{align}
This is known as the \textit{generalized Hellmann-Feynman theorem} for coupled-cluster theory.
Note that one must solve both the amplitude equations and the lambda equations in order to evaluate the equation on the right.
\end{rmk}
\end{samepage}


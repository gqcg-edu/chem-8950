\chapter{Perturbative analysis}


\begin{dfn}
\thmtitle{Correct to order $p$}
Let
$
  X\ord{p^-}
$
and
$
  X\ord{p^+}
$
denote
$
  X\ord{0}
+
  X\ord{1}
+
  \cd
+
  X\ord{p}
$
and
$
  X\ord{p}
+
  X\ord{p+1}
+
  \cd
+
  X\ord{\infty}
$.
We say that an approximation to $X$ is \textit{correct to order $p$} if it contains all of the contributions in $X\ord{p^-}$.
\end{dfn}

\begin{dfn}
\thmtitle{Truncated CI and CC}
Let $\text{CISDTQPH7}{\cd}m$ denote truncation of the CI Ansatz at $m$-tuples.
Similarly, $\text{CCS}{\cd}m$ means that we omit cluster operators of excitation level $k>m$.
Note that, unlike $C_k$, the cluster operator $T_k$ in truncated CC describes only \textit{connected $k$-tuples}, which are residual correlations that cannot be decomposed into products of smaller clusters.
As a result, truncated CC actually contains much higher excitations than CI.
\end{dfn}

\begin{ntt}
Let
$\bm{\F}_k$
be a row vector containing all unique $k$-fold substitutions of $\F$,
so that $\bm{\F}=\pma{\F\ \bm{\F}_1\,\cd\,\bm{\F}_n}$ spans $\mc{F}_n$.
Using $\bo{c}$ and $\bo{t}$ to denote column vectors of CI coefficients and CC amplitudes leads to the following relationships.
\begin{align}
  \bo{\F}\cdot\bo{c}
=
  (1 + C_1+\cd +C_n)\F
&&
  \bo{\F}\cdot\bo{t}
=
  (1 + T_1+\cd +T_n)\F
\end{align}
In Dirac notation the bra $\br{\bo{\F}}$ is transposed, so that
$
  \ip{\bm{\F}|W|\bm{\F}}
=
  [\ip{\F_\si|W|\F_\ta}]
$
is the matrix representation of $W$ in $\mc{F}_n$.
Such matrix representations will be denoted with bolded letters,
$
  \bo{W}
\equiv
  \ip{\bm{\F}|W|\bm{\F}}
$.
\end{ntt}



\begin{rmk}
\label{rmk:ci-perturbative-analysis}
\thmtitle{Perturbative analysis of CI}
Writing the CI eigenvalue equation $\bo{H}_{\text{c}}\,\bo{c}=E_{\text{c}}\,\bo{c}$ in terms of model-Hamiltonian and fluctuation-potential matrices leads to the following rearranged matrix equation
\begin{align}
\label{eq:ci-matrix-equation}
  (
  -
    \bo{H}_0
  +
    E_{\text{c}}
  )\,
  \bo{c}
=
  \bo{V}_{\text{c}}\,
  \bo{c}
\end{align}
which provide a convenient starting point for perturbative analysis and for comparison to the coupled-cluster equations.
The matrix elements of the model Hamiltonian are given by
$
  \ip{\F_\si|H_0|\F_\ta}
=
  \mc{E}_\ta
  \delta_{\si\ta}
$,
so the matrix on the left is diagonal with eigenvalues
$
-
  \mc{E}_{i_1\cd i_k}^{a_1\cd a_k}
+
  E_{\text{c}}
$.
The rows of this equation can be written in terms of CI operators as follows\footnote{See \url{https://en.wikipedia.org/wiki/Floor_and_ceiling_functions} for details on the floor and ceiling functions.}
\begin{align}
\label{eq:ci-reference-equation}
  \underset{(2^+)}{\vphantom{(}
    E_{\text{c}}
  }
=&\
  \ip{\F|
  \underset{(1)}{\vphantom{(}
    V_{\text{c}}
  }
    (
    \underset{((1^+))}{\vphantom{(}
      C_1
    }
    +
    \underset{(1^+)}{\vphantom{(}
      C_2
    }
    )
  |\F}
\\
\label{eq:ci-singles-equation}
  \underset{(1^+)}{\vphantom{(}
  c_a^i
  }
  (\hspace{1pt}
  \underset{(0)}{\vphantom{(}
    \mc{E}_a^i
  }
  +
  \underset{(2^+)}{\vphantom{(}
    E_{\text{c}}
  }
  \hspace{-2pt}
  )
=&\
  \ip{\F_i^a|
  \underset{(1)}{\vphantom{(}
    V_{\text{c}}
  }
    (
    \underset{((0))}{\vphantom{(}
      1
    }
    +
    \underset{(1^+)}{\vphantom{(}
      C_1
    }
    +
    \underset{(1^+)}{\vphantom{(}
      C_2
    }
    +
    \underset{(2^+)}{\vphantom{(}
      C_3
    }
    )
  |\F}
\\
\label{eq:ci-doubles-equation}
  \underset{(1^+)}{\vphantom{(}
  c_{ab}^{ij}
  }
  (\hspace{1pt}
  \underset{(0)}{\vphantom{(}
    \mc{E}_{ab}^{ij}
  }
  +
  \underset{(2^+)}{\vphantom{(}
    E_{\text{c}}
  }
  \hspace{-2pt}
  )
=&\
  \ip{\F_{ij}^{ab}|
  \underset{(1)}{\vphantom{(}
    V_{\text{c}}
  }
    (
    \underset{(0^+)}{\vphantom{(}
      1
    }
    +
    \underset{(1^+)}{\vphantom{(}
      C_1
    }
    +
    \underset{(1^+)}{\vphantom{(}
      C_2
    }
    +
    \underset{(2^+)}{\vphantom{(}
      C_3
    }
    +
    \underset{(2^+)}{\vphantom{(}
      C_4
    }
    )
  |\F}
\\
  \underset{(2^+)}{\vphantom{(}
  c_{abc}^{ijk}
  }
  (\hspace{1pt}
  \underset{(0)}{\vphantom{(}
    \mc{E}_{abc}^{ijk}
  }
  +
  \underset{(2^+)}{\vphantom{(}
    E_{\text{c}}
  }
  \hspace{-2pt}
  )
=&\
  \ip{\F_{ijk}^{abc}|
  \underset{(1)}{\vphantom{(}
    V_{\text{c}}
  }
    (
    \underset{(1^+)}{\vphantom{(}
      C_1
    }
    +
    \underset{(1^+)}{\vphantom{(}
      C_2
    }
    +
    \underset{(2^+)}{\vphantom{(}
      C_3
    }
    +
    \underset{(2^+)}{\vphantom{(}
      C_4
    }
    +
    \underset{(3^+)}{\vphantom{(}
      C_5
    }
    )
  |\F}
\\
  \underset{(2^+)}{\vphantom{(}
  c_{abcd}^{ijkl}
  }
  (\hspace{1pt}
  \underset{(0)}{\vphantom{(}
    \mc{E}_{abcd}^{ijkl}
  }
  +
  \underset{(2^+)}{\vphantom{(}
    E_{\text{c}}
  }
  \hspace{-2pt}
  )
\underset{\hspace{2pt}\vdots}{\vphantom{(}
=
}&\
  \ip{\F_{ijkl}^{abcd}|
  \underset{(1)}{\vphantom{(}
    V_{\text{c}}
  }
    (
    \underset{(1^+)}{\vphantom{(}
      C_2
    }
    +
    \underset{(2^+)}{\vphantom{(}
      C_3
    }
    +
    \underset{(2^+)}{\vphantom{(}
      C_4
    }
    +
    \underset{(3^+)}{\vphantom{(}
      C_5
    }
    +
    \underset{(3^+)}{\vphantom{(}
      C_6
    }
    )
  |\F}
\\
\label{eq:ci-k-tuples-equation}
  \underset{(\ceil{k/2}^+)}{\vphantom{(}
  c_{a_1\cd a_k}^{i_1\cd i_k}
  }
  (\hspace{1pt}
  \underset{(0)}{\vphantom{(}
    \mc{E}_{a_1\cd a_k}^{i_1\cd i_k}
  }
  +
  \underset{(2^+)}{\vphantom{(}
    E_{\text{c}}
  }
  \hspace{-2pt}
  )
=&\
  \ip{\F_{i_1\cd i_k}^{a_1\cd a_k}|
  \underset{(1)}{\vphantom{(}
    V_{\text{c}}
  }
    (
    \underset{(\ceil{k/2}^+-1)}{\vphantom{(}
      C_{k-2}
    }
    +
    \underset{(\ceil{(k-1)/2}^+)}{\vphantom{(}
      C_{k-1}
    }
    +
    \underset{(\ceil{k/2}^+)}{\vphantom{(}
      C_k
    }
    +
    \underset{(\ceil{(k+1)/2}^+)}{\vphantom{(}
      C_{k+1}
    }
    +
    \underset{(\ceil{k/2}^++\,1)}{\vphantom{(}
      C_{k+2}
    }
    )
  |\F}
\end{align}
where the numbers in parentheses denote orders in perturbation theory and the double parentheses denote terms which vanish under Brillouin's theorem.
The orders of the CI operators follow from the fact that each order in perturbation theory increases the maximum excitation level of the wavefunction by $+2$, starting from $\Y\ord{1}$ which contains up to doubles.
Therefore the leading contributions to $C_k$ have order $\ceil{k/2}$.
If Brillouin's theorem holds, the first-order contribution to $C_1$ vanishes and singles contribute at order $2^+$ in perturbation theory.
\end{rmk}


\begin{ex}
\label{ex:analysis-of-cisd-throuh-q}
The analysis of remark~\ref{rmk:ci-perturbative-analysis} shows that CISD is only correct to first order in the wavefunction, since triples contribute at second order.
However, $C_1$ and $C_2$ are both correct to second order, ensuring that the the CISD correlation energy is correct to third order.
In order to gain an order in perturbation theory we have to increase the truncation level by at least two, since triples and quadruples contribute at the same order.
CISDTQ is correct to second order in the wavefunction and fifth order in the energy.
\end{ex}


\begin{prop}
\label{prop:ci-orders}
\thmstatement{
$\text{CIS}{\cd}m$
is correct to order
$\floor{m/2}$
in the wavefunction and order
$2\floor{m/2}+1$
in the energy.
}
\thmproof{
The terms omitted from $\Y$ and from $E_{\text{c}}=\ip{\Y|H_{\text{c}}|\Y}/\ip{\Y|\Y}$ when we truncate at $m$-tuples are
\begin{align*}
  \Y
=
  (
    1
  +
    C_1
  +
    \cd
  +
    \cancel{C_{m+1}}
  +
    \cd
  +
    \cancel{C_n}
  )
  \F
&&
  E_{\text{c}}
=
  \fr{
    \ip{\F|H_{\text{c}}\,C_1|\F}
  +
    \cd
  +
    \ip{\F|\cancel{
        C_{m+1}\dg H_{\text{c}}\,C_{m+1}
    }|\F}
  +
    \cd
  +
    \ip{\F|\cancel{
      C_n\dg H_{\text{c}}C_n
    }|\F}
  }{
    \ip{\F|\F}
  +
    \cd
  +
    \ip{\F|\cancel{C_{m+1}\dg C_{m+1}}|\F}
  +
    \cd
  +
    \ip{\F|\cancel{C_n\dg C_n}|\F}
  }
\end{align*}
yielding errors of $\ceil{(m+1)/2}$ in the wavefunction and $2\ceil{(m+1)/2}$ in the energy.\footnote{
  There are also higher-order errors due to the indirect effect of truncation on $C_1$ through $C_m$, but these are always smaller than the direct error
}
Therefore, the wavefunction is correct to order
$
  \ceil{(m+1)/2} - 1
=
  \floor{m/2}
$
and the energy is correct to order
$
  2\ceil{(m+1)/2} - 1
=
  2\floor{m/2} + 1
$.
}
\end{prop}



\begin{rmk}
\label{rmk:cc-perturbative-analysis}
\thmtitle{Perturbative analysis of CC}
The CC equations can be written as a non-linear matrix equation.
\begin{align}
\label{eq:cc-matrix-equation}
(
-
  \bo{H}_0
+
  E_{\text{c}}\,
  \ip{\bm{\F}|\F}
  \ip{\F|\bm{\F}}
)
  \bo{t}
=
  \ip{\bm{\F}|
    V_{\text{c}}\,
    \text{exp}(T(\bo{t}))
  |\F}_{\text{c}}
\end{align}
To make the comparison with equation~\ref{eq:ci-matrix-equation} more transparent, note that the right-hand side equals
$
  (
    \bo{V}_{\text{c}}\hspace{0.8pt}
    \bo{t}
  )_{\text{C}}
+
  \mc{O}(\bo{t}^2)
$.
Non-vanishing contributions to the rows of this matrix equation can be expressed in terms of cluster operators as follows
% % ERROR
% \begin{align}
% \label{eq:cc-reference-equation}
%   \underset{(2^+)}{\vphantom{(}
%   E_{\text{c}}
%   }
% =&\
%   \ip{\F|
%   \underset{(1)}{\vphantom{(}
%     V_{\text{c}}
%   }
%     (
%     \underset{((1^{+}))}{\vphantom{(}
%       T_1
%     }
%     +
%     \underset{(1^{+})}{\vphantom{(}
%       T_2
%     }
%     +
%       \tfr{1}{2}
%     \underset{(2^{+})}{\vphantom{(}
%       T_1^2
%     }
%     )
%   |\F}_{\text{c}}
% \\[3pt]
%   \underset{(1^+)}{\vphantom{(}
%   t_a^i
%   }
%   \underset{(0)}{\vphantom{(}
%   \mc{E}_a^i
%   }
% =&\
%   \ip{\F_i^a|
%   \underset{(1)}{\vphantom{(}
%     V_{\text{c}}
%   }
%     (
%     \underset{((0))}{\vphantom{(}
%       1
%     }
%     +
%     \underset{(1^{+})}{\vphantom{(}
%       T_1
%     }
%     +
%     \underset{(1^{+})}{\vphantom{(}
%       T_2
%     }
%     +
%     \underset{(2^{+})}{\vphantom{(}
%       T_3
%     }
%     +
%       \tfr{1}{2}
%     \underset{(2^{+})}{\vphantom{(}
%       T_1^2
%     }
%     +
%     \underset{(2^{+})}{\vphantom{(}
%       T_1T_2
%     }
%     +
%       \tfr{1}{3!}
%     \underset{(3^{+})}{\vphantom{(}
%       T_1^3
%     }
%     )
%   |\F}_{\text{c}}
% \\[3pt]
% \nonumber
%   \underset{(1^+)}{\vphantom{(}
%   t_{ab}^{ij}
%   }
%   \underset{(0)}{\vphantom{(}
%   \mc{E}_{ab}^{ij}
%   }
% =&\
%   \br{\F_{ij}^{ab}}
%   \underset{(1)}{\vphantom{(}
%     V_{\text{c}}
%   }
%     (
%     \underset{(0)}{\vphantom{(}
%       1
%     }
%     +
%     \underset{(1^{+})}{\vphantom{(}
%       T_1
%     }
%     +
%     \underset{(1^{+})}{\vphantom{(}
%       T_2
%     }
%     +
%     \underset{(2^{+})}{\vphantom{(}
%       T_3
%     }
%     +
%     \underset{(3^{+})}{\vphantom{(}
%       T_4
%     }
%     +
%       \tfr{1}{2}
%     \underset{(2^{+})}{\vphantom{(}
%       T_1^2
%     }
%     +
%     \underset{(2^{+})}{\vphantom{(}
%       T_1T_2
%     }
%     +
%     \underset{(3^{+})}{\vphantom{(}
%       T_1T_3
%     }
% \\
% \label{eq:cc-doubles-equation}
% &
% \makebox[0.75\linewidth][r]{\ensuremath{
%     +
%       \tfr{1}{2}
%     \underset{(2^{+})}{\vphantom{(}
%       T_2^2
%     }
%     +
%       \tfr{1}{3!}
%     \underset{(3^{+})}{\vphantom{(}
%       T_1^3
%     }
%     +
%       \tfr{1}{2}
%     \underset{(3^{+})}{\vphantom{(}
%       T_1^2T_2
%     }
%     +
%       \tfr{1}{4!}
%     \underset{(4^{+})}{\vphantom{(}
%       T_1^4
%     }
%     )
%   \kt{\F}_{\text{c}}
% }}
% \\[3pt]
% \nonumber
%   \underset{(2^+)}{\vphantom{(}
%   t_{abc}^{ijk}
%   }
%   \underset{(0)}{\vphantom{(}
%   \mc{E}_{abc}^{ijk}
%   }
% =&\
%   \br{\F_{ijk}^{abc}}
%   \underset{(1)}{\vphantom{(}
%     V_{\text{c}}
%   }
%     (
%     \underset{(1^{+})}{\vphantom{(}
%       T_2
%     }
%     +
%     \underset{(2^{+})}{\vphantom{(}
%       T_3
%     }
%     +
%     \underset{(3^{+})}{\vphantom{(}
%       T_4
%     }
%     +
%     \underset{(4^{+})}{\vphantom{(}
%       T_5
%     }
%     +
%     \underset{(2^{+})}{\vphantom{(}
%       T_1T_2
%     }
%     +
%     \underset{(3^{+})}{\vphantom{(}
%       T_1T_3
%     }
%     +
%       \tfr{1}{2}
%     \underset{(2^{+})}{\vphantom{(}
%       T_2^2
%     }
%     +
%     \underset{(4^{+})}{\vphantom{(}
%       T_1T_4
%     }
%     +
%     \underset{(3^{+})}{\vphantom{(}
%       T_2T_3
%     }
% \\
% \label{eq:cc-triples-equation}
% &
% \makebox[0.75\linewidth][r]{\ensuremath{
%     +
%       \tfr{1}{2}
%     \underset{(3^{+})}{\vphantom{(}
%       T_1^2T_2
%     }
%     +
%       \tfr{1}{2}
%     \underset{(4^{+})}{\vphantom{(}
%       T_1^2T_3
%     }
%     +
%       \tfr{1}{2}
%     \underset{(3^{+})}{\vphantom{(}
%       T_1T_2^2
%     }
%     +
%       \tfr{1}{3!}
%     \underset{(4^{+})}{\vphantom{(}
%       T_1^3T_2
%     }
%     )
%   \kt{\F}_{\text{c}}
% }}
% \\[3pt]
% \nonumber
%   \underset{(3^+)}{\vphantom{(}
%   t_{abcd}^{ijkl}
%   }
%   \underset{(0)}{\vphantom{(}
%   \mc{E}_{abcd}^{ijkl}
%   }
% =&\
%   \br{\F_{ijkl}^{abcd}}
%   \underset{(1)}{\vphantom{(}
%     V_{\text{c}}
%   }
%     (
%     \underset{(2^{+})}{\vphantom{(}
%       T_3
%     }
%     +
%     \underset{(3^{+})}{\vphantom{(}
%       T_4
%     }
%     +
%     \underset{(4^{+})}{\vphantom{(}
%       T_5
%     }
%     +
%     \underset{(5^{+})}{\vphantom{(}
%       T_6
%     }
%     +
%     \underset{(3^{+})}{\vphantom{(}
%       T_1T_3
%     }
%     +
%       \tfr{1}{2}
%     \underset{(2^{+})}{\vphantom{(}
%       T_2^2
%     }
%     +
%     \underset{(4^{+})}{\vphantom{(}
%       T_1T_4
%     }
%     +
%     \underset{(3^{+})}{\vphantom{(}
%       T_2T_3
%     }
%     +
%     \underset{(5^{+})}{\vphantom{(}
%       T_1T_5
%     }
%     +
%     \underset{(4^{+})}{\vphantom{(}
%       T_2T_4
%     }
%     +
%       \tfr{1}{2}
%     \underset{(4^{+})}{\vphantom{(}
%       T_3^2
%     }
% \\
% \underset{\displaystyle\vdots}{{}}\hspace{2pt}
% &
% \makebox[0.75\linewidth][r]{\ensuremath{
%     +
%       \tfr{1}{2}
%     \underset{(4^{+})}{\vphantom{(}
%       T_1^2T_3
%     }
%     +
%       \tfr{1}{2}
%     \underset{(3^{+})}{\vphantom{(}
%       T_1T_2^2
%     }
%     +
%       \tfr{1}{2}
%     \underset{(5^{+})}{\vphantom{(}
%       T_1^2T_4
%     }
%     +
%     \underset{(4^{+})}{\vphantom{(}
%       T_1T_2T_3
%     }
%     +
%       \tfr{1}{3!}
%     \underset{(3^{+})}{\vphantom{(}
%       T_2^3
%     }
%     +
%       \tfr{1}{3!}
%     \underset{(5^{+})}{\vphantom{(}
%       T_1^3T_3
%     }
%     +
%       \tfr{1}{2!2!}
%     \underset{(4^{+})}{\vphantom{(}
%       T_1^2T_2^2
%     }
%     )
%   \kt{\F}_{\text{c}}
% }}
% \\[3pt]
% \label{eq:cc-k-tuples-equation}
%   \underset{({k^+}{-}\,1)}{\vphantom{(}
%   t_{a_1\cd a_k}^{i_1\cd i_k}
%   }
%   \underset{(0)}{\vphantom{(}
%   \mc{E}_{a_1\cd a_k}^{i_1\cd i_k}
%   }
% =&\
%   \br{\F_{i_1\cd i_k}^{a_1\cd a_k}}
%   \underset{(1)}{\vphantom{(}
%     V_{\text{c}}
%   }
%     (
%     \underset{({k^+}{-}\,2)}{\vphantom{(}
%       T_{k-1}
%     }
%     +
%     \underset{({k^+}{-}\,1)}{\vphantom{(}
%       T_k
%     }
%     +
%     \hspace{2pt}
%     \underset{(k^+)}{\vphantom{(}
%       T_{k+1}
%     }
%     \hspace{1pt}
%     +
%     \underset{({k^+}{+}\,1)}{\vphantom{(}
%       T_{k+2}
%     }
%     +
%       \sum_{p=2}^4
%       \fr{1}{p!}
%       \sum_{h=p-2}^{2}
%       \sum_{\bm{k}}^{\mc{C}_p(k+h)}
%     \underset{({k^+}{+}\,h-p+\f_{\bm{k}})}{\vphantom{(}
%       T_{k_1}
%       \cd
%       T_{k_p}
%     }
%     )
%   \kt{\F}_{\text{c}}
% \end{align}
where $\mc{C}_k(m)$ denotes the set of $k$-part integer compositions of $m$
and $\f_{\bm{k}}$ is the number of 1's in
$
  \bm{k}
=
  (k_1,\ld,k_p)
$.
The orders of the cluster operators follow from straightforward induction on the fact that the lowest order contribution to $T_k$ always comes from
$
(
  V_{\text{c}}
  T_{k-1}
)_{\text{c}}
$,
which means that each $T_k$ contributes at one order above $T_{k-1}$, starting from $k=3$.
If Brillouin's theorem holds, $T_1$ has order $2^+$ and the orders of the disconnected products become ${k^+}{+}\,h-p+2\,\f_{\bm{k}}$.
\end{rmk}

\begin{prop}
\label{prop:cc-orders}
\thmstatement{
$\text{CCS}{\cd}m$ is correct to order $m-1$ in the wavefunction and order $m+\floor{m/2}$ in the energy.
}
\thmproof{
  According to \cref{rmk:cc-perturbative-analysis}, $T_{m+1}$ contributes at order $m$, implying that the wavefunction is correct to order $m-1$.
  Truncation also leaves $T_m$ and $T_{m-1}$ correct to order $m$, and propagating these truncation errors down to $T_{m-2h}$ and $T_{m-1-2h}$ makes the latter correct to $m+h$.
  One of these operators is $T_2$ when $h=\floor{m/2}-1$.
  Since $T_2$ limits the error in equation \ref{eq:cc-reference-equation}, the energy is correct to order $m+\floor{m/2}$.
}
\end{prop}

\begin{ex}
Props~\ref{prop:ci-orders} and \ref{prop:cc-orders} allow us to compare the accuracies of CI and CC.
At double excitations, CI and CC are both correct to first order in the wavefunction and third order in the energy.
Triples yield no improvement for CI, whereas CC gains an order in both wavefunction and energy.
In general, the CCS${\cd}m$ wavefunction and energy improve upon the wavefunction and energy obtained from
CIS${\cd}m$ by
$
  m
-
  \floor{m/2}
-
  1
=
  \floor{(m-1)/2}
$
orders in perturbation theory.
\end{ex}

\begin{dfn}
\thmtitle{Order $p$ truncation}
If $X$ is a polynomial in $T_1,T_2,\ld,T_n$, we define
its \textit{order $p$ truncation}, denoted $X\bord{p}$,
to include all terms in the polynomial with leading contributions of order $p$ or less.
This makes $X\bord{p}$ correct to order $p$ without isolating specific orders in the cluster operators, which will in general have infinite-order contributions.
\end{dfn}


\begin{samepage}
\begin{ex}
\thmtitle{The $[\text{T}]$ correction}
Assuming Brillouin's theorem, we can complete the energy to fourth order using
\begin{align}
  t_{ab}^{ij}
\,{=}\,
  \br{\F_{ij}^{ab}}
    R_0
    V_{\text{c}}
    (
      1
    \,{+}\,
      T_1
    \,{+}\,
      T_2
    \,{+}\,
      T_3^{[2]}
    \,{+}\,
      \tfr{1}{2}
      T_1^2
    \,{+}\,
      T_1T_2
    \,{+}\,
      \tfr{1}{2}
      T_2^2
    \,{+}\,
      \tfr{1}{3!}
      T_1^3
    \,{+}\,
      \tfr{1}{2}
      T_1^2T_2
    \,{+}\,
      \tfr{1}{4!}
      T_1^4
    )
  \kt{\F}_{\text{c}}
&&
  {}\bord{2}
  t_{abc}^{ijk}
\,{=}\,
  \br{\F_{ijk}^{abc}}
    R_0
    V_{\text{c}}
    T_2
  \kt{\F}_{\text{c}}
\end{align}
where the resulting energy correction is
$
  E
-
  E^{\text{CCSD}}
=
  \ip{\F|
    V_{\text{c}}
    R_0
    V_{\text{c}}
    T_3\bord{2}
  |\F}
$.
We can introduce additional infinite order contributions by noting that
$
  T_2
  \F
=
  R_0
  V_{\text{c}}
  \F
+
  \mc{O}(V_{\text{c}}^2)
$
and that the additional terms in
$
  \ip{\F|
    T_2\dg
    V_{\text{c}}
    T_3\bord{2}
  |\F}
$
are also valid energy contributions in perturbation theory.
There is no risk of double counting since all of these contributions involve connected triples, which are absent in CCSD.
With converged CCD or CCSD $T_2$-amplitudes, this defines the $[\text{T}]$ correction.
\begin{align}
  E_{[\text{T}]}
\equiv
  \ip{\F|T_2\dg V_{\text{c}} T_3\bord{2}|\F}
=
\diagram{
  \interaction{2}{2t}{(0,+0.5)}{ddot}{overhang};
  \interaction{3}{3t}{(0,-0.5)}{ddot}{overhang};
  \node[right=2pt of 3t3 ] {[2]};
  \draw[->-,bend left ] (3t1) to (2t1);
  \draw[-<-,bend right] (3t1) to (2t1);
  \draw[->-,bend left ] (3t2) to (2t2);
  \draw[-<-=0.25,-<-=0.75, bend right]
    (3t2)
    to
      node[ddot,midway] (g1) {}
    (2t2);
  \draw (2,0) node[ddot] (g2) {};
  \draw[sawtooth] (g1) to (g2);
  \draw[->-,bend left=40 ] (3t3) to (g2);
  \draw[-<-,bend right=40] (3t3) to (g2);
}\hspace{-8pt}
+
\diagram{
  \interaction{2}{2t}{(0,+0.5)}{ddot}{overhang};
  \interaction{3}{3t}{(0,-0.5)}{ddot}{overhang};
  \node[right=2pt of 3t3 ] {[2]};
  \draw[-<-,bend left ] (3t1) to (2t1);
  \draw[->-,bend right] (3t1) to (2t1);
  \draw[-<-,bend left ] (3t2) to (2t2);
  \draw[->-=0.25,->-=0.75, bend right]
    (3t2)
    to
      node[ddot,midway] (g1) {}
    (2t2);
  \draw (2,0) node[ddot] (g2) {};
  \draw[sawtooth] (g1) to (g2);
  \draw[-<-,bend left=40 ] (3t3) to (g2);
  \draw[->-,bend right=40] (3t3) to (g2);
}
&&&&
\diagram{
  \interaction{3}{t}{(0,-0.5)}{ddot}{overhang};
  \draw[->-] (t1) to ++(-0.25,1);
  \draw[-<-] (t1) to ++(+0.25,1);
  \draw[->-] (t2) to ++(-0.25,1);
  \draw[-<-] (t2) to ++(+0.25,1);
  \draw[->-] (t3) to ++(-0.25,1);
  \draw[-<-] (t3) to ++(+0.25,1);
  \node[right=2pt of t3 ] {[2]};
}\hspace{-10pt}
\equiv
\diagram{
  \interaction{2}{t}{(0,-0.5)}{ddot}{overhang};
  \draw[->-] (t1) to ++(-0.25,1);
  \draw[-<-] (t1) to ++(+0.25,1);
  \draw[->-] (t2) to ++(-0.25,1);
  \draw[-<-=0.25,-<-=0.85]
      (t2)
    to
      node[ddot,midway] (g1) {}
    ++(+0.25,1);
  \draw[sawtooth] (g1) to ++(1,0) node[ddot] (g2) {};
  \draw[->-=0.65] (g2) to ++(-0.25,0.5);
  \draw[-<-=0.65] (g2) to ++(+0.25,0.5);
  \draw[thick,flexdotted] (-0.3,0.25) to ++(2.7,0);
}
\,{+}\,
\diagram{
  \interaction{2}{t}{(0,-0.5)}{ddot}{overhang};
  \draw[-<-] (t1) to ++(-0.25,1);
  \draw[->-] (t1) to ++(+0.25,1);
  \draw[-<-] (t2) to ++(-0.25,1);
  \draw[->-=0.25,->-=0.85]
      (t2)
    to
      node[ddot,midway] (g1) {}
    ++(+0.25,1);
  \draw[sawtooth] (g1) to ++(1,0) node[ddot] (g2) {};
  \draw[-<-=0.65] (g2) to ++(-0.25,0.5);
  \draw[->-=0.65] (g2) to ++(+0.25,0.5);
  \draw[thick,flexdotted] (-0.3,0.25) to ++(2.7,0);
}
\end{align}
\end{ex}
\end{samepage}



\begin{rmk}
\label{rmk:lowdin-partitioning}
\thmtitle{The L\"owdin partitioning method}
Partitioning the determinant basis as
$
  \bo{\F}
=
\pma{\bo{\F}_{\text{i}}\ \bo{\F}_{\text{e}}}
$
where $\text{i}$ denotes the \textit{internal space} and $\text{e}$ denotes the \textit{external space}, we can express the identity on $\mc{F}_n$ as follows.
\begin{align}
  1_n
=
  1_{\text{i}}
+
  1_{\text{e}}
&&
\begin{array}{c@{\ }c@{\ }c@{\ }c@{\ }c}
  1_{\text{i}}
&
\equiv
&
  1|_{\text{i}}
&
=
&
  \kt{\bo{\F}_{\text{i}}}
  \br{\bo{\F}_{\text{i}}}
\\[5pt]
  1_{\text{e}}
&
\equiv
&
  1|_{\text{e}}
&
=
&
  \kt{\bo{\F}_{\text{e}}}
  \br{\bo{\F}_{\text{e}}}
\end{array}
\end{align}
This allows us to decompose the wavefunction as
$
  \Y
=
  \Y_{\text{i}}
+
  \Y_{\text{e}}
$
and the Hamiltonian as
$
  H
=
  H_{\text{ii}}
+
  H_{\text{ie}}
+
  H_{\text{ei}}
+
  H_{\text{ee}}
$,
with
$
  \Y_{\text{x}}
\equiv
  1_{\text{x}}
  \Y
$
and
$
  H_{\text{xy}}
\equiv
  1_{\text{x}}
  H
  1_{\text{y}}
$.\,\footnote{To see this, insert the identity in front of $\Y$ and on either side of $H$.}
This allows us to define the
\textit{L\"owdin resolvent}
$
  R_{\text{ee}}
\equiv
  \left.
  (
    E
  -
    H
  )^{-1}
  \right|_{\text{e}}
$,
which satisfies
\begin{align}
\label{eq:external-wavefunction-resolvent-expansion}
  R_{\text{ee}}
  (
    E
  -
    H
  )
=
-
  R_{\text{ee}}
  H_{\text{ei}}
+
  1_{\text{e}}
&&
  \implies
&&
  \Y_{\text{e}}
=
  R_{\text{ee}}
  H_{\text{ei}}
  \Y_{\text{i}}
\end{align}
where the first equation follows from inserting the identity and the second follows from substituting this result into $R_{\text{ee}}$ times the Schr\"odinger equation,
$
  (
    E
  -
    H
  )
  \Y
=
  0
$.
Projecting the Schr\"odinger equation by $1_{\text{i}}$ and rearranging then yields
\begin{align}
  (
    H_{\text{ii}}
  +
    V_{\text{ii}}
  )
  \Y_{\text{i}}
=
  E
  \Y_{\text{i}}
&&
  V_{\text{ii}}
=
  H_{\text{ie}}
  R_{\text{ee}}
  H_{\text{ei}}
\end{align}
where we have made use of equation~\ref{eq:external-wavefunction-resolvent-expansion}.
This sets up an effective Schr\"odinger equation for the exact energy in the internal space.
If $V_{\text{ii}}$ is treated as a perturbation, then the zeroth order states are eigenfunctions of the Hamiltonian in the basis of internal-space determinants.
Normalizing $\Y_{\text{i}}$ to one, the exact energy is given by the
\textit{L\"owdin functional}
\begin{align}
  E
=
  \ip{\Y_{\text{i}}|H|\Y_{\text{i}}}
+
  \ip{\Y_{\text{i}}|H|\bo{\F}_{\text{e}}}
  \ip{\bo{\F}_{\text{e}}|E - H|\bo{\F}_{\text{e}}}^{-1}
  \ip{\bo{\F}_{\text{e}}|H|\Y_{\text{i}}}
\end{align}
where we have expanded the resolvent in the basis of external-space determinants.
To first order, the energy correction is
% % ERROR
% \begin{align}
%   E
% -
%   E^{\{0\}}
% = \underset{E^{\{1\}}}{\underbrace{
%   \ip{\Psi_{\text{i}}^{\{0\}}|V_{\text{ii}}|\Psi_{\text{i}}^{\{0\}}}
% }}
% +
%   \mc{O}(V_{\text{ii}}^2)
% % ERROR
% &&
%   E^{\{1\}}
% =
%   \ip{\Psi_{\text{i}}^{\{0\}}|H|\bo{\Phi}_{\text{e}}}
%   \ip{\bo{\Phi}_{\text{e}}|E - H|\bo{\Phi}_{\text{e}}}^{-1}
%   \ip{\bo{\Phi}_{\text{e}}|H|\Psi_{\text{i}}^{\{0\}}}
%\end{align}
where curly braces denote orders in $V_{\text{ii}}$.
This gives the leading correction for truncating the wavefunction ansatz.
\end{rmk}

\begin{ex}
If we include all determinants up to $m$-fold substitutions in our internal space, then $H_{\text{ii}}$ is the $\text{CIS}{\cd}m$ Hamiltonian matrix and $\Y_{\text{CIS}{\cd}m}$ is the zeroth order approximation to $\Y_{\text{i}}$.
The L\"owdin partitioning method gives
\begin{align}
  E^{\{1\}}
=
  \ip{\F|(C_{m-1} + C_m)H|\bo{\F}_{m+1}\ \bo{\F}_{m+2}}
  \ip{\bo{\F}_{m+1}\ \bo{\F}_{m+2}|E - H|\bo{\F}_{m+1}\ \bo{\F}_{m+2}}^{-1}
  \ip{\bo{\F}_{m+1}\ \bo{\F}_{m+2}|H(C_{m-1} + C_m)|\F}
\end{align}
where we have used Slater's rules to eliminate matrix elements of $H$ where the determinants differ by more than two orbitals.
Noting that to zeroth order in the fluctuation potential
$
  E
-
  H
=
  E_{\text{c}}
-
  H_{\text{c}}
\approx
-
  H_0
$, we can approximate the L\"owdin resolvent by $R_0$.
This leads to the following correction to the $\text{CIS}{\cd}m$ energy
\begin{align}
  E^{\{1\}}
\approx
  (\tfr{1}{(m+1)!})^2
  \sum_{\substack{a_1\cd a_{m+1}\\i_1\cd i_{m+1}}}
  \fr{
    |\ip{\F_{i_1\cd i_{m+1}}^{a_1\cd a_{m+1}}|V_{\text{c}}\,(C_{m-1} + C_m)|\F}|^2
  }{
    \mc{E}_{a_1\cd a_{m+1}}^{i_1\cd i_{m+1}}
  }
+
  (\tfr{1}{(m+2)!})^2
  \sum_{\substack{a_1\cd a_{m+2}\\i_1\cd i_{m+2}}}
  \fr{
    |\ip{\F_{i_1\cd i_{m+2}}^{a_1\cd a_{m+2}}|V_{\text{c}}\,C_m|\F}|^2
  }{
    \mc{E}_{a_1\cd a_{m+2}}^{i_1\cd i_{m+2}}
  }
\end{align}
which can be viewed as a perturbative correction for $(m+1)$- and $(m+2)$-tuple substitutions.
\end{ex}

\begin{rmk}
Under the TCC similarity transformation, the L\"owdin partitioning correction has the following form.
\begin{align}
  E^{\{1\}}
=
  \ip{\ol{\Y}_{\text{i}}^{\{0\}}|\ol{H}|\bo{\F}_{\text{e}}}
  \ip{\bo{\F}_{\text{e}}|E - \ol{H}|\bo{\F}_{\text{e}}}^{-1}
  \ip{\bo{\F}_{\text{e}}|\ol{H}|\ol{\Y}_{\text{i}}^{\{0\}}}
&&
\begin{array}{r@{\ }l}
  \kt{\ol{\Y}_{\text{i}}^{\{0\}}}
&=
  \text{exp}(-T)\,\kt{\Y_{\text{i}}^{\{0\}}}
\\
  \br{\ol{\Y}_{\text{i}}^{\{0\}}}
&=
  \br{\Y_{\text{i}}^{\{0\}}}\,
  \text{exp}(T)
\end{array}
\end{align}
Using
$
  1_{\text{i}}(E + \ol{H}_0) 1_{\text{e}}
=
  0
$
and
$
  1_{\text{e}}(E + \ol{H}_0)\kt{\F}
=
  0
$
and substituting in the EOM-CCS${\cd}m$ states gives the following.
\begin{align}
  E^{\{1\}}
=
  \ip{\F|(L_{m-1} + L_m)\,(V_{\text{c}}\,\text{exp}(T))_{\text{c}}|\bo{\F}_{\text{e}}}
  \ip{\bo{\F}_{\text{e}}|E - \ol{H}|\bo{\F}_{\text{e}}}^{-1}
  \ip{\bo{\F}_{\text{e}}|(V_{\text{c}}\,\text{exp}(T))_{\text{c}} R|\F}
\end{align}
where we have used the fact that the excitation level of $\bm{\F}_{\text{e}}$ is at least $m+1$ and that of $\ol{V}_{\text{c}}$ is $-2$ or higher to eliminate the contributions to $L$ that cannot form complete contractions.
This gives the first-order L\"owdin partitioning correction for ground and excited state energies obtained with a truncated TCC Ansatz.
\end{rmk}





\begin{ex}
\thmtitle{The $(m+1)_\La$ correction}
Approximating the L\"owdin resolvent by the model resolvent gives
\begin{align}
  E^{\{1\}}
\approx
  \ip{\F|(\La_{m-1} + \La_m)V_{\text{c}}\,\text{exp}(T)|\bo{\F}_{\text{e}}}_{\text{c}}
  \ip{\bo{\F}_{\text{e}}|R_0 V_{\text{c}}\,\text{exp}(T)|\F}_{\text{c}}
\end{align}
for the ground state, which effectively ``dots'' the lambda equations against the amplitude equations for excitation level $m+1$ and higher with $\La$ and $T$ truncated at $m$-fold excitations.
From \cref{rmk:cc-perturbative-analysis}, the leading contribution to the $(m+1)$-tuples amplitude equations  has order $m$.
The vector of order-$m$-truncated $(m+1)$-tuples amplitudes is given by\footnote{$\bm{k}\not\ni 1$ here means that integer compositions containing 1 are excluded.}
\begin{align}
  \bo{t}_{m+1}\bord{m}
=
  \br{\bo{\F}_{m+1}}R_0
    V_{\text{c}}
    (
      T_m
    +
      \sum_{p=2}^4
      \fr{1}{p!}
      \sum_{\bm{k}\not\ni 1}^{\mc{C}_p(m+p-1)}
      T_{k_1}
      \cd
      T_{k_p}
    )
  \kt{\F}_{\text{c}}\,
\end{align}
using equation~\ref{eq:cc-k-tuples-equation}.
The leading contribution to the $(m+1)$-tuples lambda equations comes from $\text{exp}(T)\bord{0}= 1$.
This approximation to $E^{\{1\}}$ defines the
$
  (m+1)_\La
$
correction
which,
using
$
  \kt{\bo{\F}_{\text{m+1}}}\cdot
  \bo{t}_{m+1}\bord{m}
=
  T_{m+1}\bord{m}
$, can be written as follows
\begin{align}
  E_{(m+1)_\La}
=
  \ip{\F|(\La_{m-1} + \La_m) V_{\text{c}} T_{m+1}\bord{m}|\F}\,\,,
\end{align}
forming a well-defined hierarchy of perturbatively-corrected CC methods:
CCSD(T)$_\La$,
CCSDT(Q)$_\La$,
CCSDTQ(P)$_\La$,
etc.
\end{ex}


\begin{ex}
\thmtitle{The $(\text{T})$ correction}
Since the leading contributions to $\La_1$ and $\La_2$ are given by
\begin{align}
  {}\ord{1}\la_{ij}^{ab}
=
  \br{\F}
  V_{\text{c}}
  R_0
  \kt{\F_{ij}^{ab}}
=
  {}\ord{1}t_{ab}^{ij*}
&&
  {}\ord{2}\la_i^a
=
  \br{\F}
  \La_2\ord{1}
  V_{\text{c}}
  R_0
  \kt{\F_i^a}
=
  {}\ord{2}t_a^{i*}\,\,,
\end{align}
the approximation $\La_1\approx T_1\dg, \La_2\approx T_2\dg$ will capture the leading terms in the (T)$_\La$ correction plus many higher-order ones.
The resulting $(\text{T})$ correction augments the [T] correction with an order $5^+$ contribution,
$
  \br{\F}
    T_1\dg
    V_{\text{c}}
    T_3\bord{2}
  \kt{\F}
$.
Though less rigorous than the (T)$_\La$ correction, (T) is computationally advantageous in that it avoids solution of the $\La$ equations.
\begin{align}
  E_{(\text{T})}
=
  \br{\F}
    (
      T_1\dg
    +
      T_2\dg
    )
    V_{\text{c}}
    T_3\bord{2}
  \kt{\F}
=
\diagram{
  \interaction{1}{1t}{(0,+0.5)}{ddot}{overhang};
  \interaction{2}{g}{(1,+0.5)}{ddot}{sawtooth};
  \interaction{3}{3t}{(0,-0.5)}{ddot}{overhang};
  \node[right=2pt of 3t3 ] {[2]};
  \draw[->-,bend left ] (3t1) to (1t1);
  \draw[-<-,bend right] (3t1) to (1t1);
  \draw[->-,bend left ] (3t2) to (g1);
  \draw[-<-,bend right] (3t2) to (g1);
  \draw[->-,bend left ] (3t3) to (g2);
  \draw[-<-,bend right] (3t3) to (g2);
}\hspace{-8pt}
+
\underset{\displaystyle E_{[\text{T}]}}{\underbrace{
\diagram{
  \interaction{2}{2t}{(0,+0.5)}{ddot}{overhang};
  \interaction{3}{3t}{(0,-0.5)}{ddot}{overhang};
  \node[right=2pt of 3t3 ] {[2]};
  \draw[->-,bend left ] (3t1) to (2t1);
  \draw[-<-,bend right] (3t1) to (2t1);
  \draw[->-,bend left ] (3t2) to (2t2);
  \draw[-<-=0.25,-<-=0.75, bend right]
    (3t2)
    to
      node[ddot,midway] (g1) {}
    (2t2);
  \draw (2,0) node[ddot] (g2) {};
  \draw[sawtooth] (g1) to (g2);
  \draw[->-,bend left=40 ] (3t3) to (g2);
  \draw[-<-,bend right=40] (3t3) to (g2);
}\hspace{-8pt}
+
\diagram{
  \interaction{2}{2t}{(0,+0.5)}{ddot}{overhang};
  \interaction{3}{3t}{(0,-0.5)}{ddot}{overhang};
  \node[right=2pt of 3t3 ] {[2]};
  \draw[-<-,bend left ] (3t1) to (2t1);
  \draw[->-,bend right] (3t1) to (2t1);
  \draw[-<-,bend left ] (3t2) to (2t2);
  \draw[->-=0.25,->-=0.75, bend right]
    (3t2)
    to
      node[ddot,midway] (g1) {}
    (2t2);
  \draw (2,0) node[ddot] (g2) {};
  \draw[sawtooth] (g1) to (g2);
  \draw[-<-,bend left=40 ] (3t3) to (g2);
  \draw[->-,bend right=40] (3t3) to (g2);
}
}}
\end{align}
The CCSD(T) method was originally justified on different grounds, but its truly remarkable performance is difficult to understand using ordinary perturbation theory.
Several other fifth-order terms could just as easily be included, and there is no strong \emph{a priori} reason to expect this one to be as magical as it is.
L\"owdin partitioning suggests that $E_{(\text{T})}$ is the leading correction to the error incurred by truncating the configuration space at doubles.
\end{ex}

